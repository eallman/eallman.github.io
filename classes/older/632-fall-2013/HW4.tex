\documentclass[11pt]{report}
\usepackage{amsmath,amssymb}
%\usepackage{pdfsync}

\setlength{\oddsidemargin}{0in} \setlength{\textwidth}{6.5in}
\setlength{\topmargin}{-.5in} \setlength{\textheight}{9in}

%\renewcommand{\familydefault}{\sfdefault}

\newcommand\CC{{\mathbb C}}
\newcommand\RR{{\mathbb R}}
\newcommand\QQ{{\mathbb Q}}
\newcommand\NN{{\mathbb N}}
\newcommand\ZZ{{\mathbb Z}}
\newcommand\A{{\mathbb A}}
\newcommand\image{\operatorname{Image}}
\newcommand\lex{>_{\operatorname{lex}}}
\newcommand\grlex{>_{\operatorname{grlex}}}
\newcommand\grevlex{>_{\operatorname{grevlex}}}
\newcommand\LT{\operatorname{LT}}
\newcommand\LM{\operatorname{LM}}
\newcommand\LC{\operatorname{LC}}
\newcommand\modI{\operatorname{mod} I}

\newcommand\dsp{\displaystyle}

\begin{document}

\enlargethispage{.5cm}

\noindent {\LARGE \sc HW 4 problems}

\vskip 1cm

\begin{enumerate}

\item  Hassett, Chapter 2, \#12, modified as follows:

\begin{enumerate}

\item Before beginning, read the entire problem carefully and
then consider a particular instance of the algebra homomorphism $\psi$ in problem 12:

\smallskip

Consider the ring $k[x,y_1,y_2]$ and $\phi_1(x) = x$, $\phi_2(x) = x^2$.  

\begin{enumerate}

\item Carefully define the map $\psi$ in this example.

\item Find the image of $f = x + y_1 - 2y_2$ and $g = xy_1 - y_2$ under $\psi$.

\item Is $\psi$ surjective?  Prove your answer for the general case.

\item By direct computation, show that $y_i - \phi_i \in \ker (\psi)$.

\item By direct computation, show that $g \in I$ as defined in the problem.

\end{enumerate}

\item Now do problem 12.

\end{enumerate}

\item 

\begin{enumerate}

\item Hassett, Chapter 2, \#17a.  You may (and are strongly encouraged to) use {\tt Singular} to answer this question.
Commands you will need are {\tt matrix, minor, print(A)}.

\item View any $2 \times 3$ matrix $A = (a_{ij})$ as an element of $\A^6(\RR)$.   Then consider
the set of points $V \subseteq \A^6(\RR)$ satisfying the three equations $g_1, g_2, g_3$.  That is,
$V$ is the \emph{zero set} of $g_1, g_2, g_3$.

\smallskip

Using ideas from linear algebra, give a concrete description of this set $V$.

\end{enumerate}

\item 

\begin{enumerate}

\item By hand, compute a Groebner basis for the ideal $I = \langle f_1, f_2 \rangle =
\langle x^2-y^2, xy-1 \rangle$ with respect to $\lex$ for $x > y$.
For your write up, please include all $S$-polynomials.

\item You should be able to find a smaller (in terms of number)
Groebner basis by taking only a subset of your answer to (a).  More formally,
a Groebner basis $f_1, \dots, f_r$  is called \emph{minimal} 
if for all $i$, 
$$
< \LT(f_1), \dots, \LT(f_{i-1}), \LT(f_{i+1}), \dots, \LT(f_r)\rangle \, \neq \,
< \LT(f_1), \dots, \LT(f_r)\rangle.
$$
Give a minimal Grobner basis for $I$ constructed from your answer to (a).

\item Give a concrete description of the zero set of the polynomials $f_1$ and $f_2$.  

\end{enumerate}

\item {\tt Singular} exercises.

\begin{enumerate}

\item Determine whether $f = xy^3 -z^2 +y^5-z^3$ and 
$g =-z^4yx-z^4y+z^4+z^2y^2+z^2yx+zy^2x+zy^2-zy-yx-x^2$
are in the ideal $I = \langle -x^3 + y, x^2y-z \rangle$.  If not,
give the normal form $(\modI)$ with respect to 
$\grevlex$.

\item (Calculus III) Consider the function of two variables
$$
f(x,y) = x^3y^2+x^2y^3+x^2y+xy^2.
$$
Use {\tt Singular} to compute the critical points of $f$.  You will
need the command {\tt diff}, and once you have computed a
Groebner basis $G$ for the appropriate ideal, you will need the 
commands
\begin{verbatim}
              LIB "solve.lib";
              solve G;
\end{verbatim}

\item (Lagrange multiplier problem -- easy) \,   Find the maximum and minimum values of
$f(x,y) = 2x^2 + y^2$ subject to the constraint $g(x,y) = x^2 + y^2 - 9 = 0$.  Do this
by hand and by using {\tt Singular}.


\end{enumerate}


\end{enumerate}


\end{document}

