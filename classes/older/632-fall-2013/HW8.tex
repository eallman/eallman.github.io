\documentclass[11pt]{report}
\usepackage{amsmath,amssymb}
%\usepackage{pdfsync}

\setlength{\oddsidemargin}{0in} \setlength{\textwidth}{6.5in}
\setlength{\topmargin}{-.5in} \setlength{\textheight}{9in}

%\renewcommand{\familydefault}{\sfdefault}

\newcommand\CC{{\mathbb C}}
\newcommand\RR{{\mathbb R}}
\newcommand\QQ{{\mathbb Q}}
\newcommand\NN{{\mathbb N}}
\newcommand\ZZ{{\mathbb Z}}
\newcommand\A{{\mathbb A}}
\newcommand\image{\operatorname{Im}}
\newcommand\lex{>_{\operatorname{lex}}}
\newcommand\grlex{>_{\operatorname{grlex}}}
\newcommand\grevlex{>_{\operatorname{grevlex}}}
\newcommand\LT{\operatorname{LT}}
\newcommand\LM{\operatorname{LM}}
\newcommand\LC{\operatorname{LC}}
\newcommand\modI{\operatorname{mod} I}
\newcommand\rank{\operatorname{rank}}

\newcommand\dsp{\displaystyle}

\begin{document}

\enlargethispage{.5cm}

\noindent {\LARGE \sc HW 8 problems}

\vskip 1cm

\begin{enumerate}

\item Chapter 3, \# 21

\item Prove that $V \subseteq \A^n(k)$ is irreducible if, and only if, $I(V)$ is a prime ideal.

\item Consider the $V \subseteq \A^3(\CC)$ defined by the three quadric equations:
\begin{center}
\begin{tabular}{rll}
$f_1$ & $\, = \, 2xz+2y^2+3y + z^2$ &\, = \, 0\\
$f_2$ & $\, = \, x+yz+2z$ &\, = \, 0\\
$f_3$ & $\, = \, xz+y^2+2y$ &\, = \, 0
\end{tabular}
\end{center}

Prove that $V$ is isomorphic to $\A^1(\CC)$.  Show explicitly that the coordinate rings
$k[V]$ and $k[\A^1(\CC)]$ are isomorphic.

\emph{Hint:}  Use Singular and an `appropriate' term order to find a Groebner basis for 
$I(V)$.

\item Let $C$ denote the twisted cubic in $\A^3(\RR)$.  Prove that the maps
$\phi_1, \phi_2: \A^3(\RR) \to \A^2(\RR)$ given below
define the same morphism from $C \to A^2(\RR)$.
\begin{align*}
\phi_1(x,y,z) &= (2x^2 + y^2, \, z^2-y^3+3xz),\\
 \phi_2(x,y,z) &= (2y+xz, \, 3y^2).
\end{align*}

\item Consider the morphism $\phi: \A^2(\RR) \to \A^5(\RR)$ defined
by 
$$
(u,\, v) \mapsto (u,\, v,\, u^2,\, uv, \,v^2).
$$
The image of this map is closed and called the \emph{Veronese surface},
$V =\image(\phi) = \overline{\image(\phi)}$.

\begin{enumerate}

\item Find an implicit description of $V$.

\item Prove or disprove: $V \cong \A^2(\RR)$.  Is $V$ rational?

\end{enumerate}

\item In this problem, we will show that the affine line $\A^1(\RR)$ is \emph{not}
isomorphic to the `non-singular' cubic $V = V(y^2 - x^3 + x)$. Note that $V$ is an example
of an \emph{elliptic curve} (or at least gives the $\RR$-points on an elliptic curve).

\begin{enumerate}

\item Sketch a graph of $V$.  (Be quick with this; feel free to use software.)

\item Suppose $\phi: \A^1(\RR) \to V$ is given by $t \mapsto (a(t), b(t))$.  Explain
why it must be true that ${b(t)}^2 = a(t)\big(a(t)^2 - 1) \big)$.

\item Viewing $a(t)\big(a(t)^2 - 1) \big) \in \RR[t]$, explain why the two factors must
be relatively prime.

\item Using the unique factorizations of $a(t)$ and $b(t)$ into products of irreducible
polynomials, show that $b^2 = ac^2$ for some polynomial $c(t) \in \RR[t]$.

\item From the last part, it follows that $c^2 = a^2 - 1$.  Deduce from this equation
that $c$, $a$, and, hence, $b$ must be constant polynomials, and that $V$
is not isomorphic to the affine line.

\end{enumerate}

\item Let $V \subseteq A^n(k)$ be a hypersurface defined by the single equation
$x_n - f(x_1, \dots, x_{n-1}) = 0$. Show that $V$ is isomorphic to $A^{n-1}(k)$.

\item Consider the variety $V = V(y^3 - x^2) \subseteq \A^2(\RR)$.  

\begin{enumerate}

\item Show that $y^3 - x^2$ is irreducible in $\RR[x,y]$.  Then conclude that
$V$ is irreducible, and $\RR[V]$ is an integral domain, and $\RR(V)$ is a field.

\item In \emph{one sentence}, explain why problem 3 from the take-home 
part of your midterm shows that $V$ is not isomorphic to $\A^1(\RR)$.

\item Using the term order $\lex$ with $x > y$ for polynomials
in $\RR[x,y]$, then the coordinate ring of $V$ is
$$
\RR[V] = \{ a(y) + x \,  b(y) \mid a(y), \, b(y) \in \RR[y]\}.
$$

\begin{enumerate}

\item Justify that $\RR[V]$ has the form claimed above.

\item Define multiplication for elements in $\RR[V]$.

\end{enumerate}

\item Give an explicit description of the elements of the field of rational
functions $\RR(V)$ as follows.

\begin{enumerate}

\item Suppose $0 \neq c + x \, d \in \RR[V]$, compute
$\dsp \frac{a + x \, b}{ c + x \, d} = \bigg( \frac{a + x \, b}{ c + x \, d} \bigg) \bigg(\frac{c - x \, d}{ c - x \, d} \bigg)$.

\smallskip

\item From i, conclude that $\RR(V) = \RR(y) + x \,  \RR(y)$.

\end{enumerate}

\item Now show that $V$ is rational by 

\begin{enumerate}

\item explicitly giving an isomorphism of their 
rational function fields (\emph{i.e.} show $k(V) \cong k(\A^1(\RR))$).

\item if you have not done so in the last part, then explicitly give rational
maps $\rho: \A^1(\RR) \dasharrow V$ and $\psi: V \dasharrow \A^1(\RR)$ that correspond 
to the maps of function fields from part i.  On what open sets $U$ are these maps defined?
Show informally that $\phi \circ \rho: \A^1(\RR) \dasharrow \A^1(\RR)$ and 
$\rho \circ \psi: V \dasharrow V$ are defined at some points of their domain and
are the identity on these points.

\end{enumerate}

\item Finally, make sure that you understand the point of this problem:  Give a summary of the
main conclusion.

\end{enumerate}

\item Consider the rational maps $\rho: \A^1(\RR) \dasharrow \A^3(\RR)$ and 
$\psi: \A^3(\RR) \dasharrow \A^1(\RR)$ given by
$$
\rho(t) = \big(t, \frac{1}{t}, t^2\big) \text{ and } \psi(x,y,z) = \frac{x+yz}{x-yz}.
$$
Show that $\psi \circ \rho$ is not defined at any points of $\A^1(\RR)$.
(\emph{Moral}: Compositions of rational maps may not be defined.)


\item Chapter 3, \# 25, modified as follows:

\begin{enumerate}

\item Sketch (or somehow get an image in your mind of) the surface $xyz-1=0$.  We will call this
surface $S$.

\item Outline the steps you would follow to prove that $S = V(xyz-1)$ is rational.   Use
Proposition 3.57 for your outline.   (There is one VERY HARD step which you should point out.)

\item For an slightly alternative proof that $S$ is rational, try to show that $k(S) \cong k(u,v)$ directly.

\begin{enumerate}

\item Give an isomorphism of $k[S]$ with $k[x,y,\frac{1}{xy}]$.  (Note that $k[x,y,\frac{1}{xy}]$
is the localization of $k[x,y]$ with respect to the multiplicatively closed set $\{1, \frac{1}{xy},  
\frac{1}{(xy)^2}, \dots\}.)$

\item Convince yourself by proving (or at the very least justifying without proof) that $k[S]$ is an
integral domain.

\item Compute the field $k(S)$ by computing the quotient field of $k[x,y,\frac{1}{xy}]$.

\item Now show that $k(S) \cong k(u,v) = k(\A^2)$.

\end{enumerate}

\end{enumerate}

\end{enumerate}

\end{document}

