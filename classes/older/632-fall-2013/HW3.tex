\documentclass[11pt]{report}
\usepackage{amsmath,amssymb}
%\usepackage{pdfsync}

\setlength{\oddsidemargin}{0in} \setlength{\textwidth}{6.5in}
\setlength{\topmargin}{-.25in} \setlength{\textheight}{9in}

%\renewcommand{\familydefault}{\sfdefault}

\newcommand\CC{{\mathbb C}}
\newcommand\RR{{\mathbb R}}
\newcommand\QQ{{\mathbb Q}}
\newcommand\NN{{\mathbb N}}
\newcommand\ZZ{{\mathbb Z}}
\newcommand\A{{\mathbb A}}
\newcommand\image{\operatorname{Image}}
\newcommand\lex{>_{\operatorname{lex}}}
\newcommand\grlex{>_{\operatorname{grlex}}}
\newcommand\grevlex{>_{\operatorname{grevlex}}}


\newcommand\dsp{\displaystyle}

\begin{document}

\noindent {\LARGE \sc HW 3 problems}

\vskip 1cm

\begin{enumerate}

\item[1-7.]  Hassett, Chapter 2, \#1, 2, 3 (Note: the $c_\beta$ are non-zero elements of $k$.), 4a-c, 6 (Note:  this shows
actually that monomial ideals in $\CC[x,y]$ are finitely generated, and by extension that monomial ideals in 
$\CC[x_1, \dots, x_n]$ are finitely generated.) 7, 8.

\item[8.] Hassett, Chapter 2, Application of number 11.

Digest the definition of a reduced Groebner basis.  Then consider the affine variety defined by the 
following system of linear equations, where the rows correspond to the equations $f_i = 0$, for $i = 1,2,3$.

\begin{center}
\begin{tabular}{ccccrccl}
$x$  &+ &$y$ &+ &$z$ &&= &0\\
&& $y$ &+ & $2z$ & + $w$ &= &0\\
&&$y$& +&$z$ &+ $w$ &= &0.
\end{tabular}
\end{center}

\begin{enumerate}

\item Give a Groebner basis for the ideal $I = \langle f_1, f_2, f_3 \rangle$ with respect to $\lex$.

\item Give a reduced Groebner basis for the ideal $I = \langle f_1, f_2, f_3 \rangle$ with respect to $\lex$.

\item Give a geometric description of the variety defined by the vanishing of these three polynomials.  Also
give a parametric description of this variety.
\end{enumerate}

\item[9,10.]  Hassett, Chapter 2, \#13, 14
\end{enumerate}

\end{document}

