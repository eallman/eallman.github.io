\documentclass[11pt]{report}
\usepackage{amsmath,amssymb}
%\usepackage{pdfsync}

\setlength{\oddsidemargin}{0in} \setlength{\textwidth}{6.5in}
\setlength{\topmargin}{-.5in} \setlength{\textheight}{9in}

%\renewcommand{\familydefault}{\sfdefault}

\newcommand\CC{{\mathbb C}}
\newcommand\RR{{\mathbb R}}
\newcommand\QQ{{\mathbb Q}}
\newcommand\NN{{\mathbb N}}
\newcommand\ZZ{{\mathbb Z}}
\newcommand\A{{\mathbb A}}
\newcommand\image{\operatorname{Im}}
\newcommand\lex{>_{\operatorname{lex}}}
\newcommand\grlex{>_{\operatorname{grlex}}}
\newcommand\grevlex{>_{\operatorname{grevlex}}}
\newcommand\LT{\operatorname{LT}}
\newcommand\LM{\operatorname{LM}}
\newcommand\LC{\operatorname{LC}}
\newcommand\modI{\operatorname{mod} I}
\newcommand\rank{\operatorname{rank}}

\newcommand\dsp{\displaystyle}

\begin{document}

\enlargethispage{.5cm}

\noindent {\LARGE \sc HW 6 problems}

\vskip 1cm

\begin{enumerate}


\item[1-8.] Hassett, Chapter 3, \#6, 7, 8, 9, 11, 12, 14, 15

\setcounter{enumi}{8}
\item Prove that any bijective function $f: \CC \to \CC$ is Zariski continuous.

\item Consider the complex affine line $\A^1(\CC)$.  Show that any open subset $U \subset \A^1(\CC)$
is dense.  Then show that the Zariski topology on $\A^1(\CC)$ is not Hausdorff.

\item Give the Zariski closure of the interval $(0,1) \subset \A^1(\RR)$.

\item 

\begin{enumerate}

\item Let $V \subset \A^2(\RR)$ be the variety $V(y-x^2)$.  Show that the coordinate ring
$\RR[V]$ is isomorphic to a polynomial ring in one variable over $\RR$.

\item Let $Z \subset \A^2(\RR)$ be the variety $V(xy - 1)$.  Show that the coordinate ring
$\RR[Z]$ is \emph{not} isomorphic to a polynomial ring in one variable over $\RR$.  What does
this say about $\A^1 (\RR)$ and $Z$?

\end{enumerate}

\item Some ring theory problems:  Let $f: R \to S$ be a ring homomorphism.  Prove the following

\begin{enumerate}

\item If $J \subset S$ is an ideal of $S$, then $I = f^{-1}(J)$ is an ideal of $R$.  $I$ is called the
\emph{contraction} of $J$.

\item If $J \subset S$ is a prime ideal, show that its contraction is $R$ or a prime ideal of $R$.

\item If $I \subset R$ is an ideal, show that its \emph{extension} $J = f(I) S$ is an ideal of $S$.

\item Give an example to show that if $I$ is a prime ideal of $R$, then its extension may not
be a prime ideal of $S$.

\end{enumerate}

\end{enumerate}

\end{document}

