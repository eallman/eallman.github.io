\documentclass[11pt]{report}
\usepackage{amsmath,amssymb}
%\usepackage{pdfsync}

\setlength{\oddsidemargin}{0in} \setlength{\textwidth}{6.5in}
\setlength{\topmargin}{-.5in} \setlength{\textheight}{9in}

%\renewcommand{\familydefault}{\sfdefault}

\newcommand\CC{{\mathbb C}}
\newcommand\RR{{\mathbb R}}
\newcommand\QQ{{\mathbb Q}}
\newcommand\NN{{\mathbb N}}
\newcommand\ZZ{{\mathbb Z}}
\newcommand\A{{\mathbb A}}
\newcommand\image{\operatorname{Im}}
\newcommand\lex{>_{\operatorname{lex}}}
\newcommand\grlex{>_{\operatorname{grlex}}}
\newcommand\grevlex{>_{\operatorname{grevlex}}}
\newcommand\LT{\operatorname{LT}}
\newcommand\LM{\operatorname{LM}}
\newcommand\LC{\operatorname{LC}}
\newcommand\modI{\operatorname{mod} I}
\newcommand\rank{\operatorname{rank}}

\newcommand\dsp{\displaystyle}

\begin{document}

\enlargethispage{.5cm}

\noindent {\LARGE \sc Final Exam take home problems}

\vskip .5cm

\noindent {\bf Instructions:}  These problems are due on Tuesday, December 17 at the beginning
of the in-class exam.  All work must be your own, and only your own.  You may consult
class notes, Hassett's book, and the Cox, Little, O'Shea chapters.  Each problem is
worth ten points.  Good luck.

\begin{enumerate} 

\item Hassett Chapter 6, \# 10.  For part (b), give the sets $U_1 \subset V_1$ and $U_2 \subset V_2$
on which the birational maps are defined.

\item Hassett Chapter 6, \# 11.  Then look up the statement of the Chinese Remainder theorem, and 
comment succinctly about its relation to this problem, if any.

\item Hassett Chapter 7, \# 7 and 8.  For 7, clearly indicate the dimension of the variety in
your answer.  If you are unsure of the dimensions of the varieties at first, 
then you might look up the command {\tt dim} in Singular for help.

\item Hassett Chapter 9, \# 6 $a, b, c$.  If you are interested in using {\tt Singular}, then
the commands like {\tt homog(f,w)} are useful.

\item Hassett Chapter 9, \# 16 modified as follows.  

\begin{enumerate}

\item Why is the definition of the map $\nu(2)$ given in Proposition 9.38 equivalent to
the one given in Problem 16.  Use your knowledge of the vector space of homogeneous
polynomials of degree equal to $2$ to answer.  (Your answer should be brief.)

\item Now do problem 16.  The challenge is optional.

\end{enumerate}

\item 

\begin{enumerate}

\item Cox, Little, O'Shea p. 354, \# 3

\item Cox, Little, O'Shea p. 354, \# 4 $a$-$d$

\end{enumerate}

\item Carefully read pp. 136, 137 in Hassett.  Give a birational equivalence between
the unit circle $C = V(x^2 + y^2 - 1)$ and the hyperbola $V = V(y^2 - z^2 + 1)$.  
\emph{Hint:} Use the map $\rho_{20}$ or $\rho_{02}$ and the last problem, noting
a slight change in orientation of the cone.  It may be helpful (or not) to consider
the variables $x$, $y$, and $z$ as `dummy' variables.  See also 8a in Cox, Little, O'Shea.

Your solution here should be short (please).

\item Cox, Little, O'Shea p. 354, \# 11 $a, b, c$.

\item Cox, Little, O'Shea p. 364, \# 8 $b$.

\item Cox, Little, O'Shea p. 364, \# 14 $a, b$. (Do not do answer question about the singular point.  Just homogenize.)

\end{enumerate}

\smallskip 

\noindent \emph{Extra Credit:}  Read or skim as appropriate the handout on elliptic curves that
is available on the class website. 
Consider the real elliptic curve $C$ given by $y^2 = x^3 + 17$.   

\begin{enumerate}

\item Sketch $C$.

\item p.~36, 18a

\item p.~36, 18b.

\end{enumerate}

\

\noindent {\bf Note:} For the in-class final exam, you should be able to do Chapter 4, \# 14.

\end{document}

