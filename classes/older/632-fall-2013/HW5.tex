\documentclass[11pt]{report}
\usepackage{amsmath,amssymb}
%\usepackage{pdfsync}

\setlength{\oddsidemargin}{0in} \setlength{\textwidth}{6.5in}
\setlength{\topmargin}{-.5in} \setlength{\textheight}{9in}

%\renewcommand{\familydefault}{\sfdefault}

\newcommand\CC{{\mathbb C}}
\newcommand\RR{{\mathbb R}}
\newcommand\QQ{{\mathbb Q}}
\newcommand\NN{{\mathbb N}}
\newcommand\ZZ{{\mathbb Z}}
\newcommand\A{{\mathbb A}}
\newcommand\image{\operatorname{Im}}
\newcommand\lex{>_{\operatorname{lex}}}
\newcommand\grlex{>_{\operatorname{grlex}}}
\newcommand\grevlex{>_{\operatorname{grevlex}}}
\newcommand\LT{\operatorname{LT}}
\newcommand\LM{\operatorname{LM}}
\newcommand\LC{\operatorname{LC}}
\newcommand\modI{\operatorname{mod} I}
\newcommand\rank{\operatorname{rank}}

\newcommand\dsp{\displaystyle}

\begin{document}

\enlargethispage{.5cm}

\noindent {\LARGE \sc HW 5 problems}

\vskip 1cm

\begin{enumerate}

\item[1-6.]  Hassett, Chapter 3, \#1-6a.  For 6a, only show that every finite subset $S \subset \A^n(k)$ is
a variety.

\setcounter{enumi}{6}

\item  The `twisted cubic' in $\A^3(\RR)$ is the curve $C$ defined by the vanishing
of the polynomials $y - x^2$ and $z - x^3$.  We have (or almost have) an
alternative description of this variety given parametrically; that is, $C$ is
(the closure of) the image of the morphism
\begin{align*}
\phi: \,  \A^1(\RR) &\to \A^3(\RR)\\
t &\mapsto (t, t^2,t^3).
\end{align*}
Noting that $\image (\phi) = \{(t, t^2, t^3) \mid t \in \RR\}$, prove that
$I = I (\image(\phi)) = \langle y - x^2, \, z-x^3 \rangle$.

\smallskip

\emph{Hint}:  One inclusion is easy.  For the other, using some
appropriate monomial ordering divide any $f \in I$ by
$y-x^2$ and $z-x^3$ and consider the remainder term.

\item Generalize the last problem:  Fix $d \in \ZZ^+$ and consider the curve
in $\A^d(\RR)$ given by the image of the morphism:
\begin{align*}
\phi_d : \A^1(\RR) &\to \A^d(\RR)\\
t &\mapsto (t, t^2, \dots, t^d).
\end{align*}
In slightly modified form, this is usually considered a \emph{projective} variety, but we will still
informally call this affine version the `rational normal curve.'

\smallskip

Formulate and prove a generalization of problem 1 for the rational normal curve.

\item Let $V \subseteq \A^6(\RR)$ be the variety of $2 \times 3$ matrices of rank at most $1$.  
That is,
if $A \in V$, then $\rank (A) \le 1$.  Let $I \subset \RR[a_{11}, \dots, a_{23}]$ be the ideal
generated by two polynomials $f_1, f_2$:
$$
I = \langle a_{11} a_{22} - a_{12} a_{21},  a_{12} a_{23} - a_{13} a_{22} \rangle.
$$

\smallskip

Show explicitly that $V \subsetneq V(I)$.

\end{enumerate}

\end{document}

