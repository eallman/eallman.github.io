\documentclass[letter,10pt]{article}
\usepackage[margin=1in]{geometry}
\usepackage{amsmath}
\usepackage{amssymb}
\usepackage{graphicx}
%\usepackage[all,2cell,ps]{xy}
\usepackage{pdfsync}

\newcommand{\Z}{\mathbb{Z}}
\newcommand{\C}{\mathbb{C}}
\newcommand{\N}{\mathbb{N}}
\newcommand{\Q}{\mathbb{Q}}
\newcommand{\R}{\mathbb{R}}
\newcommand{\F}{\mathbb{F}}
\newcommand{\ol}{\overline}
\newcommand{\m}{^{-1}}
\newcommand{\imp}{$\implies$}
\newcommand{\la}{\langle}
\newcommand{\ra}{\rangle}
\newcommand{\g}[1]{\la {#1} \ra} %list of generators
\newcommand{\eq}{&=&}
\newcommand{\proof}{\textbf{Proof:}}
\newcommand{\solution}{\textbf{Solution:}}
\newcommand{\modp}[1]{\,\operatorname{mod}\,{#1}}
\newcommand{\GL}{\operatorname{GL}}
\newcommand{\SL}{\operatorname{SL}}
\newcommand\divides{\, \Big{|} \,}

\begin{document}

\centerline{\sc Some solutions and some comments}
\centerline{\sc September 29, 2008}

\

\parskip=0.5\baselineskip


Homework \# 1

{\bf Section 0.3}

\begin{enumerate}

\item[4.]   Compute the remainder when $37^{100}$ is divided by
$29$.

This question asks you to determine $37^{100} \modp{29}$, where
$29$ is a prime.  Notice that $\varphi(29) = 28 = \vert {(\Z/29\Z)}^*\vert$.
Thus,
\begin{align*}
37^{100} &\equiv 37^{28 \cdot 3 + 16} \modp{29}\\
&\equiv 37^{16} \modp{29}\\
&\equiv 8^{16} \modp{29}\\
&\equiv 23 \modp{29}.
\end{align*}

\item[5.] Compute the last two digits of $9^{1500}$.

Here you want to compute $9^{1500} \modp{100}$, so we
first compute that $\varphi(100) = 40$.
\begin{align*}
9^{1500} \modp{100} &\equiv 9^{20} \modp{100}\\
&\equiv 3^{40} \modp{100}\\
&\equiv 1 \modp{100}.
\end{align*}

It follows that the last two digits are 01.

\end{enumerate}

\

Homework \#2

{\bf Section 2.1}

\begin{itemize}

\item[6.]   Give an example of a non-Abelian group $G$ in which
the set of torsion elements of $G$ is not a subgroup.

Example:  Consider the group $\GL_2(\R)$.  

There are lots of examples in this group.  For instance,
$\vert \begin{pmatrix} 0 & -1\\ 1 & -1 \end{pmatrix} \vert = 3$,
$\vert \begin{pmatrix} 0 & -1\\ 0 & ~~1 \end{pmatrix} \vert = 4$,
but their product is not a torsion element.  (Check this.)

\end{itemize}

{\bf Section 2.2}

\begin{itemize}

\item[4.]  For each of $S_3$, $D_8$, and $Q_8$ compute the centralizers
of each element and find the center of each group.  Does Lagrange's theorem
help with the computations?

\begin{itemize}

\item[(a)]\;$S_3=\{1,\;(1\;2),\;(1\;3),\;(2\;3),\;(1\;2\;3),\;(1\;3\;2)\}$
\renewcommand{\c}[2]{C_{S_3}(\{#1\}) &=& \langle#2 \rangle \\} %centralizers
\begin{eqnarray*}
  C_{S_3}(\{1\}) &= &S_3\\
  \c{1\;2}{(1\;2)}
  \c{1\;3}{(1\;3)}
  \c{2\;3}{(2\;3)}
  C_{S_3}(\{(132)\})&= &\langle (123) \rangle = C_{S_3}(\{(132)\}).
\end{eqnarray*}

\item[(b)] $Q_8 = \{1,-1,i,-i,j,-j,k,-k\}$
\renewcommand{\c}[2]{C_{Q_8}(\{#1\}) &=& #2\\} %centralizers
\newcommand{\cc}[3]{C_{Q_8}(\{#1\}) &=& \{#2\} = \g{#3}\\}
\begin{eqnarray*}
  \c{1}{Q_8}
  \c{-1}{Q_8}
  \cc{i}{1,-1,i,-i}{i}
  \cc{j}{1,-1,j,-j}{j}
  \cc{k}{1,-1,k,-k}{k}
  \cc{-i}{1,-1,i,-i}{i}
  \cc{-j}{1,-1,j,-j}{j}
  \cc{-k}{1,-1,k,-k}{k}
\end{eqnarray*}

\item[(c)] 
$D_8 = \{1,\;r,\;r^2,\;r^3,\;s,\;sr,\;sr^2,\;sr^3\}$
\renewcommand{\c}[2]{C_{D_8}(\{#1\}) &=& #2\\} %centralizers
\begin{eqnarray*}
  \c{1}{D_8}
  \c{r}{\{1,\;r,\;r^2,\;r^3\}=\g{r}}
  \c{r^2}{\{1,\;r,\;r^2,\;r^3,\;s,\;sr,\;sr^2,\;sr^3\} = D_8}
  \c{r^3}{\{1,\;r,\;r^2,\;r^3\} = \g{r}}
  \c{s}{\{1,\;r^2,\;s,\;sr^2\}=\g{s,\;r^2}}
  \c{sr}{\{1,\;r^2,\;sr,\;sr^3\} = \g{rs,\;r^2}}
  \c{sr^2}{\{1,\;r^2,\;s,\;sr^2\} = \g{s,\;r^2}}
  \c{sr^3}{\{1,\;r^2,\;sr,\;sr^3\} = \g{rs,\;r^2}}
\end{eqnarray*}

\item[(d)] $D_{16} = \{1,\;r,\;r^2,\;r^3,\;r^4,\;r^5,\;r^6,\;r^7,\;s,\;sr,\;sr^2,\;sr^3,\;sr^4,\;sr^5,\;sr^6,\;sr^7\}$
--- really for problem 6 in Section 2.5
\renewcommand{\c}[2]{C_{D_{16}}(\{#1\}) &=& #2\\} %centralizers
\renewcommand{\cc}[3]{C_{D_{16}}(\{#1\}) &=& \{#2\} = \g{#3}\\}
\begin{eqnarray*}
  \c{1}{D_{16}}
  \cc{r}{1,\;r,\;r^2,\;r^3,\;r^4,\;r^5,\;r^6,\;r^7}{r}
  \cc{r^2}{1,\;r,\;r^2,\;r^3,\;r^4,\;r^5,\;r^6,\;r^7}{r}
  \cc{r^3}{1,\;r,\;r^2,\;r^3,\;r^4,\;r^5,\;r^6,\;r^7}{r}
  \c{r^4}{\{1,\;r,\;r^2,\;r^3,\;r^4,\;r^5,\;r^6,\;r^7,\;s,\;sr,\;sr^2,\;sr^3,\;sr^4,\;sr^5,\;sr^6,\;sr^7\}=D_{16}}
  \cc{r^5}{1,\;r,\;r^2,\;r^3,\;r^4,\;r^5,\;r^6,\;r^7}{r}
  \cc{r^6}{1,\;r,\;r^2,\;r^3,\;r^4,\;r^5,\;r^6,\;r^7}{r}
  \cc{r^7}{1,\;r,\;r^2,\;r^3,\;r^4,\;r^5,\;r^6,\;r^7}{r}
  \cc{s}{1,\;r^4,\;s,\;sr^4}{s,\;r^4}
  \cc{sr}{1,\;r^4,\;,\;sr,\;sr^5}{sr^5,\;r^4}
  \cc{sr^2}{1,\;r^4,\;sr^2,\;sr^6}{sr^2,\;sr^4}
  \cc{sr^3}{1,\;r^4,\;sr^3,\;sr^7}{sr^3,\;r^4}
  \cc{sr^4}{1,\;r^4,\;sr^4,\;s}{s,\;r^4}
  \cc{sr^5}{1,\;r^4,\;sr,\;sr^5}{sr^5,\;r^4}
  \cc{sr^6}{1,\;r^4,\;sr^2,\;sr^6}{sr^2,\;r^4}
  \cc{sr^7}{1,\;r^4,\;sr^3,\;sr^7}{sr^3,\;r^4}
\end{eqnarray*}
\end{itemize}

\newpage

\item[6.]  Let $H$ be a subgroup of $G$.

\begin{itemize}

\item[(a)] Show that $H \le N_G(H)$.

COMMENTS:  To establish this, it is enough to show that 
$H \subseteq N_G(H)$ since we already know that $H$
is a subgroup.  (This follows immediately from the definition
of $N_G(H)$.)  Don't waste your words.

\end{itemize}

\item[8.]  Let $G=S_n$, fix an $i \in \{1, 2, \dots, n\}$
and let $G_i = \{\sigma \in G \mid \sigma(i) = i\}$ (the
stabilizer of $i$).  Use group actions to show that $G_i$
is a subgroup of $G$.

COMMENTS:  Let $G$ act on $A = \{1, \dots, n\}$ by $\sigma\cdot i
= \sigma(i)$.  Then by problem 4a in Section 1.7, the stabilizer of any
point in $A$ is a subgroup of $G$.  (I.e. once you establish that
this is a group action, you are done....)

It is helpful to notice that $G_i$ acting on $A$ is an action isomorphic
to $S_{n-1}$ acting on $A \setminus \{i\}$.  This lets us see that
$\vert G_i \vert = (n-1)!$.  Alternatively, just think about $G_i$; it
consists of all permutations of $n$ that don't move $i$.  Thus, it
consists of all possible permutations on $A' = \{1, \dots, i-1, i+1, \dots, n\}$.

\end{itemize}

\

{\bf Section 2.3}

\begin{itemize}

\item[8.]  Let $\Z_{48} = <x>$ be a cyclic group of order
$48$.  For which integers $a$ does the map $\phi_a: \bar{1} \mapsto
x^a$ extend to an \emph{isomorphism} from $\Z/48\Z$ onto $Z_{48}$?

\noindent {\bf Solution:}  The map $\phi_a$ above extends to a
well-defined homomorphism if, and only if, $(a, 48) = 1$, \emph{i.e.}
the generator $\overline 1$ must be sent to a generator. 

\smallskip

\noindent \emph{Proof.}  Define the map $\phi_a$ by
$$
\phi_a(\overline m) = x^{am}, \ \ \forall \overline m \in \Z/48\Z.
$$

We must show that $\phi_a$ is well-defined and an isomorphism exactly
when $a$ and $48$ are relatively prime.  Note that if $\phi_a$ is
well-defined, then it will also be a homomorphism since then
$$\phi_a(\overline m \overline n) = x^{aman} =
x^{am}x^{an} = \phi_a(\overline m) \phi_a(\overline n).$$

To show that $\phi_a$ is well-defined, suppose that $\overline m =
\overline m' \in \Z/48\Z$.  Then $\exists k \in \Z$ so that $m' = m
+ 48 k$. Thus, $\phi_a(\overline m') = x^{am'} = x^{a(m + 48k)} =
x^{am}x^{a48k} = x^{am}$ in $Z_{48}$.  So $\phi_a (m) = \phi_a(m')$,
as needed.


\item[9.]  Let $Z_{36} = \, <x>$ be a cyclic group of order
$36$.  For which integers $a$ does the map $\psi_a: \bar{1} \mapsto
x^a$ extend to a \emph{well-defined homomorphism} from $\Z/48\Z$
into $Z_{36}$?

\smallskip

\noindent Comments: The insight for this problem is that $\bar 1$,
the generator of $\Z/48 \Z$, must be mapped by $\psi_a$ to an element
of order dividing $48$ in $Z_{36}$.  Thus, $\vert x^a \vert \divides
48$.  We also know that $\vert x^a \vert = \frac{36}{(36,a)}$ in
$Z_{36}$.  Putting this together, we have that $\frac{36}{(36,a)}
\divides 48$.  For this to be true, we must have that $3 \mid
(36,a)$, or equivalently that $3 \mid a$, since $9 \mid 36$, but $3
\,\| \, 48$.

\bigskip

\noindent {\bf Solution:}  The map $\psi_a$ above extends to a
well-defined homomorphism if, and only if, $3 \mid a$.

\smallskip

\noindent \emph{Proof.}  Define the map $\psi_a$ by
$$
\psi_a(\bar m) = x^{am}, \ \ \forall \bar m \in \Z/48\Z.
$$


Let $\bar m = \bar m' \in \Z/48\Z$.  Then, there is an integer $k$ so
that $m' = m + 48k$. We have
\begin{align*}
\psi_a(\bar m) &= \psi_a(\bar m')\\
\iff x^{am} &= x^{am'} \\
\iff x^{am} &= x^{a(m + 48k)} \\
\iff x^{am} &= x^{am}x^{48ka}�\\
\iff 1 &=x^{48ka} \text{ in } Z_{36} \text{ by left cancellation}\\
\iff 36 &\divides 48ka\\
\iff 36 &\divides 48a
\end{align*}
since $x$ has order $36$ in $Z_{36}$ and $k$ depended on the
representative $m'$ for $\bar m$ chosen.  Finally, $36 \divides 48a
\iff 3 \divides a$.

\end{itemize}

\end{document}

