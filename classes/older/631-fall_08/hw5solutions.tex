\documentclass[10pt]{article}

\usepackage[margin=1in, head=1in]{geometry}
\usepackage{amsmath, amssymb, amsthm}
\usepackage{fancyhdr}
\usepackage{graphicx}

% if using a MAC, you may want to uncomment the following line
% to enable reverse searches.
%\usepackage{pdfsync}

\usepackage{fancyhdr}

% Headers and footers
\fancyhf{}
\rfoot{\thepage}

%\setcounter{secnumdepth}{0}

% macros for algebra class
\renewcommand{\theenumi}{\alph{enumi}}
\renewcommand{\emptyset}{\varnothing}
\newcommand{\R}{\mathbb{R}}
\newcommand{\C}{\mathbb{C}}
\newcommand{\Z}{\mathbb{Z}}
\newcommand{\N}{\mathbb{N}}
\newcommand{\Q}{\mathbb{Q}}
\renewcommand{\b}{\textbf}
\newcommand{\re}{\text{Re}}
\newcommand{\im}{\text{Im}}
\renewcommand{\iff}{\Leftrightarrow}
\newcommand{\zbar}{\overline{z}}
\newcommand\SL{\operatorname{SL}}
\newcommand\GL{\operatorname{GL}}
\newcommand{\divides}{\, \Big | \,}

\newcommand{\normsubeq}{\trianglelefteq}
\newcommand{\normsub}{\triangleleft}
\newcommand{\gen}[1]{\left\langle #1 \right\rangle}
\newcommand{\sect}[1]{\vspace{.25in}\noindent\textbf{Section #1}}
\renewcommand{\phi}{\varphi}
\renewcommand{\epsilon}{\varepsilon}
\newcommand{\zmod}[1]{\Z/#1 \Z}
\newcommand{\la}{\langle}
\newcommand{\ra}{\rangle}
\newcommand\inv{^{-1}}

\parindent=0in
\parskip=0.5\baselineskip

% LOOK HERE
% change assignment number and possibly date below
\newcommand\header{\sc Math 631 \hfill Homework 5 \hfill October 8, 2008}

\begin{document}

\header
\section*{Section 4.1}

\begin{itemize}

\item[1.]  Let $G$ be a group acting on $A$, a nonempty set. Prove that if $a,b \in A$ and $b=g \cdot a$ for some $
g \in G$, then $G_b=gG_ag^{-1}$ where $G_a$ is the stabilizer of
$a$. Deduce that if $G$ acts transitively on $A$ then the kernel of
the action is $\cap_{g \in G}gG_ag^{-1}$.
% display your last name
\begin{proof}(Bastille) \ Let $a,b \in A$, $g \in G$ such that $b=g \cdot a$. Let $c \in G_b$. Then note that $$c=(gg^{-1})c(gg^{-1})=g(g^{-1}cg)g^{-1}.$$
But
\begin{align*}
(g^{-1}cg)\cdot a &= (g^{-1}c)\cdot (g \cdot a)=(g^{-1}c) \cdot b \quad \text{ since }b= g \cdot a \\
                                    &= g^{-1} \cdot (c \cdot b)= g^{-1} \cdot b \quad \text{ since } c \in G_b \\
                                    &= g^{-1} \cdot (g \cdot a)= (g^{-1}g) \cdot a=1 \cdot a =a.
\end{align*}
Therefore $g^{-1}cg \in G_a$ and thus $c \in gG_ag^{-1}$ and so $G_b \subseteq gG_ag^{-1}$. (1) \\
Now assume $d \in gG_ag^{-1}$. Then there exists $f \in G_a$ such
that $d=gfg^{-1}$. Then
\begin{align*}
d \cdot b &= (gfg^{-1}) \cdot (g \cdot a)=(gfg^{-1}g) \cdot a \\
                    &= (gf) \cdot a=g \cdot (f \cdot a)= g \cdot a \quad \text{ since } f \in G_a \\
                    &=b.
\end{align*}
Therefore, $d \in G_b$ and thus $gG_ag^{-1} \subseteq G_b$. (2)
Combining (1) and (2) leads to $G_b=gG_ag^{-1}$.

If $G$ acts transitively on $A$, then the kernel of the action, $K$,
is by definition:
$$K= \cap_{b \in A} G_b,$$
but if we fix $a \in A$, then because the action is transitive, for
any $b \in A$, there exists $g \in G$ such that $g= b \cdot a$, so
in particular $A=\{b \in A \}=\{g \cdot a | g \in G \}$. But by the
result just proven, then we have: $G_b=gG_ag^{-1}$ when $b= g \cdot
a$. Hence $K= \cap_{b \in A}G_b= \cap_{g \in G}gG_ag^{-1}$.

\end{proof}

\item[2.] Let $G$ be a permutation group on the set $A$, let $\sigma \in G$ and let $a \in A$.  Prove that $\sigma G_a\sigma\inv = G_{\sigma(a)}$.  Deduce that if $G$ acts transitively on $A$ then $$\bigcap_{\sigma \in G}\sigma G_a\sigma\inv = 1.$$

\begin{proof}(Hazlett)
Choose $\lambda \in \sigma G_a\sigma\inv$.  Then $\lambda = \sigma
\rho \sigma\inv$ for some $\rho \in G_a$.  So $\sigma\rho \sigma\inv
\cdot \sigma(a) = \sigma\rho\cdot a = \sigma \cdot a = \sigma(a)$.
Thus $\lambda \in G_{\sigma(a)}$.  Select $\tau \in G_{\sigma(a)}$.
Notice that $\sigma\inv\tau\sigma\cdot a = \sigma\inv\tau
\cdot\sigma(a) = \sigma\inv \cdot a = \sigma(a)$.  Hence
$\sigma\inv\tau\sigma \in G_a$.  Consequently $\tau \in \sigma
G_a\sigma\inv$.  Therefore $\sigma G_a\sigma\inv = G_{\sigma(a)}$.

We can deduce that $$\bigcap_{\sigma \in G}\sigma G_a\sigma\inv =
\bigcap_{\sigma \in G} G_{\sigma(a)} = \bigcap_{b \in A} G_b.$$  If
$G$ is the trivial group then the only element in $G$ will be $1$
and $\bigcap_{b \in A} G_b = G = 1$.  If $G$ is not the trivial
group there are no elements that fix each $b \in A$ except 1.
Consequently $\bigcap_{b \in A} G_b = 1$.  Therefore
$$\bigcap_{\sigma \in G}\sigma G_a\sigma\inv = \bigcap_{b \in A} G_b
= 1.$$
\end{proof}

\end{itemize}

\section*{Chapter 4.2}

\begin{itemize}

\item[8.]  Prove that if $H$ has finite index $n$ then there is a normal subgroup $K$ of $G$ with $K \leq H$ and $|G : K| \leq n!$.
% display your last name
(Schamel)
\begin{proof}
Let $G$ act by left multiplication on the left cosets of $H$ in $G$.
Let $\pi_H$ denote the permutation represented by this action.  By
Theorem 4.2.3, then $\ker \pi_H \normsub G$ and $\ker \pi_H \leq H$.
Furthermore, $G/\ker \pi_H \cong \pi_H(G)$ and $|\pi_H(G)| = n$,
since this action is transitive.  Thus, by Cayley's Theorem,
$\pi_H(G)$ is isomorphic to a subgroup of $S_n$, and hence $|G :
\ker \pi_H| = |G/\ker \pi_H| \leq |S_n| = n!$.
\end{proof}

\item[9.]  Prove that if $p$ is a prime and $G$ is a group of order $p^\alpha$ for some $\alpha \in \Z^+$, then every subgroup of index $p$ is normal in $G$.  Deduce that every group of order $p^2$ has a normal subgroup of order $p$.

\begin{proof}(Schamel)
Since $p$ is the smallest prime dividing $p^\alpha$, we have by
Corollary 4.2.5 that every subgroup of index $p$ is normal in $G$.
Let $G$ be a group of order $p^2$.  Then $|G| \geq 2^2$, so $G$ has
at least three elements of order greater than $1$.  If $x \in G$ and
$|x| = p$ then $\gen{x} = p$ and we are done.  By Lagrange's
Theorem, the only possible order of an element $x \in G$ besides 1
and $p$ is $p^2$.  But then, if $|x| = p^2$ then $G$ is cyclic of
order $p^2$ and so contains an element of order $p$.  Hence every
group of order $p^2$ has a normal subgroup of order $p$.
\end{proof}


\item[10.]  Prove that every non-abelian group of order 6 has a nonnormal subgroup of order 2.  Use this to classify groups of order 6.

\begin{proof}(Allman)

Let $G$ be a non-Abelian group where $|G|=6$. Then by Lagrange's
Theorem, $G$ has an element $x$ of order $2$ and an element $y$ of
order $3$.  Since $[G : \langle y \rangle] = 2$, the subgroup
$\langle y \rangle \lhd G$.  Consider the element $x$ and its
conjugate $y x y^{-1} = z$.  Then $z \neq e$, since $x \neq e$. In
addition, $z \neq x$, since $G$ is not Abelian and the elements $x$
and $y$ must generate $G$ by counting.  It follows that $\langle x
\rangle$ is not a normal subgroup of $G$.

Since $\langle y \rangle \lhd G$ and $G$ is non-Abelian, it must be
that $xyx^{-1} = y^2 = y^{-1}$.  Thus, $G$ has a presentation as $G
= \langle x, y \mid x^2 = e,~y^3 = e,~xyx^{-1} = y^{-1} \rangle$ and
$G$ is isomorphic to $D_6$ (and therefore also to $S_3$).  Finally,
if $G$ were Abelian, then it must be generated by the product of an
element of order $2$ with an element of order $3$, and thus is
isomorphic to $\Z/6\Z$.
\end{proof}

\end{itemize}

\section*{Chapter 4.3}

\begin{itemize}

\item[2.]Find all conjugacy classes and their sizes in the following groups: \\
(Baggett) *Note: All orders for centralizers of elements in (a) and
(b) were computed in Exercise 2.5.6
\begin{itemize}

\item[a.] $D_8$
\begin{center}
\begin{tabular}{cc}
Conjugacy Class & Size\\
\hline
$O_1 = \{1\}$ & $[D_8:C_{D_8}(1)] = 8/8 = 1$\\
$O_r = \{r, r^3\}$ & $[D_8:C_{D_8}(r)] = 8/4 = 2$\\
$O_{r^2} = \{r^2\}$ & $[D_8:C_{D_8}(r^2)] = 8/8 = 1$\\
$O_s = \{s, sr^2\}$ & $[D_8:C_{D_8}(s)] = 8/4 = 2$\\
$O_{sr} = \{sr, sr^3\}$ & $[D_8:C_{D_8}(sr)] = 8/4 = 2$
\end{tabular}
\end{center}

\item[b.] $Q_8$
\begin{center}
\begin{tabular}{cc}
Conjugacy Class & Size\\
\hline
$O_1 = \{1\}$ & $[Q_8:C_{Q_8}(1)] = 8/8 = 1$\\
$O_{-1} = \{-1\}$ & $[Q_8:C_{Q_8}(-1)] = 8/8 = 1$\\
$O_i = \{i, -i\}$ & $[Q_8:C_{Q_8}(i)] = 8/4 = 2$\\
$O_j = \{j, -j\}$ & $[Q_8:C_{Q_8}(j)] = 8/4 = 2$\\
$O_k = \{k, -k\}$ & $[Q_8:C_{Q_8}(k)] = 8/4 = 2$
\end{tabular}
\end{center}

\item[c.] $A_4$
\begin{center}
\begin{tabular}{cc}
Conjugacy Class & Size\\
\hline
$O_1 = \{1\}$ & 1\\
$O_{(12)(34)} = \{(12)(34),(13)(24),(14)(23)\}$ & 3\\
$O_{(123)} = \{(123),(134),(142),(243)\}$ & 4\\
$O_{(132)} = \{(132),(124),(143),(234)\}$ & 4\\

\end{tabular}
\end{center}

\end{itemize}

\item[5.] If the center of G is of index n, prove that every conjugacy class has at most n elements.

\begin{proof}(Baggett) \ Let $g \in G$ and let $O_g$ be the conjugacy class containing g. Note that $Z(G) \leq C_G(g)$.
We have that $[G:Z(G)] = [G:C_G(g)][C_G(g):Z(G)]$. Since
$[C_G(g):Z(G)] \geq 1$, it follows that $[G:Z(G)] \geq [G:C_G(g)]$.
Therefore, $|O_g| = [G:C_G(g)] \leq [G:Z(G)] = n$. Thus, every
conjugacy class has at most n elements.
\end{proof}

\item[6.]  Assume $G$ is a non-abelian group of order $15$.  Prove that $Z(G)=1$.  Use the fact that $\langle  g \rangle \leq C_{G}(g)$ for
all $g \in G$ to show that there is at most one possible class
equation for $G$.  [Use Exercise 36, Section 3.1.]

\begin{proof}(Mobley) \  We know from Exercise 4 in Section 3.2 that if $|G|=pq$ where $p$ and $q$ are prime that either $G$ is abelian or
$Z(G)=1$.  Since we know $G$ is non-abelian and $15=3\cdot 5$, it
must be the case that $Z(G)=1$.

By the class equation, $15=1+\displaystyle
\sum_{i=1}^{r}|G:C_{G}(g_i)|$.  Thus, $\displaystyle
\sum_{i=1}^{r}|G:C_{G}(g_i)|=14$.  Since the size of each orbit must
divide $G$, we need to consider $1, 3$ and $5$ such that $14=a\cdot
1 + b\cdot 3+ c\cdot 5$ where $a, b$ and $c$ are the number of
orbits of the respective size and $0\leq a\leq 1, 0\leq b \leq 3$
and $0\leq c \leq 2$.  However, orbits of size one are contained in
the center. Therefore, $14=b\cdot 3+ c\cdot 5$ and $b=3$ and $c=1$.
There is no other combination of $b$ and $c$ that will produce $14$.

\end{proof}

\item[7.]  For $n=3,4,6$ and $7$ make lists of the partitions of $n$ and give representatives for the corresponding conjugacy classes of $S_n$.

\begin{proof}(Mobley) \ For $S_3$, the table would be as follows.

\begin{center}
  \begin{tabular}{ l  c  r | }
    Element Representative & Number of each type \\ \hline
    e & 1 \\ \hline
    (1 2) & 3 \\ \hline
    (1 2 3) & 2 \\ \hline
    \hline
    Total & 6
  \end{tabular}
\end{center}

For $S_4$, the table would be as follows.

\begin{center}
  \begin{tabular}{ l  c  r | }
    Element Representative & Number of each type \\ \hline
    e & 1 \\ \hline
    (1 2) & 6 \\ \hline
    (1 2)(3 4) & 3 \\ \hline
    (1 2 3) & 8 \\ \hline
    (1 2 3 4) & 6 \\ \hline
    \hline
    Total & 24
  \end{tabular}
\end{center}

For $S_6$, the table would be as follows.

\begin{center}
  \begin{tabular}{ l  c  r | }
    Element Representative & Number of each type \\ \hline
    e & 1 \\ \hline
    (1 2) & 15 \\ \hline
    (1 2)(3 4) & 45 \\ \hline
    (1 2)(3 4)(5 6) & 15 \\ \hline
    (1 2 3) & 40 \\ \hline
    (1 2 3)(4 5 6) & 40 \\ \hline
    (1 2 3 4) & 90 \\ \hline
    (1 2 3 4)(5 6) & 90 \\ \hline
    (1 2 3 4 5) & 144 \\ \hline
    (1 2 3 4 5 6) & 120 \\ \hline
    (1 2 3)(4 5) & 120 \\ \hline
    \hline
    Total & 720
  \end{tabular}
\end{center}

For $S_7$, the table would be as follows.

\begin{center}
  \begin{tabular}{ l  c  r | }
    Element Representative & Number of each type \\ \hline
    e & 1 \\ \hline
    (1 2) & 21 \\ \hline
    (1 2)(3 4) & 105 \\ \hline
    (1 2)(3 4)(5 6) & 105 \\ \hline
    (1 2 3) & 70 \\ \hline
    (1 2 3)(4 5 6) & 280 \\ \hline
    (1 2 3 4) & 210 \\ \hline
    (1 2 3 4)(5 6) & 630 \\ \hline
    (1 2 3 4 5) & 504 \\ \hline
    (1 2 3 4 5 6) & 840 \\ \hline
    (1 2 3)(4 5) & 420 \\ \hline
    (1 2 3)(4 5)(6 7) & 210 \\ \hline
    (1 2 3 4 5 6 7) & 720 \\ \hline
    (1 2 3 4 5)(6 7) & 504 \\ \hline
    (1 2 3 4)(5 6 7) & 420 \\ \hline
    \hline
    Total & 5040
  \end{tabular}
\end{center}

\end{proof}

\item[9.] Show $|C_{S_n}((1\,2)(3\,4))| = 8\cdot (n-4)!$. for all $n\geq 4$. Determine the elements in this centralizer explicitly.

\begin{proof}
We know the order of $C_{S_n}(1\, 2)(3\, 4)$ is the order of $S_n$
divided by the number of conjugates of $(1\, 2)(3\, 4)$. Notice
$$\mathcal{O}_{(1\, 2)(3\, 4)} = \binom{n}{4}\cdot 3 = \frac{3\cdot n!}{4!(n-4)!} = \frac{n!}{8(n-4)!}.$$
Therefore:
$$|C_{S_n}((1\,2)(3\,4))| = n! \cdot \left(\frac{8(n-4)!}{n!}\right) = 8(n-4)!.$$
Recall that if $\sigma \in C_{S_a}(\tau)$ for some $\sigma, \tau \in
S_a$, then $\sigma \in C_{S_n}(\tau)$ for all $n \geq a$. Since
$|C_{S_4}((1\, 2)(3\, 4))| = 8$, Then $\sigma\tau \in C_{S_n}((1\,
2)(3\, 4))$ where $\sigma \in C_{S_4}((1\, 2)(3\,4))$, and $\tau$ is
an element in $S_n$ which does not contain 1, 2, 3, or 4 in its
cycle decomposition.
\end{proof}

\item[12.] Find a representative for each conjugacy class of order 4 in $S_8$ and $S_{12}$.
\begin{proof}
In $S_8$:
\begin{equation*}
\begin{matrix}
\text{Representative} & \text{class size} \\
(1\, 2\, 3\, 4) & 420 \\
(1\, 2\, 3\, 4)(5\, 6) & 2520 \\
(1\, 2\, 3\, 4)(5\, 6)(7\, 8) & 1260  \\
(1\, 2\, 3\, 4)(5\, 6\, 7\, 8) & 1260 \\
\end{matrix}
\end{equation*}

In $S_{12}$:
\begin{equation*}
\begin{matrix}
\text{Representative} & \text{class size} \\
(1\, 2\, 3\, 4) &  2970 \\
(1\, 2\, 3\, 4)(5\, 6) & 83160  \\
(1\, 2\, 3\, 4)(5\, 6)(7\, 8) & 623700 \\
(1\, 2\, 3\, 4)(5\, 6\, 7\, 8) & 623700 \\
(1\, 2\, 3\, 4)(5\, 6)(7\, 8)(9\, 10) & 1247400 \\
(1\, 2\, 3\, 4)(5\, 6\, 7\, 8)(9\, 10) & 374220 \\
(1\, 2\, 3\, 4)(5\, 6)(7\, 8)(9\, 10)(11\, 12) & 311850 \\
(1\, 2\, 3\, 4)(5\, 6\, 7\, 8)(9\, 10)(11\, 12) & 187110 \\
(1\, 2\, 3\, 4)(5\, 6\, 7\, 8)(9\, 10\, 11\, 12) & 1247400 \\

\end{matrix}
\end{equation*}
\end{proof}

\item[26.]  Let $G$ be a transitive permutation group on the
finite set $A$ with $|A|>1$. Show that there is some $\sigma\in G$
such that $\sigma(a)\ne a$ for all $a\in A$.

% display your last name
\begin{proof}(Gillispie) Because $A$ is finite, we know that $G\le S_{A}$,
which has order $|S_{A}|=|A|!$, and so $|G|\le|A|!$ and is thus
finite.\\
Consider $\cup_{a\in A}G_{a}$.\\
Because $G$ is a transitive group, and by proposition 4.1.2, $|A|=|\mathcal{O}_{a}|=|G:G_{a}|$.\\
Since $G$ is finite $|G:G_{a}|=\frac{|G|}{|G_{a}|}$, and so $|G_{a}|=\frac{|G|}{|G:G_{a}|}=\frac{|G|}{|A|}$.\\
I claim that $|\cup_{a\in A}G_{a}|<|G|$, which implies the existance
of $\sigma\in G$ such that $\sigma\notin\cup_{a\in A}G_{a}$, that
is, $\sigma(a)\ne a$ for all $A$.\\
But, notice that $e(a)=a$ for all $a\in A$, and so $e\in G_{a}$
for all $a\in A$, so all stabilizers contain $e$.\\
So\begin{eqnarray*}
|\cup_{a\in A}G_{a}| & \le & 1+\sum_{a\in A}|G_{a}-\{e\}|\\
 & = & 1+\sum_{a\in A}|G_{a}|-1\\
 & = & 1+\sum_{a\in A}\frac{|G|}{|A|}-1\\
 & = & 1+|A|\frac{|G|}{|A|}-|A|\\
 & = & 1+|G|-A\\
 & \le & |G|-1\\
 & < & |G|.\end{eqnarray*}
As needed.
\end{proof}

\item[30.] If $G$ is a group of odd order, prove for any non-identity element
$x\in G$ that $x$ and $x^{-1}$ are not conjugate in $G$.

\begin{proof}[Proof (Granade)]
Choose some $x\in G\backslash\left\{ e\right\} $. Then,
$\left|\mathcal{O}_{x}\right|$ divides $\left|G\right|$. But then,
since $\left|G\right|$ is odd, we have that $2\nmid\left|G\right|$,
and so $2\nmid\left|\mathcal{O}_{x}\right|$. Suppose for the sake of
contradiction that there exists $x\in G$ such that $x$ is conjugate
to $x^{-1}$. Then $\mathcal{O}_{x}=\mathcal{O}_{x^{-1}}$.

Next, choose $y\in\mathcal{O}_{x}$. By definition, there exists
$g\in G$ such that $y=gxg^{-1}$, and so
$y^{-1}=gx^{-1}g^{-1}\in\mathcal{O}_{x^{-1}}=\mathcal{O}_{x}$. Since
$\left|y\right|\mid\left|G\right|$, and since
$2\nmid\left|G\right|$, we have that $\left|y\right|\ne2$ and thus
that $y\ne y^{-1}$. We may therefore partition $\mathcal{O}_{x}$
into pairs of the form $\left\{ y,y^{-1}\right\} $, showing that
$\mathcal{O}_{x}$ is of even order. This is a contradiction, and so
we conclude that there does not exist any $x\in G$ such that
$x^{-1}\in\mathcal{O}_{x}$.
\end{proof}

\end{itemize}

 \end{document}
