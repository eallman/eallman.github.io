\documentclass[10pt]{article}

\usepackage[margin=1in, head=1in]{geometry}
\usepackage{amsmath, amssymb, amsthm}
\usepackage{fancyhdr}
\usepackage{graphicx}

% if using a MAC, you may want to uncomment the following line
% to enable reverse searches.
\usepackage{pdfsync}

% Headers and footers
\fancyhf{}
\rfoot{\thepage}

%\setcounter{secnumdepth}{0}

% macros for algebra class
\renewcommand{\theenumi}{\alph{enumi}}
\renewcommand{\emptyset}{\varnothing}
\newcommand{\R}{\mathbb{R}}
\newcommand{\C}{\mathbb{C}}
\newcommand{\Z}{\mathbb{Z}}
\newcommand{\N}{\mathbb{N}}
\newcommand{\Q}{\mathbb{Q}}
\renewcommand{\b}{\textbf}
\newcommand{\re}{\text{Re}}
\newcommand{\im}{\text{Im}}
\renewcommand{\iff}{\Leftrightarrow}
\newcommand{\zbar}{\overline{z}}
\newcommand\SL{\operatorname{SL}}
\newcommand\GL{\operatorname{GL}}
\newcommand{\divides}{\, \Big | \,}

\newcommand{\normsubeq}{\trianglelefteq}
\newcommand{\normsub}{\triangleleft}
\newcommand{\gen}[1]{\left\langle #1 \right\rangle}
\newcommand{\sect}[1]{\vspace{.25in}\noindent\textbf{Section #1}}
\renewcommand{\phi}{\varphi}
\renewcommand{\epsilon}{\varepsilon}
\newcommand{\zmod}[1]{\Z/#1 \Z}
\newcommand{\la}{\langle}
\newcommand{\ra}{\rangle}
\newcommand\inv{^{-1}}
\newcommand{\Aut}{\text{Aut}}
\newcommand{\Inn}{\text{Inn}}
\renewcommand{\char}{\text{ char }}

\parindent=0in
\parskip=0.5\baselineskip

% LOOK HERE
% change assignment number and possibly date below
\newcommand\header{{\sc Math 631 \hfill Homework 6 \hfill October 15, 2008}}

\begin{document}

\header  

\section*{Section 4.3}

\begin{itemize}

\item[34.] Prove that if $p$ is a prime and $P$ is a subgroup of $S_p$ of order $p$, then $|N_{S_p}(P)| = p(p-1)$.  [Argue that every conjugate of $P$ contains exactly $p-1$ p-cycles and use the formula for the number of $p$-cycles to compute the index of $N_{S_p}(P)$ in $S_p$.]

\begin{proof}(Hazlett)
Note, by Cauchy's Theorem we can find an element $\sigma \in P$ where $|\sigma| = p$.  However, the only elements of order $p$ in $S_p$ are $p$-cycles.  Hence $P$ contains a $p$-cycle $\sigma$.  Additionally we know $P = \langle\sigma\rangle$ since $|P| = p$.  We also note that powers of $p$-cycles are also $p$-cycles.  Hence $P$ contains the identity and $p-1$ $p$-cycles.  Since conjugation preserves cycle structure we know that every conjugate of $P$ will also have $p-1$ $p$-cycles.  The number of $p$-cycles in $S_p$ is $(p-1)!$.  We deduce that there are $\frac{(p-1)!}{p-1} = (p-2)!$ conjugates of $P$.  If we let $G$ act on $P$ by conjugation we can count the number of elements in any orbit, ie. the number of conjugates of $P$, by the equation $|O_P| = [S_p:N_{S_p}(P)]$.  This gives us $(p-2)! = \frac{|S_p|}{|N_{S_p}(P)|} = \frac{p!}{|N_{S_p}(P)|}$.  Therefore $|N_{S_p}(P)| = \frac{p!}{(p-2)!} = p(p-1)$.
\end{proof}

\end{itemize}

\section*{Chapter 4.4}

\begin{itemize}

\item[1.]  If $\sigma \in \Aut(G)$ and $\varphi_g$ is a conjugation by $g$ prove $\sigma\varphi_g \sigma^{-1} = \varphi_{\sigma(g)}$.  Deduce that $\Inn(G) \normsubeq \Aut(G)$.
% display your last name 

\begin{proof}
(Schamel)
Consider $a \in G$.  Then
\begin{align*}
(\sigma\varphi_g \sigma^{-1})(a) &= \sigma(\varphi_g(\sigma^{-1}(a))) \\
                                 &= \sigma(g\sigma^{-1}(a)g^{-1}) \\
                                 &= \sigma(g) \sigma(\sigma^{-1}(a)) \sigma(g^{-1}) \\
                                 &= \sigma(g) a (\sigma(g))^{-1} \\
                                 &= \varphi_{\sigma(g)}(a).
\end{align*}
Thus $\sigma\varphi_g \sigma^{-1} = \varphi_{\sigma(g)}$.  Hence, for all $\sigma \in \Aut(G)$, $\sigma \Inn(G) \sigma^{-1} = \Inn(G)$ and so $\Inn(G) \normsubeq \Aut(G)$.
\end{proof}


\item[6.] Prove that characteristic subgroups are normal. Give an example of a normal subgroup that is not characteristic.

% display your last name 
\begin{proof}(Bastille) \ Let $H \text{char} G$, and let $g \in G$. Define 
\begin{align*}
\sigma_g : 	& \;G \rightarrow gGg^{-1}=G \\
						& \sigma_g(x)=gxg^{-1}.
\end{align*}
Note that by Proposition 13, since $G \trianglelefteq G, \sigma_g \in \text{Aut}(G)$. Furthermore since $H \text{char}G$, we must have $\sigma_g(H)=H$, i.e. $gHg^{-1}=H$. Since $g$ was chosen arbitrarily, it follows that $H$ is normal in $G$.
\end{proof}
\underline{Remark} Not all normal subgroups are characteristic. Let $G= (\Q,+)$, and consider the subset $S= \Z \subseteq \Q$. Note that $0 \in S$ so $S$ is not empty and if $a,b \in \Z$ so is $a-b$ so $S \leq G$. Furthermore, $S \trianglelefteq G$ since $G$ is Abelian. Define the following map:
\begin{align*}
\sigma_{1/2}: & \; \Q \rightarrow \Q \\
							&\sigma_{1/2}(q)=\frac{1}{2}q.
\end{align*}
Note that $\sigma_{1/2}$ is a homomorphism (1): if $q_1,q_2 \in \Q$,
$$ \sigma_{1/2}(q_1+q_2)=\frac{1}{2}(q_1+q_2)=\frac{1}{2}q_1+\frac{1}{2}q_2=\sigma_{1/2}(q_1)+\sigma_{1/2}(q_2).$$
Furthermore $\sigma_{1/2}$ is bijective (2):
\begin{itemize}
\item if $\sigma_{1/2}(q_1)=\sigma_{1/2}(q_2)$ then $\frac{1}{2}q_1=\frac{1}{2}q_2 \Rightarrow q_1=q_2$ so $\sigma_{1/2}$ is injective;
\item if $r \in \Q$ then $2r \in \Q$ and $\sigma_{1/2}(2r)=\frac{1}{2}(2r)=r$ so $\sigma_{1/2}$ is surjective.
\end{itemize}
Combining (1) and (2) leads to $\sigma_{1/2} \in \text{Aut}(\Q)$. Yet $\sigma_{1/2}(\Z) \neq \Z$ since for example $3 \in \Z$ but $\sigma_{1/2}(3)=\frac{3}{2} \notin \Z$. So $\Z$ is not characteristic in $\Q$.


\item[7.] If $H$ is the unique subgroup of a given order in a group 
$G$, prove $H$ is characteristic in $G$.

\begin{proof}(Baggett) \ Take any $\varphi \in Aut(G)$. Since $\varphi$ is 
an isomorphism, $H \cong \varphi(H)$. Thus, $\varphi(H) \leq G$ and $|\varphi(H)| = |H|$.
Since $H$ is the unique subgroup of a given order in $G$, it follows that 
$\varphi(H) = H$. Thus, $H$ is characteristic in $G$.
\end{proof}

\item[8.] Let $G$ be a group with subgroups $H$ and $K$ with $H \leq K$. 
\begin{itemize}

\item[(a)] Prove that if $H$ is characteristic in $K$ and $K$ is normal in $G$, then $H$ is normal in $G$. 

\begin{proof} (Lawless) 


\end{proof}

\item[(b)] Prove that if $H$ is characteristic in $K$ and $K$ is characteristic in $G$, then $H$ is characteristic in $G$. Use this to prove that the Klein 4-group is characteristic in $S_4$. 

\begin{proof} (Lawless) 
Let $H \char K \char G$. Let $\sigma \in Aut(G)$. Then since $K \char G$,  $\sigma(K) = K$. Thus, $\sigma|_K$ is an automorphism of $K$. Since $H \char K$, then $\sigma|_K(H) = H$. Therefore, $\sigma(H) = H$, and so $H \char G$. 

We will show $V_4 \char A_4 \char S_4$. First, we know $A_4$ is a subgroup of order 12 in $S_4$. We will show $A_4$ is the only subgroup of $S_4$ of order 12. Assume $H \leq S_4$ and $|H| = 12$. Thus, there exists some $\sigma \in H$ with $|\sigma| = 3$. Since $[S_4:H] = 2$, we know that $H \unlhd S_4$, and as such all conjugates of $\sigma$ must be in $H$. Thus, all 3-cycles are in $H$. Since $A_4$ is generated by the 3-cycles of $S_4$, we know $G = A_4$. Thus, $A_4$ is the only subgroup of $S_4$ of order 12, so $A_4 \char S_4$. 

Since $V_4$ is the only subgroup of order 4 in $A_4$, then we know $V_4 \char A_4$. Thus, $V_4 \char S_4$. 
\end{proof}

\item[(c)] Give an example to show that if $H$ is normal in $K$ and $K$ is characteristic in $G$, then $H$ need not be normal in $G$. 

\begin{proof} (Lawless) 
We have shown $V_4 \char A_4$. Consider the subgroup $\la (1\, 2)(3\, 4)\ra \leq V_4$. Notice, however,
$$(1\,2\,3)(1\,2)(3\,4)(1\,3\,2) = (1\, 4)(2\, 3) \notin \la (1\, 2)(3\, 4)\ra.$$
Thus, $\la (1\,2)(3\,4)\ra$ is not a normal subgroup of $A_4$. 

\end{proof}

\end{itemize}
\end{itemize}

\section*{Chapter 5.2}

\begin{itemize}

\item[2.]  In each of parts (a) to (e) give the lists of invariant
  factors for all abelian groups of the specified order:

  \textbf{(a)} order 270, \textbf{(b)} order 9801, \textbf{(c)} order
  320, \textbf{(d)} order 105, \textbf{(e)} order 44100.

\begin{proof}(Mobley) \

\begin{item} (a) 

\begin{center}
  \begin{tabular}{ l  c }
    $n_k$ values & Invariant Factor \\ \hline
    $n_1=2\cdot 3^3\cdot 5$ & $A\cong \Z / 270\Z$ \\ \hline
	\hline    
	$n_1=2\cdot 3^2\cdot 5$ &  \\ \hline
	$n_2=3$ & $A\cong \Z / 90\Z \times \Z / 3\Z$ \\ \hline
    \hline
	$n_1=2\cdot 3\cdot 5$ & \\ \hline
	$n_2=3$ & \\ \hline
	$n_3=3$ & $A\cong \Z / 30\Z \times \Z / 3\Z \times \Z / 3\Z$ \\ \hline
	\hline
  \end{tabular}
\end{center} 

\end{item}

\begin{item} (b)

\begin{center}
  \begin{tabular}{ l  c }
    $n_k$ values & Invariant Factor \\ \hline
    $n_1=3^4 \cdot 11^2$ & $A\cong \Z / 9801\Z$ \\ \hline
	\hline    
	$n_1=3^3 \cdot 11^2$ &  \\ \hline
	$n_2=3$ & $A\cong \Z / 3267\Z \times \Z / 3\Z$ \\ \hline
    \hline
	$n_1=3^2 \cdot 11^2$ & \\ \hline
	$n_2=3$ & \\ \hline
	$n_3=3$ & $A\cong \Z / 1089\Z \times \Z / 3\Z \times \Z / 3\Z$ \\ \hline
	\hline
	$n_1=3^2 \cdot 11^2$ & \\ \hline
	$n_2=9$ & $A\cong \Z / 1089\Z \times \Z / 9\Z$ \\ \hline
	\hline
	$n_1=3 \cdot 11^2$ & \\ \hline
	$n_2=3$ & \\ \hline
	$n_3=3$ & \\ \hline
	$n_4=3$ & $A\cong \Z / 363\Z \times \Z / 3\Z \times \Z / 3\Z \times \Z / 3\Z$ \\ \hline
	\hline
	$n_1=3^4 \cdot 11$ &  \\ \hline
	$n_2=11$ & $A\cong \Z / 891\Z \times \Z / 11\Z$ \\ \hline
    \hline
	$n_1=3^3 \cdot 11$ &  \\ \hline
	$n_2=3 \cdot 11$ & $A\cong \Z / 297\Z \times \Z / 33\Z$ \\ \hline
	\hline
	$n_1=3^2 \cdot 11$ & \\ \hline
	$n_2=3^2 \cdot 11$ & $A\cong \Z / 99\Z \times \Z / 99\Z$ \\ \hline
	\hline
	$n_1=3^2 \cdot 11$ & \\ \hline
	$n_2=3 \cdot 11$ &  \\ \hline
	$n_3=3$ & $A\cong \Z / 99\Z \times \Z / 33\Z \times \Z / 3\Z$ \\ \hline
	\hline
	$n_1=3 \cdot 11$ & \\ \hline
	$n_2=3 \cdot 11$ & \\ \hline
	$n_3=3$ & \\ \hline
	$n_4=3$ & $A\cong \Z / 33\Z \times \Z / 33\Z \times \Z / 3\Z \times \Z / 3\Z$ \\ \hline
	\hline
  \end{tabular}
\end{center} 

\end{item}

\begin{item} (c)

\begin{center}
  \begin{tabular}{ l  c }
    $n_k$ values & Invariant Factor \\ \hline
    $n_1=2^6 \cdot 5$ & $A\cong \Z / 320\Z$ \\ \hline
	\hline    
	$n_1=2^5 \cdot 5$ &  \\ \hline
	$n_2=2$ & $A\cong \Z / 160\Z \times \Z / 2\Z$ \\ \hline
    \hline
	$n_1=2^4 \cdot 5$ & \\ \hline
	$n_2=2^2$ & $A\cong \Z / 80\Z \times \Z / 4\Z$ \\ \hline
	\hline
	$n_1=2^4 \cdot 5$ & \\ \hline
	$n_2=2$ &  \\ \hline
	$n_3=2$ & $A\cong \Z / 80\Z \times \Z / 2\Z \times \Z / 2\Z$ \\ \hline
	\hline
	$n_1=2^3 \cdot 5$ & \\ \hline
	$n_2=2^3$ & $A\cong \Z / 40\Z \times \Z / 8\Z$ \\ \hline
	\hline
	$n_1=2^3 \cdot 5$ &  \\ \hline
	$n_2=2^2$ & \\ \hline
	$n_3=2$ & $A\cong \Z / 40\Z \times \Z / 4\Z \times \Z / 2\Z$ \\ \hline
    \hline
	$n_1=2^3 \cdot 5$ &  \\ \hline
	$n_2=2$ & \\ \hline
	$n_3=2$ & \\ \hline
	$n_4=2$ & $A\cong \Z / 40\Z \times \Z / 2\Z \times \Z / 2\Z \times \Z / 2\Z$ \\ \hline
	\hline
	$n_1=2^2 \cdot 5$ & \\ \hline
	$n_2=2^2$ & \\ \hline
	$n_3=2^2$ & $A\cong \Z / 20\Z \times \Z / 4\Z \times \Z / 4\Z$ \\ \hline
	\hline
	$n_1=2^2 \cdot 5$ & \\ \hline
	$n_2=2^2$ &  \\ \hline
	$n_3=2$ & \\ \hline
	$n_4=2$ & $A\cong \Z / 20\Z \times \Z / 4\Z \times \Z / 2\Z \times \Z / 2\Z$ \\ \hline
	\hline
	$n_1=2^2 \cdot 5$ & \\ \hline
	$n_2=2$ & \\ \hline
	$n_3=2$ & \\ \hline
	$n_4=2$ & \\ \hline
	$n_5=2$ & $A\cong \Z / 20\Z \times \Z / 2\Z \times \Z / 2\Z \times \Z / 2\Z \times \Z / 2\Z$ \\ \hline
	\hline
	$n_1=2 \cdot 5$ & \\ \hline
	$n_2=2$ & \\ \hline
	$n_3=2$ & \\ \hline
	$n_4=2$ & \\ \hline
	$n_5=2$ & \\ \hline
	$n_6=2$ & $A\cong \Z / 10\Z \times \Z / 2\Z \times \Z / 2\Z \times \Z / 2\Z \times \Z / 2\Z \times \Z / 2\Z$ \\ \hline
	\hline
  \end{tabular}
\end{center} 

\end{item}

\begin{item} (d)

\begin{center}
  \begin{tabular}{ l  c }
    $n_k$ values & Invariant Factor \\ \hline
    $n_1=3 \cdot 5 \cdot 7$ & $A\cong \Z / 105\Z$ \\ \hline
	\hline   
  \end{tabular}
\end{center} 

\end{item}

\begin{item} (e)

\begin{center}
  \begin{tabular}{ l  c }
    $n_k$ values & Invariant Factor \\ \hline
    $n_1=2^2 \cdot 3^2 \cdot 5^2 \cdot 7^2$ & $A\cong \Z / 44100\Z$ \\ \hline
	\hline    
	$n_1=2 \cdot 3^2 \cdot 5^2 \cdot 7^2$ &  \\ \hline
	$n_2=2$ & $A\cong \Z / 22050\Z \times \Z / 2\Z$ \\ \hline
    \hline
	$n_1=2 \cdot 3 \cdot 5^2 \cdot 7^2$ & \\ \hline
	$n_2=2 \cdot 3$ & $A\cong \Z / 7350\Z \times \Z / 6\Z$ \\ \hline
	\hline
	$n_1=2 \cdot 3 \cdot 5 \cdot 7^2$ & \\ \hline
	$n_2=2 \cdot 3 \cdot 5$ & $A\cong \Z / 1470\Z \times \Z / 30\Z$ \\ \hline
	\hline
	$n_1=2 \cdot 3 \cdot 5 \cdot 7$ & \\ \hline
	$n_2=2 \cdot 3 \cdot 5 \cdot 7$ & $A\cong \Z / 210\Z \times \Z / 210\Z$ \\ \hline
	\hline
	$n_1=2^2 \cdot 3 \cdot 5^2 \cdot 7^2$ &  \\ \hline
	$n_2=3$ & $A\cong \Z / 14700\Z \times \Z / 3\Z$ \\ \hline
    \hline
	$n_1=2^2 \cdot 3 \cdot 5 \cdot 7^2$ &  \\ \hline
	$n_2=3 \cdot 5$ & $A\cong \Z / 2940\Z \times \Z / 15\Z$ \\ \hline
	\hline
	$n_1=2^2 \cdot 3 \cdot 5 \cdot 7$ & \\ \hline
	$n_2=3 \cdot 5 \cdot 7$ & $A\cong \Z / 420\Z \times \Z / 105\Z$ \\ \hline
	\hline
	$n_1=2^2 \cdot 3^2 \cdot 5 \cdot 7^2$ & \\ \hline
	$n_2=5$ & $A\cong \Z / 8820\Z \times \Z / 5\Z$ \\ \hline
	\hline
	$n_1=2 \cdot 3^2 \cdot 5 \cdot 7^2$ & \\ \hline
	$n_2=2 \cdot 5$ & $A\cong \Z / 4410\Z \times \Z / 10\Z$ \\ \hline
	\hline
	$n_1=2^2 \cdot 3^2 \cdot 5^2 \cdot 7$ & \\ \hline
	$n_2=7$ & $A\cong \Z / 6300\Z \times \Z / 7\Z$ \\ \hline
	\hline
	$n_1=2 \cdot 3^2 \cdot 5^2 \cdot 7$ & \\ \hline
	$n_2=2 \cdot 7$ & $A\cong \Z / 3150\Z \times \Z / 14\Z$ \\ \hline
	\hline
	$n_1=2^2 \cdot 3 \cdot 5^2 \cdot 7$ & \\ \hline
	$n_2=3 \cdot 7$ & $A\cong \Z / 2100\Z \times \Z / 21\Z$ \\ \hline
	\hline
	$n_1=2^2 \cdot 3^2 \cdot 5 \cdot 7$ & \\ \hline
	$n_2=5 \cdot 7$ & $A\cong \Z / 1260\Z \times \Z / 35\Z$ \\ \hline
	\hline
	$n_1=2 \cdot 3 \cdot 5^2 \cdot 7$ & \\ \hline
	$n_2=2 \cdot 3 \cdot 7$ & $A\cong \Z / 1050\Z \times \Z / 42\Z$ \\ \hline
	\hline
	$n_1=2 \cdot 3^2 \cdot 5 \cdot 7$ & \\ \hline
	$n_2=2 \cdot 5 \cdot 7$ & $A\cong \Z / 630\Z \times \Z / 70\Z$ \\ \hline
	\hline
  \end{tabular}
\end{center} 

\end{item}

\end{proof}


\item[3.]  In each of parts (a) to (e) give the lists of elementary
  divisors for all abelian groups of the specified order and then
  match each list with the corresponding list of invariant factors
  found in the preceding exercise: \textbf{(a)} order 270,
  \textbf{(b)} order 9801, \textbf{(c)} order 320, \textbf{(d)} order
  105, \textbf{(e)} order 44100.

\begin{proof}(Buchholz) \

\begin{item} (a)
\begin{center}
  \begin{tabular}{ l c c  }
    Orders $p^{\beta}$ & Partitions of $\beta$ & Abelian Groups\\ \hline
    $2^1$ & $1$ & $Z_{2}$ \\ \hline
  	$3^{3}$ & $3$; $2,1$; $1,1,1$ & $Z_{27}$, $Z_9 \times Z_3 $, $Z_3 \times Z_3 \times Z_3$ \\ \hline
	  $5^1$ & $1$ & $Z_{5}$\\ \hline
	\hline
  \end{tabular}
\end{center} 

Taking all the possible ways gives all the isomorphism types:

\begin{itemize}
	\item $Z_{27} \times Z_{5} \times Z_{2}$
	\item $Z_9 \times Z_3 \times Z_{5} \times Z_{2}$
	\item $Z_5 \times Z_3 \times Z_3 \times Z_{3} \times Z_{2}$ 
\end{itemize}

Matching with the list of invariant factors we get
 \begin{itemize}
	\item $Z_{27} \times Z_{5} \times Z_{2} \cong \Z / 270\Z$
	\item $Z_9 \times Z_3 \times Z_{5} \times Z_{2} \cong \Z / 90\Z \times \Z / 3\Z$
	\item $Z_5 \times Z_3 \times Z_3 \times Z_{3} \times Z_{2} \cong \Z / 30\Z \times \Z / 3\Z \times \Z / 3\Z$ 
\end{itemize}


\end{item}


\begin{item} (b)
\begin{center}
  \begin{tabular}{ l c c  }
    Orders $p^{\beta}$ & Partitions of $\beta$ & Abelian Groups\\ \hline
    $3^4$ & $4$; $3,1$; $2,2$; $2,1,1$;& $Z_{81}$, $Z_{27} \times Z_3 $, $Z_9 \times Z_9$, $Z_9 \times Z_3 \times Z_3$,\\
    $\text{ }$ &  $1,1,1,1$ & $Z_3 \times Z_3 \times Z_3 \times Z_3$ \\ \hline   
		$11^{2}$ & $2$; $1,1$ & $Z_{121}$, $Z_{11}\times Z_{11}$ \\ \hline
	\hline
  \end{tabular}
\end{center} 

Taking all the possible ways gives all the isomorphism types:

\begin{itemize}
	\item $Z_{121} \times Z_{81}$
	\item $Z_{121} \times Z_{27} \times Z_3$
	\item $Z_{121} \times Z_9 \times Z_9$ 
	\item $Z_{121} \times Z_9 \times Z_3 \times Z_3$
	\item $Z_{121} \times Z_3 \times Z_3 \times Z_3 \times Z_3$
	\item $Z_{81} \times Z_{11} \times Z_{11}$
	\item $Z_{27} \times Z_{11} \times Z_{11} \times Z_3$
	\item $Z_{11} \times Z_{11} \times Z_9 \times Z_9$ 
	\item $Z_{11} \times Z_{11} \times Z_9 \times Z_3 \times Z_3$
	\item $Z_{11} \times Z_{11} \times Z_3 \times Z_3 \times Z_3 \times Z_3$
	\end{itemize}

Matching with the list of invariant factors we get
\begin{itemize}
	\item $Z_{121} \times Z_{81} \cong \Z / 9801\Z$
	\item $Z_{121} \times Z_{27} \times Z_3 \cong\Z / 3267\Z \times \Z / 3\Z$
	\item $Z_{121} \times Z_9 \times Z_9 \cong \Z / 1089\Z \times \Z / 9\Z$ 
	\item $Z_{121} \times Z_9 \times Z_3 \times Z_3 \cong \Z / 1089\Z \times \Z / 3\Z \times \Z / 3\Z$
	\item $Z_{121} \times Z_3 \times Z_3 \times Z_3 \times Z_3 \cong \Z / 363\Z \times \Z / 3\Z \times \Z / 3\Z \times \Z / 3\Z$
	\item $Z_{81} \times Z_{11} \times Z_{11} \cong \Z / 891\Z \times \Z / 11\Z$
	\item $Z_{27} \times Z_{11} \times Z_{11} \times Z_3 \cong \Z / 297\Z \times \Z / 33\Z$
	\item $Z_{11} \times Z_{11} \times Z_9 \times Z_9 \cong \Z / 99\Z \times \Z / 99\Z$ 
	\item $Z_{11} \times Z_{11} \times Z_9 \times Z_3 \times Z_3 \cong \Z / 99\Z \times \Z / 33\Z \times \Z / 3\Z$
	\item $Z_{11} \times Z_{11} \times Z_3 \times Z_3 \times Z_3 \times Z_3 \cong \Z / 33\Z \times \Z / 33\Z \times \Z / 3\Z \times \Z / 3\Z$
\end{itemize}


\end{item}

\begin{item} (c)
\begin{center}
  \begin{tabular}{ l c  c  }
    Orders $p^{\beta}$ & Partitions of $\beta$ & Abelian Groups\\ \hline
    $2^6$ & $6$; $5,1$; $4,2$; $4,1,1$; & $Z_{64}$, $Z_{32} \times Z_2 $, $Z_{16} \times Z_4$, $Z_{16} \times Z_2 \times Z_2$,\\
      
		
		 $\text{ }$ & $3,3$; $3,2,1$; $3,1,1,1$; & $Z_8 \times Z_8$,  $Z_8 \times Z_4 \times Z_2$, $Z_8 \times Z_2 \times Z_2 \times Z_2$,\\
		
	  	$\text{ }$ & $2,2,2$;$2,2,1,1$;  &  $Z_4 \times Z_4 \times Z_4$, $Z_4 \times Z_4 \times Z_2 \times Z_2$,  \\
		
		
			$\text{ }$ & $2,1,1,1,1$; $1,1,1,1,1,1$ & $Z_4 \times Z_2 \times Z_2 \times Z_2 \times Z_2$, $Z_2 \times Z_2 \times Z_2 \times Z_2 \times Z_2 \times Z_2$\\ \hline
		
		$5^{1}$ & $1$ & $Z_{5}$ \\ \hline
	\hline
  \end{tabular}
\end{center} 

Taking all the possible ways gives all the isomorphism types:

\begin{itemize}
	\item $Z_{64} \times Z_{5}$
	\item $Z_{32} \times Z_{5} \times Z_2 $
	\item $Z_{16} \times Z_{5} \times Z_4$ 
	\item $Z_{16} \times Z_{5} \times Z_2 \times Z_2$
	\item $Z_{8} \times Z_8 \times Z_5$
	\item $Z_{8} \times Z_5 \times Z_4 \times Z_2$
	\item $Z_{8} \times Z_5 \times Z_2 \times Z_2 \times Z_2$
	\item $Z_{5} \times Z_4 \times Z_4 \times Z_4$
	\item $Z_{5} \times Z_4 \times Z_4 \times Z_2 \times Z_2$
	\item $Z_{5} \times Z_4 \times Z_2 \times Z_2 \times Z_2 \times Z_2$
	\item $Z_{5} \times Z_2 \times Z_2 \times Z_2 \times Z_2 \times Z_2 \times Z_2$
\end{itemize}


Matching with the list of invariant factors we get
\begin{itemize}
	\item $Z_{64} \times Z_{5} \cong \Z / 320\Z $
	\item $Z_{32} \times Z_{5} \times Z_2 \cong \Z / 160\Z \times \Z / 2\Z$
	\item $Z_{16} \times Z_{5} \times Z_4 \cong \Z / 80\Z \times \Z / 4\Z$ 
	\item $Z_{16} \times Z_{5} \times Z_2 \times Z_2 \cong \Z / 80\Z \times \Z / 2\Z \times \Z / 2\Z$
	\item $Z_{8} \times Z_8 \times Z_5 \cong \Z / 40\Z \times \Z / 8\Z$
	\item $Z_{8} \times Z_5 \times Z_4 \times Z_2 \cong \Z / 40\Z \times \Z / 4\Z \times \Z / 2\Z$
	\item $Z_{8} \times Z_5 \times Z_2 \times Z_2 \times Z_2 \cong \Z / 40\Z \times \Z / 2\Z \times \Z / 2\Z \times \Z / 2\Z$
	\item $Z_{5} \times Z_4 \times Z_4 \times Z_4 \cong \Z / 20\Z \times \Z / 4\Z \times \Z / 4\Z$
	\item $Z_{5} \times Z_4 \times Z_4 \times Z_2 \times Z_2 \cong \Z / 20\Z \times \Z / 4\Z \times \Z / 2\Z \times \Z / 2\Z$
	\item $Z_{5} \times Z_4 \times Z_2 \times Z_2 \times Z_2 \times Z_2 \cong \Z / 20\Z \times \Z / 2\Z \times \Z / 2\Z \times \Z / 2\Z \times \Z / 2\Z$
	\item $Z_{5} \times Z_2 \times Z_2 \times Z_2 \times Z_2 \times Z_2 \times Z_2 \cong \Z / 10\Z \times \Z / 2\Z \times \Z / 2\Z \times \Z / 2\Z \times \Z / 2\Z \times \Z / 2\Z$
\end{itemize}


\end{item}

\begin{item} (d)
\begin{center}
  \begin{tabular}{ l c c  }
    Orders $p^{\beta}$ & Partitions of $\beta$ & Abelian Groups\\ \hline
    $3^1$ & $1$ & $Z_{3}$ \\ \hline
	  $5^1$ & $1$ & $Z_{5}$\\ \hline
	  $7^1$ & $1$ & $Z_{7}$\\ \hline
	\hline
  \end{tabular}
\end{center} 

Taking all the possible ways gives all the isomorphism types:

\begin{itemize}
	\item $Z_7 \times Z_{5} \times Z_{3}$
\end{itemize}

Matching with the list of invariant factors we get
\begin{itemize}
\item $Z_7 \times Z_{5} \times Z_{3}\cong \Z / 105\Z$
\end{itemize}
\end{item}

\begin{item} (e)
\begin{center}
  \begin{tabular}{ l c  c  }
    Orders $p^{\beta}$ & Partitions of $\beta$ & Abelian Groups\\ \hline
    $2^{2}$ & $2$; $1,1$ & $Z_{4}$, $Z_{2}\times Z_{2}$ \\ \hline
    $3^{2}$ & $2$; $1,1$ & $Z_{9}$, $Z_{3}\times Z_{3}$ \\ \hline
    $5^{2}$ & $2$; $1,1$ & $Z_{25}$, $Z_{5}\times Z_{5}$ \\ \hline
    $7^{2}$ & $2$; $1,1$ & $Z_{49}$, $Z_{7}\times Z_{7}$ \\ \hline
	\hline
  \end{tabular}
\end{center} 

Taking all the possible ways gives all the isomorphism types:

\begin{itemize}
	\item $Z_{49} \times Z_{25} \times Z_9 \times Z_4$
	\item $Z_{49} \times Z_{25} \times Z_9 \times Z_2 \times Z_2$
	\item $Z_{49} \times Z_{25} \times Z_3 \times Z_3 \times Z_2 \times Z_2$
	\item $Z_{49} \times Z_{25} \times Z_4 \times Z_3 \times Z_3$
	\item $Z_{49} \times Z_{9} \times Z_5 \times Z_5 \times Z_2 \times Z_2$
	\item $Z_{49} \times Z_{9} \times Z_5 \times Z_5 \times Z_4$
	\item $Z_{49} \times Z_{5} \times Z_5 \times Z_3 \times Z_3 \times Z_2 \times Z_2$	
	\item $Z_{49} \times Z_{5} \times Z_5 \times Z_4 \times Z_3 \times Z_3$	
	\item $Z_{25} \times Z_{9} \times Z_7 \times Z_7 \times Z_2 \times Z_2$
	\item $Z_{25} \times Z_{9} \times Z_7 \times Z_7 \times Z_4$
	\item $Z_{25} \times Z_{7} \times Z_7 \times Z_4 \times Z_3 \times Z_3$
	\item $Z_{25} \times Z_{7} \times Z_7 \times Z_3 \times Z_3 \times Z_2 \times Z_2$
	\item $Z_{9} \times Z_7 \times Z_7 \times Z_5 \times Z_5 \times Z_4$
	\item $Z_{9} \times Z_7 \times Z_7 \times Z_5 \times Z_5 \times Z_2 \times Z_2 $
	\item $Z_{7} \times Z_7 \times Z_5 \times Z_5 \times Z_4 \times Z_3 \times Z_3$
	\item $Z_{7} \times Z_7 \times Z_5 \times Z_5 \times Z_3 \times Z_3 \times Z_2 \times Z_2 $
\end{itemize}

Matching with the list of invariant factors we get
\begin{itemize}
	\item $Z_{49} \times Z_{25} \times Z_9 \times Z_4 \cong \Z / 44100\Z$
	\item $Z_{49} \times Z_{25} \times Z_9 \times Z_2 \times Z_2 \cong \Z / 22050\Z \times \Z / 2\Z$
	\item $Z_{49} \times Z_{25} \times Z_3 \times Z_3 \times Z_2 \times Z_2\cong \Z / 7350\Z \times \Z / 6\Z$
	\item $Z_{49} \times Z_{25} \times Z_4 \times Z_3 \times Z_3\cong \Z / 14700\Z \times \Z / 3\Z$
	\item $Z_{49} \times Z_{9} \times Z_5 \times Z_5 \times Z_2 \times Z_2 \cong \Z / 4410\Z \times \Z / 10\Z$
	\item $Z_{49} \times Z_{9} \times Z_5 \times Z_5 \times Z_4 \cong \Z / 8820\Z \times \Z / 5\Z$
	\item $Z_{49} \times Z_{5} \times Z_5 \times Z_3 \times Z_3 \times Z_2 \times Z_2 \cong \Z / 1470\Z \times \Z / 30\Z$	
	\item $Z_{49} \times Z_{5} \times Z_5 \times Z_4 \times Z_3 \times Z_3 \cong \Z / 2940\Z \times \Z / 15\Z$	
	\item $Z_{25} \times Z_{9} \times Z_7 \times Z_7 \times Z_2 \times Z_2 \cong \Z / 3150\Z \times \Z / 14\Z$
	\item $Z_{25} \times Z_{9} \times Z_7 \times Z_7 \times Z_4 \cong \Z / 6300\Z \times \Z / 7\Z$
	\item $Z_{25} \times Z_{7} \times Z_7 \times Z_4 \times Z_3 \times Z_3 \cong \Z / 2100\Z \times \Z / 21\Z$
	\item $Z_{25} \times Z_{7} \times Z_7 \times Z_3 \times Z_3 \times Z_2 \times Z_2 \cong \Z / 1050\Z \times \Z / 42\Z$
	\item $Z_{9} \times Z_7 \times Z_7 \times Z_5 \times Z_5 \times Z_4 \cong \Z / 1260\Z \times \Z / 35\Z$
	\item $Z_{9} \times Z_7 \times Z_7 \times Z_5 \times Z_5 \times Z_2 \times Z_2 \cong \Z / 630\Z \times \Z / 70\Z$
	\item $Z_{7} \times Z_7 \times Z_5 \times Z_5 \times Z_4 \times Z_3 \times Z_3 \cong \Z / 420\Z \times \Z / 105\Z$
	\item $Z_{7} \times Z_7 \times Z_5 \times Z_5 \times Z_3 \times Z_3 \times Z_2 \times Z_2 \cong \Z / 210\Z \times \Z / 210\Z$
\end{itemize}

\end{item}

\end{proof}


\item[4.]  In each of parts (a) to (d) determine which pair of abelian
groups listed are isomorphic.

% display your last name 
(Gillispie)  

\textbf{(a)} $\{4,9\}$, $\{6,6\}$, $\{8,3\}$, $\{9,4\}$, $\{6,4\}$,
$\{64\}$.   

The first and fourth are isomorphic.

\begin{proof}  Since this is simply a rearrangement of the cyclic
groups generating the groups we have$\{4,9\}\cong\{9,4\}$.

Also the invariant factor representation of$\{8,3\}$ is $\{2^{3}\cdot3\}$
and $\{6,4\}$ is it's own invariant factor representation. Since
the invariant factor representation is unique by the fundamental theorem
of finitely generated abelian groups, we know these are not isomorphic.
Also the order of each of these groups is 24 whereas the order of
$\{4,9\}$ is 36, and so neither $\{8,3\}$ nor $\{6,4\}$ is isomorphic
to $\{4,9\}$.

Also the invariant factor representation of $\{4,9\}$ is $\{36\}$
and the invariant factor representation of $\{6,6\}$ is itself, by
the FToFGAG we have that these are not isomorphic groups.

None of the above groups is of order 64, and as such none are isomorphic
to $\{64\}$.
\end{proof}

\textbf{(b)} $\{2^{2},\,2\cdot3^{2}\}$, $\{2^{2}\cdot3,\,2\cdot3\}$,
$\{2^{3}\cdot3^{2}\}$, $\{2^{2}\cdot3^{2},\,2\}$.  

The first and fourth are isomorphic.

\begin{proof}  Note that $\{2^{2},\,2\cdot3^{2}\}\cong\{2^{2},\,2,\,3^{2}\}\cong\{2^{2}\cdot3^{2},\,2\}$. 

The highest element order of $\{2^{2},\,2\cdot3^{2}\}$ is $lcm(2^{2},2\cdot3^{2})=2^{2}3^{2}$.

But the highest element order of $\{2^{3}\cdot3^{2}\}$ is $2^{3}\cdot3^{2}$
and so $\{2^{3}\cdot3^{2}\}\ncong\{2^{2},\,2\cdot3^{2}\}$.\\
And the highest order element of $\{2^{2}\cdot3,\,2\cdot3\}$ is $lcm(2^{2}\cdot3,\,2\cdot3)=2^{2}\cdot3$,
and so $\{2^{2},\,2\cdot3^{2}\}\ncong\{2^{2}\cdot3,\,2\cdot3\}\ncong\{2^{3}\cdot3^{2}\}$.
\end{proof}

\textbf{(c)} $\{5^{2}\cdot7^{2},\,3^{2}\cdot5\cdot7\}$, $\{3^{2}\cdot5^{2}\cdot7,\,5\cdot7^{2}\}$,
$\{3\cdot5^{2},\,7^{2},\,3\cdot5\cdot7\}$, $\{5^{2}\cdot7,\,3^{2}\cdot5,\,7^{2}\}$.  

The first, second, and fourth are isomorphic.

\begin{proof} Note that $ $$\{5^{2}\cdot7^{2},\,3^{2}\cdot5\cdot7\}\cong\{5^{2},7^{2},3^{2},5,7\cong\{7^{2}\cdot5^{2}\cdot3^{2},7\cdot5\}$.\\
And $\{3^{2}\cdot5^{2}\cdot7,5\cdot7^{2}\}\cong\{3^{2},5^{2},7,5,7^{2}\}\cong\{7^{2}\cdot5^{2}\cdot3^{2},7\cdot5\}$.\\
Finally $\{5^{2}\cdot7,3^{2}\cdot5,7^{2}\}\cong\{5^{2},,3^{2},5,7^{2}\}\cong\{7^{2}\cdot5^{2}\cdot3^{2},7\cdot5\}$.\\
But notice that the highest element order in $\{7^{2}\cdot5^{2}\cdot3^{2},7\cdot5\}$
is $7^{2}\cdot5^{2}\cdot3^{2}$, and the highest element order in
$\{3\cdot5^{2},7^{2},3\cdot5\cdot7\}$ is $lcm(3\cdot5^{2}\cdot7,3\cdot5\cdot7^{2})=3\cdot5^{2}\cdot7^{2}$
and hence $\{3\cdot5^{2},\,7^{2},\,3\cdot5\cdot7\}\ncong\{7^{2}\cdot5^{2}\cdot3^{2},7\cdot5\}$.\\
\end{proof}

\textbf{(d)} $\{2^{2}\cdot5\cdot7,\,2^{3}\cdot5^{3},\,2\cdot5^{2}\}$,
$\{2^{3}\cdot5^{3}\cdot7,\,2^{3}\cdot5^{3}\}$, $\{2,\,2^{2}\cdot7,\,2^{3},\,5^{3},\,5^{3}\}$,
$\{2\cdot5^{3},\,2^{2}\cdot5^{3},\,2^{3},\,7\}$.

The third and fourth are isomorphic.

\begin{proof} 

Note that, $\{2,\,2^{2}\cdot7,\,2^{3},\,5^{3},\,5^{3}\}\cong\{2,\,2^{2},\,7,\,2^{3},\,5^{3},\,5^{3}\}\cong\{7\cdot5^{3}\cdot2^{3},\,5^{3}\cdot2^{2},\,2\}$.\\
And $\{2\cdot5^{3},\,2^{2}\cdot5^{3},\,2^{3},\,7\}\cong\{2,\,5^{3},\,2^{2},\,5^{3},\,2^{3},\,7\}\cong\{7\cdot5^{3}\cdot2^{3},\,5^{3}\cdot2^{2},\,2\}$.
\\
And so $\{2,\,2^{2}\cdot7,\,2^{3},\,5^{3},\,5^{3}\}\cong\{7\cdot5^{3}\cdot2^{3},\,5^{3}\cdot2^{2},\,2\}$.

But we have that $\{2^{2}\cdot5\cdot7,\,2^{3}\cdot5^{3},\,2\cdot5^{2}\}\cong\{2^{2},\,5,\,7,\,2^{3},\,5^{3},\,2,\,5^{2}\}\cong\{7\cdot5^{3}\cdot2^{3},\,5^{2}\cdot2^{2},\,2\cdot5\}$.\\
And $\{2^{3}\cdot5^{3}\cdot7,\,2^{3}\cdot5^{3}\}\cong\{2^{3},\,5^{3},\,7,\,2^{3},\,5^{3}\}\cong\{7\cdot5^{3}\cdot2^{3},\,2^{3}\cdot5^{3}\}$.

These are the invariant factor decomposition representations of these
abelian groups, and by the fundamental theorem of finitely generated
Abelian groups we know that these are unique and hence the second
two groups are not isomorphic to each other or the prior groups.
\end{proof}

\item [9.] Let $A=Z_{60}\times Z_{45}\times Z_{12}\times Z_{36}$. Find
the number of elements of order 2 and the number of subgroups of index
2 in $A$.

\emph{(Granade).}  Let $Z_{60}=\left\langle a\right\rangle $,
$Z_{45}=\left\langle b\right\rangle $, $Z_{12}=\left\langle
  c\right\rangle $ and $Z_{36}=\left\langle d\right\rangle $.  Then,
let $x\in A$ such that $x=\left(a^{i},b^{j},c^{k},d^{l}\right)$ for
$i,j,k,l$ being reduced to the smallest non-negative power (that is,
reduce each modulo the order of the respective groups). We shall adopt
the abbreviated notation $\overline{\left(i,j,k,l\right)}$.  The order
of $x$ is then given by
$\left|x\right|=\operatorname{lcm}\left(i,j,k,l\right)$.  Thus, if
$\left|x\right|=2$, we must have that
$\left|i\right|,\left|j\right|,\left|k\right|,\left|l\right|\in\left\{
  1,2\right\} $ and that at least one of
$\left|i\right|,\left|j\right|,\left|k\right|,\left|l\right|$ is
2. But then, $Z_{45}$ contains no elements of order $2$, and so the
elements of order 2 in $A$ are given by:\[
\overline{\left(0,0,0,18\right)},\,\overline{\left(0,0,6,0\right)},\,\overline{\left(0,0,6,18\right)},\,\overline{\left(30,0,0,0\right)},\,\overline{\left(30,0,0,18\right)},\,\overline{\left(30,0,6,0\right)},\,\overline{\left(30,0,6,18\right)}\]
That is, there are 7 elements of order 2 in $A$.

As for subgroups $H$ of index 2, we have that $\left[A:H\right] = 2$
implies that $H\lhd A$ and so we may consider $A/H$. But then, this
means that $H$ is the kernel of a surjective homomorphism $\phi:G\to Z_{2}$.
Thus, the question \emph{really} asks us to find distinct homomorphisms
$Z_{60}\times Z_{45}\times Z_{12}\times Z_{36}\twoheadrightarrow Z_{2}$.
Consider:\[
\phi_{0001}\left(\overline{\left(i,j,k,l\right)}\right)=l\bmod2\]
Then, since modular arithmetic is well-behaved, we have that $\phi_{0001}$
is a homomorphism. To see that $\phi_{0001}$ is surjective, note
that $\left(0,0,0,0\right)\mapsto0$ and that $\left(0,0,0,1\right)\mapsto1$.
The odd name given to this homomorphism reflects that we consider
the element $l$, but that we do not consider $i,j,k$. Thus, name
similar functions:\[
\phi_{s_{1}s_{2}s_{3}s_{4}}\left(\overline{\left(i,j,k,l\right)}\right)=\left(s_{1}i+s_{2}j+s_{3}k+s_{l}l\right)\bmod2\]
Then, we claim that $\phi_{s_{1}s_{2}s_{3}s_{4}}$ is a surjective
homomorphism unless $s_{1}s_{2}s_{3}s_{4}=0000$ or unless $s_{2}=1$.
To see this, note that $\phi_{s_{1}1s_{3}s_{4}}$ fails to be a homomorphism
as it fails to even be a function:\begin{eqnarray*}
\phi_{s_{1}1s_{3}s_{4}}\left(\overline{\left(0,0,0,0\right)}\right) & = & 0\\
\phi_{s_{1}1s_{3}s_{4}}\left(\overline{\left(0,45,0,0\right)}\right) & = & 1\end{eqnarray*}
Similarly, note that $\phi_{0000}$ fails to be surjective, since
$\phi_{0000}\left(x\right)=0$ for all $x\in A$. From there, it remains
to be shown that all other functions of the form $\phi_{s_{1}s_{2}s_{3}s_{4}}$
are surjective homomorphisms. To see this, note that since if $s_{2}=0$,
then $\phi$ is well-defined, as each of $60,12,36$ is divisible
by 2. Moreover, since each of the functions of this form has at least
one of $s_{1},s_{3},s_{4}\ne0$, we may choose exactly one selected
coordinate to be one, and thus each function is surjective.

This gives us that each subgroup corresponds to an element of $\left(Z_{2}\times Z_{2}\times Z_{2}\right)\backslash\left(0,0,0\right)$
for a total of seven subgroups:\[
\ker\phi_{0001},\,\ker\phi_{0010},\,\ker\phi_{0011},\,\ker\phi_{1000},\,\ker\phi_{1001},\,\ker\phi_{1010},\,\ker\phi_{1011}\]
\end{itemize}

 \end{document}
