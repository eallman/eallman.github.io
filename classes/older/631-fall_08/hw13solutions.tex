\documentclass[10pt]{article}

\usepackage[margin=1in, head=1in]{geometry}
\usepackage{amsmath, amssymb, amsthm}
\usepackage{fancyhdr}
\usepackage{graphicx}
\usepackage{braket}

% if using a MAC, you may want to uncomment the following line
% to enable reverse searches.
%\usepackage{pdfsync}

% Headers and footers
\fancyhf{}
\rfoot{\thepage}

%\setcounter{secnumdepth}{0}

% macros for algebra class
\renewcommand{\theenumi}{\alph{enumi}}
\renewcommand{\emptyset}{\varnothing}
\newcommand{\R}{\mathbb{R}}
\newcommand{\C}{\mathbb{C}}
\newcommand{\Z}{\mathbb{Z}}
\newcommand{\N}{\mathbb{N}}
\newcommand{\Q}{\mathbb{Q}}
\renewcommand{\b}{\textbf}
\newcommand{\re}{\text{Re}}
\newcommand{\im}{\text{Im}}
\renewcommand{\iff}{\Leftrightarrow}
\newcommand{\zbar}{\overline{z}}
\newcommand\SL{\operatorname{SL}}
\newcommand\GL{\operatorname{GL}}
\newcommand{\divides}{\, \Big | \,}

\newcommand{\normsubeq}{\trianglelefteq}
\newcommand{\normsub}{\triangleleft}
\newcommand{\gen}[1]{\left\langle #1 \right\rangle}
\newcommand{\sect}[1]{\vspace{.25in}\noindent\textbf{Section #1}}
\renewcommand{\phi}{\varphi}
\renewcommand{\epsilon}{\varepsilon}
\newcommand{\zmod}[1]{\Z/#1 \Z}
\newcommand{\la}{\langle}
\newcommand{\ra}{\rangle}
\newcommand\inv{^{-1}}
\newcommand{\Aut}{\text{Aut}}
\newcommand{\Inn}{\text{Inn}}
\renewcommand{\char}{\text{ char }}
\newcommand{\Syl}{\operatorname{Syl}}
%\newcommand{\set}[1]{\left\{ #1 \right\}}

\newcommand{\tor}{\operatorname{Tor}}
\newcommand{\noun}[1]{\textsc{#1}}

\parindent=0in
\parskip=0.5\baselineskip

% LOOK HERE
% change assignment number and possibly date below
\newcommand\header{{\sc Math 631 \hfill Homework 12 \hfill December 12, 2008}}

\begin{document}

\header

\section*{Section 12.3}

\begin{itemize}

\item[4.] Prove that the Jordan canonical form for the matrix 
$
\begin{pmatrix}
~~9 & ~~4 & ~~5\\
-4 & ~~0 & -3\\
-6 & -4 & -2
\end{pmatrix}
$
is that given at the beginning of the chapter.  Explicitly determine why this
matrix cannot be diagonalized.

\begin{proof}
Let $A$ be the matrix given above, then the Jordan canonical form of 
$A$ is
$$
J =
\begin{pmatrix}
3 & 0 & 0\\
0 & 2 & 1\\
0 & 0 & 2
\end{pmatrix},
$$
and $P =
\begin{pmatrix}
~~3 & ~~4 & -2\\
-2 & -2 & ~~2\\
-2 & -4 & ~~2
\end{pmatrix}
$ 
puts $A$ in its Jordan form.

This matrix can not be diagonalized because the geometric multiplicity of
the eigenvalue $\lambda = 2$ is only one, while its algebraic multiplicity
is $2$.  This can be determined from the Jordan block for $\lambda = 2$.
\end{proof}

\item[6.]  Determine which of the four matrices are similar:
$$
A = \begin{pmatrix}
-1 & ~~ 4 & -4\\
~~2 & -1 & ~~3\\
~~0 & -4 & ~~3
\end{pmatrix}, \ 
B = \begin{pmatrix}
-3 & -4 & 0\\
~~2 & ~~3 & 0\\
~~8 & ~~8 & 1
\end{pmatrix}, \ 
C = \begin{pmatrix}
   -3 & ~~2 & -4\\
~~2 & ~~1 & ~~0\\
~~3 &    -1 & ~~3
\end{pmatrix}, \ 
D = \begin{pmatrix}
   -1 & ~~4 & -4\\
~~0 &    -3 & ~~2\\
~~0 &    -4 & ~~3
\end{pmatrix}.
$$

\begin{proof}
Put each of these matrices in Jordan canonical form and then use
Theorem 23 on page 493.

The Jordan canonical forms are, respectively,
$$
J_A = \begin{pmatrix}
-1 & 0 & 0\\
~~0 & 1 & 1\\
~~0 & 0 & 1
\end{pmatrix}, \ 
J_B = \begin{pmatrix}
-1 & 0 & 0\\
~~0 & 1 & 0\\
~~0 & 0 & 1
\end{pmatrix}, \ 
J_C = \begin{pmatrix}
-1 & 0 & 0\\
~~0 & 1 & 1\\
~~0 & 0 & 1
\end{pmatrix}, \ 
J_D = \begin{pmatrix}
-1 & ~~0 & 0\\
~~0 & -1 & 0\\
~~0 & ~~0 & 1
\end{pmatrix}.  
$$
\end{proof}
From this canonical form, we see that only $A$ and $C$ are similar.
We also see that $A$ and $C$ are not diagonalizable, but both $B$
and $D$ are.  However, $B$ and $D$ have different (multi)-sets of eigenvalues.
\end{itemize}


\end{document}
