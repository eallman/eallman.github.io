\documentclass[10pt]{article}

\usepackage[margin=1in, head=1in]{geometry}
\usepackage{amsmath, amssymb, amsthm}
\usepackage{fancyhdr}
\usepackage{graphicx}

% if using a MAC, you may want to uncomment the following line
% to enable reverse searches.
%\usepackage{pdfsync}

% Headers and footers
\fancyhf{}
\rfoot{\thepage}

%\setcounter{secnumdepth}{0}

% macros for algebra class
\renewcommand{\theenumi}{\alph{enumi}}
\renewcommand{\emptyset}{\varnothing}
\newcommand{\R}{\mathbb{R}}
\newcommand{\C}{\mathbb{C}}
\newcommand{\Z}{\mathbb{Z}}
\newcommand{\N}{\mathbb{N}}
\newcommand{\Q}{\mathbb{Q}}
\renewcommand{\b}{\textbf}
\newcommand{\re}{\text{Re}}
\newcommand{\im}{\text{Im}}
\renewcommand{\iff}{\Leftrightarrow}
\newcommand{\zbar}{\overline{z}}
\newcommand\SL{\operatorname{SL}}
\newcommand\GL{\operatorname{GL}}
\newcommand{\divides}{\, \Big | \,}

\newcommand{\normsubeq}{\trianglelefteq}
\newcommand{\normsub}{\triangleleft}
\newcommand{\gen}[1]{\left\langle #1 \right\rangle}
\newcommand{\sect}[1]{\vspace{.25in}\noindent\textbf{Section #1}}
\renewcommand{\phi}{\varphi}
\renewcommand{\epsilon}{\varepsilon}
\newcommand{\zmod}[1]{\Z/#1 \Z}
\newcommand{\la}{\langle}
\newcommand{\ra}{\rangle}
\newcommand\inv{^{-1}}
\newcommand{\Aut}{\text{Aut}}
\newcommand{\Inn}{\text{Inn}}
\renewcommand{\char}{\text{ char }}
\newcommand{\Syl}{\operatorname{Syl}}
\newcommand{\set}[1]{\left\{ #1 \right\}}

\parindent=0in
\parskip=0.5\baselineskip

% LOOK HERE
% change assignment number and possibly date below
\newcommand\header{{\sc Math 631 \hfill Homework 10 \hfill November 19, 2008}}

\begin{document}

\header

\section*{Section 8.2}

\begin{itemize}

\item[5.] (Lawless) Let $R$ be the quadratic integer ring $\Z[\sqrt{-5}]$. Define the ideals $I_2 = ( 2, 1+\sqrt{-5} )$, $I_3 = ( 3, 2+\sqrt{-5})$, and $I'_3 = ( 3, 2 - \sqrt{-5})$.
\begin{itemize}
\item[(a)] Prove that $I_2$, $I_3$, and $I_3'$ are nonprincipal ideals in $R$. 
\begin{proof}
Assume $I_2$ is a principal ideal. This would imply there is some $a + b\sqrt{-5}$ such that 
\begin{equation}\label{1}
2 = \alpha(a+b\sqrt{-5})
\end{equation}
\begin{equation}\label{2}
1+\sqrt{-5} = \beta(a+b\sqrt{-5})
\end{equation}
Taking the norms on both side of equation \eqref{1} give us $4 = N(\alpha)(a^2+5b^2)$. So $a^2 + 5b^2 =$ 1,2, or 4.

If the value of $a^2 + 5b^2$ is 4, then $N(\alpha) = 1$, and so $\alpha = \pm 1$. So $a+b\sqrt{-5} = \pm 4$. However, this is impossible, since 4 does not divide the coefficients of $1+\sqrt{-5}$, as is required by \eqref{2}. 

The value of $a^2+5b^2$ cannot be 2, since there are no integer solutions to $a^2 + 5b^2 = 2$. This leaves $a^2 + 5b^2 = 1$. Then $a+b\sqrt{-5} = \pm 1$, and so $1 \in I_2$. Thus, there exists some $\gamma,\delta \in \Z[\sqrt{-5}]$ such that 
$$2\gamma + (1+\sqrt{-5})\delta = 1.$$
Multiplying both sides by $1 - \sqrt{-5}$ gives
$$(1 - \sqrt{-5})2\gamma + 6\delta = 1 - \sqrt{-5}.$$
Since both terms on the left hand side are divisible by 2, this would imply $1-\sqrt{-5}$ is a multiple of 2 in $\Z[\sqrt{-5}]$, a contradiction. 

Therefore, $I_2 = (2,1+\sqrt{-5})$ is not a principal ideal. The proof that $I_3$ is not principal is given on p.273 in the book, and the proof that $I_3'$ is nonprincipal is similar. 
\end{proof}


\item[(b)] Prove that the product of two nonprincipal ideals can be principal by showing $I_2^2$ is the principal ideal generated by 2. 
\begin{proof}
We will show $(2) = I_2^2$. We first show $I_2 \subseteq (2)$. Recall that if $I,J$ are ideals in a ring $R$, then $IJ = ( \{ab \,|\, a \in I, b \in J\})$. So $I_2^2$ is generated by the elements $2^2 = 4$, $2(1+\sqrt{-5})$, and $(1 + \sqrt{-5})^2 = 1 + 2\sqrt{-5} + 5 = 6 + 2\sqrt{-5} = 2(3 + \sqrt{-5})$. Thus $I_2^2$ is generated by elements of the form $2(a+b\sqrt{-5})$, for some $a,b \in \Z$. Therefore, $I_2^2 \subseteq (2)$.

Since $1+\sqrt{-5} \in I_2$, and $1 - \sqrt{-5} = 2 - (1 + \sqrt{-5}) \in I_2$, then $6 = (1 + \sqrt{-5})(1 - \sqrt{-5}) \in I_2^2$. So we have $6,4 \in I_2^2$, and so $2 = 6 - 4 \in I_2^2$. Thus $(2) \subseteq I_2^2$, and so $(2) = I_2^2$. 
\end{proof}


\item[(c)] Prove that $I_2I_3 = (1 - \sqrt{-5})$ and $I_2I_3' = (1+\sqrt{-5})$ are principal. Conclude $(6) = I_2^2I_3I_3'$. 

First, we show $I_2I_3 = (1 - \sqrt{-5})$. Notice $I_2I_3$ will be generated 
by the elements $(2)(3) = 6 = (1-\sqrt{-5})(1+\sqrt{-5})$, $2(2 + \sqrt{-5}) = 
4+2\sqrt{-5} = (1-\sqrt{-5})(-1+\sqrt{-5})$, $3(1+\sqrt{-5}) = 3+3\sqrt{-5} = 
(1-\sqrt{-5})(-2+\sqrt{-5})$ and $(2+\sqrt{-5})(1+\sqrt{-5}) = -3 + 3\sqrt{-5} 
= (1-\sqrt{-5})(-3)$. So every element in $I_2I_3$ can be written in the form 
$(1 - \sqrt{-5})(a+b\sqrt{-5})$ for $a,b \in \Z$, and so $I_2I_3 \subseteq 
(1-\sqrt{-5})$. Since $1-\sqrt{-5} = 4+2\sqrt{-5} - (3+3\sqrt{-5})  \in 
I_2I_3$, we get $(1-\sqrt{-5}) \subseteq I_2I_3$, and so $(1-\sqrt{-5}) = 
I_2I_3$, and so $I_2I_3$ is principal.
\\


By a similar argument, we also get $I_2I_3' = (1+\sqrt{-5})$ is also principal. It remains to show $(6) = I_2^2I_3I_3'$. Since $Z$ is commutative, we have $(I_2I_3)(I_2I_3') = I_2^2I_3I_3'$. Thus, $I_2^2I_3I_3'$ is generated by $(1-\sqrt{-5})(1+\sqrt{-5}) = 6$. As such, $I_2^2I_3I_3' = (6)$. 

\end{itemize}

\end{itemize}

\section*{Section 8.3}

\begin{itemize}

\item[3.] Determine all the representations of the integer $2130797 = 17^2\cdot73\cdot101$ as a sum of two squares.

\begin{align*}
(\pm851)^2 &+ (\pm1186)^2\\
(\pm1069)^2 &+ (\pm994)^2\\
(\pm1411)^2 &+ (\pm374)^2\\
(\pm1309)^2 &+ (\pm646)^2\\
(\pm1421)^2 &+ (\pm334)^2\\
(\pm1459)^2 &+ (\pm46)^2
\end{align*}
Each of the six equations above gives us four representations.  We
can also swap the order of the terms to double the number of
representations.  This gives us a total of 48 representations.


\item[6.]
\begin{enumerate}
\item[(a)] Prove that the quotient ring $\Z[i]/(1+i)$ is a field of order 2.
\item[(b)] Let $q \in \Z$ be a prime with $q \equiv 3 \bmod 4$. Prove that the quotient ring $\Z[i]/(q)$ is a field with $q^2$ elements.
\item[(c)] Let $p \in \Z$ be a prime with $p \equiv 1 \bmod 4$ and write $p=\pi \bar{\pi}$ as in Proposition 18. Show that the hypotheses for the Chinese Remainder Theorem (Theorem 17 in Section 7.6) are satisfied and that $\Z[i]/(p) \cong \Z[i]/(\pi) \times \Z[i]/(\bar{\pi})$ as rings. Show that the quotient ring $\Z[i]/(p)$ has order $p^2$ and conclude that $\Z[i]/(\pi)$ and $\Z[i]/(\bar{\pi})$ are both fields of order $p$.
\item[(a)]
\begin{proof} (Bastille)
 By Proposition 18, $1+i$ is irreducible in $\Z[i]$, therefore it is prime since $\Z[i]$ is a PID (p.290 and Proposition 11). So $(1+i)$ is a prime ideal, and hence by Proposition 7, $(1+i)$ is a maximal ideal. Therefore $\Z[i]/(1+i)$ is a field. We now show that it is of order 2.

\underline{Long version} Consider the following map:
\begin{align*}
\varphi:\quad   &\Z[i] \to Z_2 \\
                &\varphi(a+bi)=(a^2+b^2) \bmod 2.
\end{align*}
We verify that $\varphi$ is a surjective ring homomorphism since for
all $a+bi, c+di \in \Z[i]$:
    \begin{align*}
            \varphi((a+bi)+(c+di))  &=\varphi(a+c+(b+d)i)=((a+c)^2+(b+d)^2)\bmod 2 \\
                                                            &=(a^2+2ac+c^2+b^2+2bd+d^2)\bmod 2= (a^2+b^2+c^2+d^2)\bmod 2 \\
                                                            &=(a^2+b^2)\bmod 2 +(c^2+d^2)\bmod 2 = \varphi(a+bi)+\varphi(c+di); \\
            \varphi((a+bi)(c+di))       &= \varphi(ac-bd+(ad+bc)i)=((ac-bd)^2+(ad+bc)^2)\bmod 2 \\
                                                            &= (a^2c^2-2abcd+b^2d^2+a^2d^2+2abcd+b^2c^2) \bmod 2 \\
                                                            &= (a^2(c^2+d^2)+b^2(c^2+d^2))\bmod 2= ((a^2+b^2)(c^2+d^2))\bmod 2 \\
                                                            &=(a^2+b^2)\bmod 2 \cdot (c^2+d^2)\bmod 2=\varphi(a+bi)\varphi(c+di); \\
            \varphi(1)=1 \quad \varphi(2)=0.
    \end{align*}
We claim that $\ker \varphi = (1+i)$. Note that because norms are
multiplicative, if $a+bi=\alpha(1+i)$ for some $\alpha \in \Z[i]$,
then $a^2+b^2=N(a+bi)=N(\alpha)N(1+i)=2N(\alpha)$ and therefore
$a+bi \in \ker \varphi$. Thus $(1+i) \subseteq \ker \varphi$.
Conversely, if $a^2+b^2=2k$, then either both $a,b$ are odd or both
$a,b$ are even. In both cases, $a+b$, $b-a$ are even and
$$a+bi=\left(\frac{a+b}{2}+\frac{b-a}{2}i\right)(1+i),$$ hence $a+bi
\in (1+i)$ and $\ker \varphi \subseteq (1+i)$. Thus $\ker
\varphi=(1+i)$. Therefore, by the First Isomorphism Theorem,
$\Z[i]/(1+i) \cong Z_2$ so $\Z[i]/(1+i)$ is a field of order 2.

\underline{Short version} Because $\Z[i]$ is a Euclidean domain, for
any $c \in \Z[i]$, there exist $\alpha, r \in \Z[i]$ such that:
$$c=\alpha(1+i)+r \quad \text{ where } \quad r=0 \quad \text{ or } \quad N(r)<N(1+i)=2.$$
If $r=0$ then $c \in (1+i)$, and if $N(r)=1$ then $r$ is a unit,
i.e. $r= \pm 1, \pm i$. We claim
$1+(1+i)=i+(1+i)=-1+(1+i)=-i+(1+i)$. Indeed note that $1-i=-i(1+i)
\in (1+i)$ so $1+(1+i)=i+(1+i)$, and similarly $1+i=1-(-i)=i-(-1)
\in (1+i)$ so $1+(1+i)=-i+(1+i)$ and $i+(1+i)=-1+(1+i)$. So all
these cosets are equal. Hence
$$\Z[i]/(1+i)=\left\{(1+i),1+(1+i)\right\}, $$ and so $\Z[i]/(1+i)$
is a field of order 2.
\end{proof}
\item[(b)]
\begin{proof} (Bastille) By Proposition 18, $q$ is irreducible in $\Z[i]$ so $q$ is prime since $\Z[i]$ is a PID. Furthermore, the prime ideal $(q)$ is maximal by Proposition 7. Therefore $\Z[i]/(q)$ is a field. Because $\Z$ is a Euclidean domain, for any $c,d \in \Z$, there exist $b_1,b_2,r_1,r_2 \in \Z$ such that:
\begin{align*}
c&=b_1q+r_1 \quad \text{ and } \quad 0 \leq r_1 < q, \\
d&=b_2q+r_2 \quad \text{ and } \quad 0 \leq r_2 < q.
\end{align*}
Therefore for any $c+di \in \Z[i]$, we have that $c+di+(q)= b_1q+r_1
+ (b_2q+r_2)i+(q)=(b_1+b_2i)q+r_1+r_2i+(q)= r_1+r_2i+(q).$ So
$$\Z[i]/(q)=\left\{r_1+r_2i \ | \ r_1,r_2 \in \Z, \quad 0 \leq r_1,r_2 < q \right\}. $$
So we have $q$ choices for $r_1$, and $q$ choices for $r_2$ so we
need only show that all $q^2$ cosets thus formed are distinct.
Assume $r_1,r_1^{'},r_2,r_2^{'} \in \Z$ such that $0 \leq
r_1,r_1^{'},r_2,r_2^{'} <q$. Then
\begin{align*}
r_1+r_2i+(q)=r_1^{'}+r_2^{'}i+(q) \quad &\Leftrightarrow \quad r_1-r_1^{'}+(r_2-r_2^{'})i \in (q) \\
                                                                                &\Leftrightarrow \quad r_1-r_1^{'}+(r_2-r_2^{'})i =q(a+bi)=qa+qbi \text{    where    } a,b \in \Z.
\end{align*}
But $0 \leq r_1-r_1^{'}<q$, and $0 \leq r_2-r_2^{'} < q$ so we must
have $a=b=0$ and hence $r_1=r_1^{'}$ and $r_2=r^{'}_2$. This in
turns implies that all cosets described are distinct, and therefore
$\left| \Z[i]/(q)\right|=q^2$.
\end{proof}
\item[(c)]
\begin{proof} (Bastille) To verify the hypotheses of the Chinese Remainder Theorem, we need only show that $(\pi)$ and $(\bar{\pi})$ are comaximal ideals in $\Z[i]$ since we aready have that $\Z[i]$ is a commutative ring with 1. Note that, as in part (a), because $\pi=a+bi$ and $\bar{\pi}=a-bi$ are irreducible in $\Z[i]$, a PID, $(\pi), (\bar{\pi})$ are maximal ideals in $\Z[i]$ (by Proposition 18, 11, and 7). So we need only show that one is not a subset of the other to conclude that their sum is the whole ring. Suppose to the contrary that $\pi=\alpha\bar{\pi}$, then $\alpha$ would have to be a unit since $\pi$ is irreducible. But we are given that they are distinct irreducible, hence they cannot be associates. Therefore $\pi \notin (\bar{\pi})$ and hence, $(\pi)+(\bar{\pi})=\Z[i]$, i.e. $(\pi)$ and $(\bar{\pi})$ are comaximal. Then it follows from the Chinese Remainder Theorem that the map  $\Z[i] \to \Z[i]/(\pi)\times \Z[i]/(\bar{\pi})$ defined by $r \longmapsto (r+(\pi), r+(\bar{\pi}))$ is a surjective ring homomorphism with kernel: $(\pi)\cap (\bar{\pi})=(\pi)(\bar{\pi})=(\pi \bar{\pi})=(p)$. So by the First Isomorphism Theorem,
$$ \Z[i]/(p) \cong \Z[i]/(\pi)\times \Z[i]/(\bar{\pi}).$$
Using the division algorithm in $\Z$ -- the same argument as in part
(b) but for $p$ instead of $q$ -- we can conclude that $\Z[i]/(p)$
has order $p^2$. And because $(\pi), (\bar{\pi})$ are maximal ideals
in $\Z[i]$, we can conclude again that $\Z[i]/(\pi)$ and
$\Z[i]/(\bar{\pi})$ are fields. Furthermore
$\left|\Z[i]/(\pi)\right| \neq 1$, $\left|\Z[i]/(\bar{\pi})\right|
\neq 1$ otherwise $(\pi)$ (respectively $(\bar{\pi})$) equals
$\Z[i]=(1)$ but 1 and $\pi$ (resp. $\bar{\pi}$) can not be
associates since $\pi$ (resp. $\bar{\pi}$) is irreducible and 1 is a
unit. Therefore we must have
$\left|\Z[i]/(\pi)\right|=\left|\Z[i]/(\bar{\pi})\right|=p$.
\end{proof}
\end{enumerate}
\end{itemize}

\section*{Section 9.1}

\begin{itemize}

\item[4.]  Prove that the ideals $(x)$ and $(x,y)$ are prime ideals in $\Q[x,y]$ but only the latter ideal is a maximal ideal.

\begin{proof}(Mobley) \ Let $\phi : \Q [x,y] \rightarrow \Q[y]$ such that
$\phi(x)=0$, $\phi(y)=y$ and $\phi(q)=q$ for all $q \in \Q$.  Then
any polynomial in $\Q[x,y]$ with a term that has an $x$ will go to
zero and the only terms that remain are those with y's and
constants.  We compute the $\ker \phi = (x)$.  By the First
Isomorphism Theorem, $\Q [x,y]/(x) \cong \Q[y]$.  Since $\Q[y]$ is
an integral domain, it follows that $\Q [x,y]/(x)$ is as well.  By
Proposition 13 on page 255 of the text $(x)$ is a prime ideal.

Now consider $\Gamma : \Q [x,y]\rightarrow \Q$ such that
$\Gamma(x)=0$, $\Gamma(y)=0$ and $\Gamma(1)=1$.  Thus any term in
$\Q[x,y]$ that has an $x$ or $y$ will go to zero and only rationals
are left.  We can see that $\ker\Gamma = (x,y)$.  Using the First
Isomorphism Theorem again, we have that $\Q [x,y]/(x,y) \cong \Q$.
Since $\Q$ is a field so is $\Q [x,y]/(x,y)$.  Then it follows that
$(x,y)$ is a maximal ideal and also a prime ideal.

\end{proof}

\item[9.]  Prove that a polynomial ring in infinitely many variables with coefficients in any commutatitive ring contains ideals that are not finitely generated.


\begin{proof}(Mobley) \ Let $R[x_1, x_2, ...]$ be a commutative ring.
We need to show that the ideal $I=(x_1, x_2, ... , x_n,...)\subseteq
R[x_1, x_2, ...]$ is not finitely generated.  To this end, suppose
to the contrary that $I$ is finitely generated.  Then $I$ is
generated by a finite number of polynomials, $I=<p_1, p_2, ...
,p_n>$.  Note that there is a finite number of variables $x_i$
appearing in any $p_i$.  Assume $k=\max \lbrace i \mid x_i$ \text{
such that } $x_i$ \text{is a variable in} $p_i\rbrace$.  Then
$I\subseteq <x_1, x_2, ... ,x_k>$.

We claim that $x_{k+1}\notin<x_1, x_2, ... ,x_k>$ and therefore
$x_{k+1}\notin I$.  Suppose $x_{k+1}\in I$.  Then for ring elements
$g_i$,

$$x_{k+1}=\Sigma_{i=1}^{n} g_ip_i.$$

First we collect all the terms on the right side of the equation
that have an $x_1$ in them.  After factoring out the $x_1$ term from
these, we have $c_1x_1$ such that

$$ \Sigma _{i=1}^{n} g_ip_i - c_1x_1$$

where $\Sigma_{i=1}^{n}g_ip_i - c_1x_1$ no longer contains the
variable $x_1$.  We continue doing this again for the next variable
$x_2$ which results in $\Sigma _{i=1}^{n} g_ip_i - c_2x_2 - c_1x_1$.
Notice that $\Sigma _{i=1}^{n} g_ip_i - c_2x_2 - c_1x_1$ no longer
contains the variables $x_1$ or $x_2$.  We follow the same procedure
for $x_3$ (which results in $ \Sigma _{i=1}^{n} g_ip_i - c_3x_3 -
c_2x_2 - c_1x_1$) and so on until we have finished the process with
$x_k$.  Then we have

$$x_{k+1}=\Sigma_{i=1}^{n} g_ip_i=c_1x_1+c_2x_2+...+c_kx_k.$$

But we realize that on the left most side of the equation there are
no terms with $x_1$ and therefore $c_1=0$.  This is true for all
$x_i$ with $i=\lbrace1,2,...,k\rbrace$ and therefore $c_i=0$ for
$i=\lbrace1,2,...,k\rbrace$.  But then $x_{k+1}=0$ and we have a
contradiction.  Therefore a polynomial ring in infinitely many
variables with coefficients in any commutatitive ring contains
ideals that are not finitely generated.


\end{proof}

\item [13.] Prove that the rings $F\left[x,y\right]/\left(y^{2}-x\right)$
and $F\left[x,y\right]/\left(y^{2}-x^{2}\right)$ are not isomorphic
for any field $F$.

\begin{proof}[Proof (Granade)]
Note that
$\left(y^{2}-x^{2}\right)=\left(y-x\right)\left(y+x\right)$ is
reducible. Thus, $F\left[x,y\right]/\left(y^{2}-x^{2}\right)$
contains zero-divisors. Concretely,
$\overline{\left(y-x\right)}\overline{\left(y+x\right)}=\overline{y^{2}-x^{2}}=\overline{0}$.

By contrast, we claim that $y^{2}-x$ is not reducible, and hence
$F\left[x,y\right]/\left(y^{2}-x\right)$ is a field. To see this,
suppose that $y^{2}-x=f\left(x,y\right)g\left(x,y\right)$ is a
non-trivial factorization for some $f,g\in F\left[x,y\right]$. Then,
since $y^{2}-x$ has multidegree $\left(1,2\right)$, we must have
that one of $f$ and $g$ has degree $ $$0$ in $x$ and that the other
must have degree 1 in $x$. Without loss of generality, let
$f\left(x,y\right)$ have degree 1 in $x$. Thus,
$g\left(x,y\right)=g\left(y\right)$. Since this factorization is
non-trivial, $\deg g\left(y\right)\ge1$, and so
$y^{2}-x=f\left(x,y\right)\cdot yg_{0}\left(y\right)$ for some
$g_{0}\in F\left[y\right]$. This is a contradiction, as $y\nmid x$.
We conclude that $y^{2}-x$ is irreducible as claimed.

Since $F\left[x,y\right]/\left(y^{2}-x\right)$ is a field but
$F\left[x,y\right]/\left(y^{2}-x^{2}\right)$ has zero-divisors, they
cannot be isomorphic for any field $F$.
\end{proof}

\end{itemize}

\section*{Section 9.2}

\begin{itemize}

\item [1.] Let $f\left(x\right)\in F\left[x\right]$ be a polynomial of
degree $n\ge1$ and let bars denote passage to the quotient
$F\left[x\right]/\left(f\left(x\right)\right)$. Prove that for each
$\overline{g\left(x\right)}$ there is a unique polynomial
$g_{0}\left(x\right)$ of degree $\le n-1$ such that
$\overline{g\left(x\right)}=\overline{g_{0}\left(x\right)}$
(equivilantly, the elements
$\overline{1},\overline{x},\dots,\overline{x^{n-1}}$ are a
\emph{basis} of the vectorspace
$F\left[x\right]/\left(f\left(x\right)\right)$ over $F$ --- in
particular, the dimension of this space is $n$).

\begin{proof} [Proof (Granade)]
Let $\overline{g\left(x\right)}\in
F\left[x\right]/\left(f\left(x\right)\right)$. Then, by the division
algorithm on $F\left[x\right]$, there exist unique polynomials
$q\left(x\right)$ and $r\left(x\right)$ such that:\begin{eqnarray*}
g\left(x\right) & = & q\left(x\right)f\left(x\right)+r\left(x\right)\\
\deg r\left(x\right) & < & \deg f\left(x\right)\end{eqnarray*} Thus,
$\overline{g\left(x\right)}=\overline{q\left(x\right)f\left(x\right)+r\left(x\right)}=\overline{q\left(x\right)}f\left(x\right)+\overline{r\left(x\right)}=\overline{r\left(x\right)}$.
Since $r\left(x\right)$ is unique, we are done.
\end{proof}


\item[9.2.2]   Let $F$ be a finite field of order
$q$ and let $f(x)$ be a polynomial in $F[x]$ of degree $n\ge1$, then
$F[x]/(f(x))$ has $q^{n}$elements.

\begin{proof}
(Gillispie) Let $f(x)=a_{n}x^{n}+\cdots+a_{0}\in F[x]$ and let
$I=(f(x))$. By problem 9.2.1 we know that the elements
$\bar{1},\bar{x},\cdots\bar{,x^{n-1}}$ form a basis for the vector
space $F[x]/I$ over $F$. So every element of this vector space may
be expressed as a linear combination of
$\bar{1},\bar{x},\cdots,\bar{x^{n-1}}$ with coefficients from $F$.
Since there are $n$ elements in the basis, and $q$ elements in $F$,
there are $q^{n}$ elements in the vector space, and hence $q^{n}$
elements in $F[x]/I$.

\end{proof}

\item[9.2.3] Let $f(x)$ be a polynomial in $F[x]$,
then $F[x]/(f(x))$ is a field if and only if $f(x)$ is irreducible.

\begin{proof}

(Gillispie) We may assume $f(x) \neq 0$, since $f(x)=0$ is not
irreducible element and $F[x]/(0)$ is not a field.\\
Since $F$ is a field, from Cor. 9.4 we know that $F[x]$ is a PID.
\\
Since $F[x]$ is a PID by Prop. 7.12 and $f(x)\ne0$, $F[x]/(f(x))$ is
a field if and only if $(f(x))$ is a maximal ideal. By corollary
7.14 and Prop. 8.7 $(f(x))$ is maximal if and only if it is a prime
ideal. We also know that $(f(x))$ is prime in $F[x]$ if and only if
$f(x)$ is prime in $F[x]$.Finally from Prop. 8.11 since $F[x]$ is a
PID, $f(x)$ is prime if and only if it is irreducible.
\end{proof}

\item[6.] Describe (briefly) the ring structure of the following rings:
\begin{center}
a. $\Z[x]/(2)$ \ \ b. $\Z[x]/(x)$ \ \ c. $\Z[x]/(x^2)$ \ \ d.
$\Z[x]/(x^2,y^2,2)$
\end{center}
Show that $\alpha^2 = 0$ or 1 for every $\alpha$ in the last ring
and determine those elements with $\alpha^2 = 0$. Determine the
characteristics of each of these rings.
\\ (Baggett)

\begin{itemize}
\item[a.] $\Z[x]/(2) \cong \Z/2\Z[x]$
\\ Since $\Z/2\Z$ is a field, $\Z[x]/(2)$ is a ED, a PID, and a UFD with characteristic 2.
\item[b.] $\Z[x]/(x) \cong \Z$
\\ $\Z[x]/(x)$ is a ED, a PID, and a UFD with characteristic 0.
\item[c.] $\Z[x]/(x^2) \cong \{a + bx$ $|$ $a,b \in \Z$ and $x^2 = 0 \}$
\\ $\Z[x]/(x)$ has characteristic 0.
\item[d.] $\Z[x]/(x^2,y^2,2) \cong \{a + bx + cy + dxy$ $|$ $a,b,c,d \in \Z/2\Z$ and $x^2 = y^2 = 0 \}$.
\\ Let $\alpha \in \Z[x]/(x^2,y^2,2)$. Let $\alpha'$ be the image of $\alpha$ under the above isomorphism,
i.e.
\\ $\alpha' = a + bx + cy + dxy$ with $a,b,c,d \in \Z/2\Z$. Then
\begin{center}
$(\alpha')^2 = (a + bx + cy + dxy)(a + bx + cy + dxy) = a^2 + 2abx +
2acy + 2(ad + bc)xy = a^2$.
\end{center}
If $a = \overline{0}$, then $(\alpha')^2 = \overline{0}$. If $a =
\overline{1}$, then $(\alpha')^2 = \overline{1}$. Hence, $\alpha^2 =
0$ or 1. We have that $\alpha^2 = 0$ if $\alpha = p(x) +
(x^2,y^2,2)$ with the constant term of $p(x)$ being even. Lastly,
$\Z[x]/(x^2,y^2,2)$ has characteristic 2.

\end{itemize}

\item[7.]  Determine all the ideals of the ring $\Z[x]/(2,x^3+1)$.

% display your last name
(Schamel) Note $(2)(x^3+1)/(2) \equiv (x^3+1)$ so the third
isomorphism theorem for rings and proposition 2 of section 9.1 give
us
\[\Z[x]/(2)(x^3+1) \equiv (\Z[x]/(2))/((2)(x^3+1)/(2)) \equiv (\Z/2\Z)[x]/(x^3+1).\]
In $(\Z/2\Z)[x]$ (a U.F.D by corollary 4 of 9.2), we can factor
$(x^3+1)$ into $(x+1)(x^2+x+1)$.  Since $\Z/2\Z$ is a field,
factoring into irreducibles must reduce degree. Hence, $x+1$ is
irreducible since it has degree one, and $x^2+x+1$ must factor into
two degree one polynomials.  The only degree one polynomials in
$(\Z/2\Z)[x]$ are $x$ and $x+1$, but $x\cdot x = x^2$,
$x(x+1)=x^2+x$ and $(x+1)(x+1)= x^2 + 1$, so $x^2+x+1$ is also
irreducible in $(\Z/2\Z)[x]$.  Thus $(\overline{x+1})$ and
$(\overline{x^2+x+1})$ will be distinct proper ideals of
$\Z[x]/(2,x^3+1)$, along with $(\overline{0})$ and $(\overline{1})$.
Furthermore, since $(\Z/2\Z)[x]$ is a U.F.D., it is also a P.I.D,
and hence the only proper ideals of $(\Z/2\Z)[x]$ containing
$(x^3+1)$ are $(x+1)$ and $(x^2+x+1)$.  The correspondence
isomorphism theorem for rings then gives us that there are exactly
two non-trivial proper ideals of $(\Z/2\Z)[x]/(x^3+1)$, so we have
completed our search.
\end{itemize}

\section*{Chapter 9.3}
\begin{itemize}

\item[1.] Let $R$ be an integral domain with quotient field $F$ and let $p(x)$ be a monic polynomial in $R[x]$.  Assume that $p(x) = a(x)b(x)$ where $a(x)$ and $b(x)$ are monic polynomials in $F[x]$ of smaller degree than $p(x)$.  Prove that if $a(x) \notin R[x]$ then $R$ is not a Unique Factorization Domain.  Deduce that $\Z[2\sqrt{2}]$ is not a U.F.D.

\begin{proof}(Schamel)
By way of contradiction, suppose $R$ is a U.F.D.  Then, by Gauss' Lemma, there exist non-zero $\alpha,\beta \in F$ such that $\alpha a(x) = \tilde{a}(x)$ and $\beta b(x) = \tilde{b}(x)$ for some $\tilde{a}(x),\tilde{b}(x) \in R[x]$ such that $p(x) = \tilde{a}(x)\tilde{b}(x)$.  In particular, $p(x) = \alpha \beta a(x) b(x)$.  Since $a(x)$ and $b(x)$ are monic, we have that $a(x) b(x)$ is also monic, and the highest degree coefficients of $p(x)$ and $a(x) b(x)$ give us that $1 = \alpha \beta \cdot 1$.  Hence $\beta = \alpha^{-1}$ and both are units.  But we also have that $a(x)$ and $b(x)$ are monic, so the coefficients of the largest terms of $\tilde{a}(x)$ and $\tilde{b}(x)$ are given by $\alpha$ and $\beta$ respectively, so $\alpha,\beta \in R$.  Hence $\alpha$ and $\beta$ are units in $R$, so $p(x) = \beta \tilde{a}(x) \alpha \tilde{b}(x) = a(x)b(x)$ gives a factorization of $p(x)$ in $R[x]$, and thus $a(x), b(x) \in R[x]$, a contradiction. \\

Now let $R = \Z[2\sqrt{2}] = \set{a + 2b\sqrt{2}: a,b \in \Z}$.  The
field of fractions of $R$ is then $F = \Q[\sqrt{2}]$.  Then the
polynomial $p(x) = x^2 - 2$ is reducible in $F[x]$ to monic terms by
$p(x) = (x+\sqrt{2})(x+\sqrt{2})$.  However, $x+\sqrt{2} \notin
R[x]$, so by our previous result, $\Z[2\sqrt{2}]$ is not a U.F.D.
\end{proof}


\item[3.]  Let $F$ be a field.  Prove that the set $R$ of polynomials in $F[x]$ whose coefficient of $x$ is equal to 0 is a subring of $F[x]$ and that $R$ is not a U.F.D.

\begin{proof}(Buchholz)

Let $R$ be a set of polynomials in $F[x]$ whose coefficient of $x$
is equal to 0.  First note that $0\in R$ and therefore non-empty.
Now we must show that $R$ is closed under subtraction and
multiplication.  Let $f(x),g(x)\in R$ where
$f(x)=a_0+a_2x^2+a_3x^3+\cdots$ and $g(x)=b_0+b_2x^2+b_3x^3+\cdots$.
Then
\begin{align*}
f(x)-g(x)&=(a_0+a_2x^2+a_3x^3+\cdots)-(b_0+b_2x^2+b_3x^3+\cdots)\\
&=(a_0-b_0)+(a_2-b_2)x^2+(a_3-b_3)x^3+\cdots,
\end{align*}
which is contained in $R$.  Hence $f(x)-g(x)\in R$.  Now consider,
\begin{align*}
f(x)g(x)&=(a_0+a_2x^2+a_3x^3+\cdots)(b_0+b_2x^2+b_3x^3+\cdots)\\
&=(a_0b_0)+(a_2b_2)x^2+\cdots,
\end{align*}
which is contained in $R$.  Hence $f(x)g(x)\in R$.  Therefore $R$ is
a subring of $F[x]$. Now we must show that $R$ is not a U.F.D.
First note that $x^2$ and $x^3$ are irreducible elements since
$x\notin R$.  So $x^6$ can be written as $(x^2)^3=x^2x^2x^2$ and
$(x^3)^2=x^3x^3$.  But $x^2$ and $x^3$ are not associates.  Hence
$R$ is not a U.F.D.
\end{proof}

\end{itemize}

\end{document}
