\documentclass[11pt]{report}

\usepackage{pdfsync, color}
\usepackage{amsmath, amssymb, amsfonts}
\setlength{\textwidth}{6in}
\setlength{\oddsidemargin}{.25in}

\newcommand\ZZ{\mathbb Z}
\newcommand\QQ{\mathbb Q}
\newcommand\EE{\mathbb E}
\newcommand\gen{\sqrt[3]{2}}
\newcommand\Rp{R_{\mathfrak p}}
\newcommand\Rf{R_f}

\begin{document}

\thispagestyle{empty}

\begin{center}
{\sc Galois Theory

\smallskip

Homework problems

\smallskip

due Thursday, February 12}

\end{center}

\bigskip

\begin{enumerate}

\item Consider the set $V = \QQ(\gen) = \left\{ a + b \gen + c {(\gen)}^2 \mid a, b, c \in \QQ \right\}$.
Show that $V$ is a vector space over $\QQ$ with basis $\{1, \gen, {(\gen)}^2 \}$.

\smallskip

As an alternative, you can prove the more general statement:  Suppose that $\alpha$ is a root 
of an irreducible polynomial $f(x) = x^3 + n$ for $n$ some integer.   Prove that $\{1, \alpha,
\alpha^2 \}$ is a basis for $V = \{ a_0 + a_1 \alpha + \dots + a_s \alpha^s \mid a_i \in \QQ, \, 
s \text{ a non-negative integer} \}$,
the collection of polynomials in $\alpha$ with rational coefficients.  Thus, $V$ has
dimension $3$ as a vector space.

\item Assume $R$ is a commutative ring with $1$.  Let $\mathfrak{S}$ be a multiplicatively
closed set.  

\begin{enumerate}

\item Suppose that $\mathfrak{p}$ is a prime ideal of $R$ and set $\mathfrak{S} = 
R \smallsetminus \mathfrak{p}$.

\begin{enumerate}

\item Show that $\mathfrak S$ is multiplicatively closed with $1$.

\item Prove the converse to (i):  Namely, that if $\frak A$ is an ideal of
$R$ and $\mathfrak S = R \smallsetminus \frak{A}$ is a multiplicatively
closed set with $1$, then $\mathfrak{A}$ is a prime ideal.
(The upshot of this is that localization of rings takes place at
\emph{prime} ideals of $R$, if you require that $\mathfrak S$ contain
$1$.)

\item Define the \emph{localization $\Rp$ of $R$ at $\mathfrak{p}$} with elements
$$
\Rp = \left\{ \left[ \frac{r}{s} \right] \mid r \in R, \, s \in \mathfrak{S} \right\}.
$$

\begin{enumerate}

\item By consulting a book, or better yet, thinking about the construction of the
quotient field of a domain, define the appropriate equivalence relationship 
for the elements $\frac{r}{s} = (r,s) \in R \times \frak{S}$ in the classes listed above.
(\emph{Hint:} Be a tad careful here.  In problem, vi (c) below, we will allow $\frak S$
to have zero divisors.)
\end{enumerate}

\item Convince yourself that $\Rp$ is a ring.  Convince me that you have done
this, but showing me the definition of $\, \cdot \,$ in $\Rp$.

\item What are the units of $\Rp$?

\item Assume further that $R = \ZZ$ and $\mathfrak{p} = (5)$.  What
are the elements of $\ZZ_{(5)}$?  (You can describe them explicitly.)
What are the units of $\ZZ_{(5)}$?  
What are the prime ideals of $\ZZ_{(5)}$?  What are all
the ideals of $\ZZ_{(5)}$?  Draw the ideal lattice diagram for $\ZZ_{(5)}$.

\end{enumerate}

\item Let $\EE$ denote the positive even integers in the ring $R = \ZZ$.  Note
that $\EE$ is multiplicatively closed.

Show that the \emph{localization of $\ZZ$ at $\EE \doteq \EE^{-1} \ZZ$}
is the rational numbers $\QQ$.

\item Let $f$ be any element of $R$ and $\mathfrak{S} = \left\{ f^n \mid n \in \ZZ^+ \cup \{0\} \right\}$.
Define $\Rf$, the \emph{localization of $R$ at $f$}, to be the set of equivalence
classes of the form $\left[ \frac{r}{f^n} \right]$ for $r \in R, \, n \ge 0$ in the `usual' way.


\begin{enumerate}

\item Show that $f$ is nilpotent if, and only if, $\Rf = 0$.

\item Show that if $f$ is not nilpotent, then $f$ becomes a unit in $\Rf$.

\end{enumerate}

\end{enumerate}

\item Express $f(x_1,x_2,x_3) = x_1^3 + x_2^3 + x_3^3$ as a function
        of the elementary symmetric polynomials. 

\end{enumerate}


\end{document}
