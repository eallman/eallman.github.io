\documentclass[presentation]{beamer}

\mode<presentation> {
  \setbeamercovered{transparent}
}

%\usepackage{tipa}
\usepackage{pdfsync}
\usepackage{verbatim}
\usepackage{amsmath}
%\usepackage{showkeys}

\newcommand{\imagespath}{/Users/eallman/Pictures/latex-images/}

\setbeamersize{text margin left=1cm}
\setbeamersize{text margin right=1cm}


%%%%%%%%%%%%%%%%%%%%%%%%%
% For colors
%\usepackage[usenames]{color}

\definecolor{DRed}{rgb}{0.55,0.1,0}  % essentially brown

%\definecolor{myblue}{rgb}{0.25, 0, 0.75}
\definecolor{myblue}{rgb}{.2, .2, .7}
\definecolor{myorange}{rgb}{1,.5,0}
\definecolor{mypurple}{rgb}{.5,0,.5}
\definecolor{mygreen}{rgb}{0.2656,0.5039,0.2148}
\definecolor{myred}{rgb}{0.75,0,0.25}
\definecolor{r1}{rgb}{.75,0,.25}
%\definecolor{myfootercolor}{rgb}{0.2656,0.5039,0.2148}
\definecolor{myfootercolor}{rgb}{.5,.24,0}
\definecolor{myseqcolor}{rgb}{.5,.25,0}
\definecolor{g2}{rgb}{.5,.5,.5}
\definecolor{g3}{rgb}{.4,.4,.4}

\setbeamercolor{testcolor}{fg=myfootercolor}

%\useoutertheme{infolines}
%\useoutertheme[footline=authortitle]{miniframes}
\setbeamertemplate{navigation symbols}{}

\usetheme{boxes}
\addfootbox{testcolor}{\hskip .5cm Identifying species trees from gene trees}
\addfootbox{testcolor}{\hfill
\insertframenumber/\inserttotalframenumber \hskip .5cm}

\usepackage{amsfonts,amsmath,amssymb}

% ESA
\newcommand\dsp{\displaystyle}
\newcommand\image{\operatorname{Im}}
\newcommand{\R}{\mathbb{R}}
\newcommand{\CC}{\mathbb{C}}
\newcommand\diag{\operatorname{diag}}
\newcommand{\Cof}{\operatorname{Cof}}
\newcommand{\Sec}{\operatorname{Sec}}
\newcommand{\PP}{\mathbb{P}}
\newcommand{\GL}{\operatorname{GL}(k,\CC)}
\newcommand{\Diag}{\operatorname{Diag}}
\newcommand{\adj}{\operatorname{adj}}
\newcommand{\GLr}{\operatorname{GL}(k,\R)}
\newcommand{\GLtwo}{\operatorname{GL}(2,\CC)}
\newcommand{\Flat}{\operatorname {Flat}}
\newcommand{\join}{\operatorname{Join}}

%
% Extra stuff for Eliz and John
\newcommand\bul{\textcolor{myblue}{$\bullet$\thickspace}}
\newcommand\prob{\operatorname{Pr}}
\newcommand{\tc}{\text{:}} % tc=tight colon, for use in math mode in Newick trees
%\newcommand\PP{\mathbb P}


% from Seth
\theoremstyle{plain}
\newtheorem{prop}{Proposition}
\newtheorem{thm}{Theorem}
\newtheorem{Def}{Definition}
\newtheorem{cor}{Corollary}
\newtheorem*{thmstar}{Theorem*}
\newtheorem*{lem}{Lemma}
\newtheorem*{propstar}{Proposition*}
\newtheorem*{conj}{Conjecture}

\theoremstyle{definition}
\newtheorem*{defn}{Definition}
\newtheorem*{ex}{Example}
\newtheorem*{pr}{Problem}
\newtheorem*{alg}{Algorithm}
\newtheorem*{ques}{Question}

% Miscellaneous Special Capitals
%Blackboard Capitals
\newcommand{\zz}{\mathbb{Z}}
\newcommand{\nn}{\mathbb{N}}
\newcommand{\pp}{\mathbb{P}}
\newcommand{\qq}{\mathbb{Q}}
\newcommand{\rr}{\mathbb{R}}
\newcommand{\cc}{\mathbb{C}}
\newcommand{\kk}{\mathbb{K}}
\newcommand{\bbe}{\mathbb{E}}

% Boldface Lowercase
\newcommand{\bfa}{\mathbf{a}}
\newcommand{\bfb}{\mathbf{b}}
\newcommand{\bfc}{\mathbf{c}}
\newcommand{\bfd}{\mathbf{d}}
\newcommand{\bfe}{\mathbf{e}}
\newcommand{\bff}{\mathbf{f}}
\newcommand{\bfh}{\mathbf{h}}
\newcommand{\bfi}{\mathbf{i}}
\newcommand{\bfj}{\mathbf{j}}
\newcommand{\bfm}{\mathbf{m}}
\newcommand{\bfn}{\mathbf{n}}
\newcommand{\bfp}{\mathbf{p}}
\newcommand{\bfq}{\mathbf{q}}
\newcommand{\bfr}{\mathbf{r}}
\newcommand{\bft}{\mathbf{t}}
\newcommand{\bfu}{\mathbf{u}}
\newcommand{\bfv}{\mathbf{v}}
\newcommand{\bfw}{\mathbf{w}}
\newcommand{\bfx}{\mathbf{x}}
\newcommand{\bfy}{\mathbf{y}}
\newcommand{\bfz}{\mathbf{z}}

% Calligraphics
\newcommand{\ca}{\mathcal{A}}
\newcommand{\cf}{\mathcal{F}}
\newcommand{\cg}{\mathcal{G}}
\newcommand{\cm}{\mathcal{M}}
\newcommand{\LL}{\mathcal{L}}
\newcommand{\cn}{\mathcal{N}}
\newcommand{\ci}{\mathcal{I}}
\newcommand{\ce}{\mathcal{E}}
\newcommand{\cC}{\mathcal C}
\newcommand{\cX}{\mathcal X}

\title{Tales from sabbatical:\\ Species trees from gene trees}

\author{Elizabeth S. Allman}

\date{UAF\\December 8, 2011}

\begin{document}

%\small
\baselineskip=5mm

\section{Title}

\frame [plain]{\titlepage}

\frame{\frametitle{Table of contents}\tableofcontents}

\begin{frame}[plain]
% Title page

\textcolor{magenta}{I always make my own title page.}

\textcolor{myblue}{\textbf{Tales from sabbatical:}}

\smallskip

\textcolor{myblue}{\textbf{\hskip 1cm Identifying Species Trees from Gene Trees}}

\vskip .9cm

\begin{minipage}{1.3in}
{\small Elizabeth S.~Allman}

\smallskip

{\footnotesize University of Alaska}

\smallskip

{\footnotesize Fairbanks}

\underbar{\hskip 3cm }

\medskip

{\footnotesize

\smallskip

%Chapman 104

\smallskip

December 8, 2011

}
\end{minipage}
\begin{minipage}{2.7in}
\hskip 1cm
%\includegraphics[height=1.6in]{\imagespath fgBear.eps}
\textcolor{magenta}{Input figure here.  If you use pdf figures, then you should
use pdflatex for texing your file.}
\end{minipage}

\vskip .7cm

\centerline{\emph{UAF Mathematics and Statistics Colloquium}}

\end{frame}

\section{Multispecies coalescent model}

\subsection{columns example}

\begin{frame}
\frametitle{Columns}

Incomplete lineage sorting is modeled by  .....

\medskip

\centerline{\textcolor{myred}{\large \emph{If you come to my talk,}}}

\begin{itemize}

\item Then you will find out.

\bigskip

\begin{columns}

\column{.1cm}

\column{4.5cm} \textcolor{magenta}{This is column 1.}

\column{1.9cm} \textcolor{magenta}{This is column 2.}
\column{1cm}
\thinspace \ 
\end{columns}

\end{itemize}

\end{frame}

\subsection{overlay example}

\begin{frame}
\frametitle{Overlay}

\begin{overlayarea}{\textwidth}{3cm}
\begin{minipage}{3cm}

\only<1>{
\textcolor{magenta}{appears only on slide 1}
}
\only<2>{
\textcolor{magenta}{appears only on slide 2}
}
\end{minipage}
%\end{overlayarea}
\begin{minipage}{.5cm}
\qquad
\end{minipage}
\begin{minipage}{7cm}
Overlay areas avoid ``bobbles" when text is replaced on a slide.

\medskip 

\end{minipage}

\bigskip

\end{overlayarea}

If $k$ lineages enter a population from below, then 

\begin{itemize}

\item lots happens
 

\end{itemize}

\end{frame}

\end{document}


