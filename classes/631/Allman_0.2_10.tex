\documentclass{report}
\usepackage{amsthm, amsmath, amssymb,verbatim,xspace, graphicx}
\usepackage{pdfsync, listings, color}  
\usepackage[normalem]{ulem}

\setlength{\oddsidemargin}{0in} \setlength{\textwidth}{6in}
\setlength{\topmargin}{-0.25in} \setlength{\textheight}{8.5in}

\newcommand{\assignmentNumber}[1]{Homework #1 Selected Solutions}
\newcommand{\duedate}[1]{#1}

\newcommand\R{{\mathbb R}}
\newcommand\Q{{\mathbb Q}}
\newcommand\N{{\mathbb N}}
\newcommand\Z{{\mathbb Z}}
\newcommand\Zn{\mathbb Z / n \mathbb Z}
\newcommand\eP{\varphi}
\newcommand\dsp{\displaystyle}

\title{Student Solution Template}


\theoremstyle{plain}
\newtheorem{thm}{Theorem}
\newtheorem{lem}[thm]{Lemma}
\newtheorem{prop}[thm]{Proposition}
\newtheorem{cor}[thm]{Corollary}
\newtheorem{conj}{Conjecture}
\newtheorem{quest}{Question}
\newtheorem*{rem}{Remark}

\date{August 24, 2018}

\begin{document} 

% Change assignment number below
\centerline{\sc \Large \assignmentNumber{$0$}}
\smallskip
% Change due date below
\centerline{\duedate{August 31, 2018}}

\begin{enumerate}

\item[\S 0.2: 10.] Prove for any given positive integer $N$ there exist only finitely many integers $n$ with 
$\varphi(n) = N$ where $\varphi$ denotes Euler's $\varphi$-function.  Conclude in particular that $\varphi(n)$
tends to infinity as $n$ tends to infinity.

\begin{proof} (Allman) 

Fix $N \in \Z^+$ and suppose that $n$ is an integer such that $\varphi(n) = N$.  We will show that any such
$n$ is bounded above by some positive integer $K$.  That is, for all $n \in \Z^+$ with $\varphi(n) = N$,
$n \le K$.  It follows immediately that there are only finitely many such $n$.

Note first that if $p \, \vert \, n$, then $p \le N+1$.  This follows from the multiplicative formula for the Euler $\varphi$-function.
Specifically, if $p$ exactly divides $n$, $p \,\Vert \, n$, then $\varphi(p) = p-1 \, \vert \, N$, and if $p^a \, \vert \, n$ for
$a > 1$, then $\varphi(p^a) = p^{a-1} (p-1) \, \vert \, N$.  In both cases, $p$ is at most $N+1$.

Let $p_1 = 2, p_2 = 3, \dots, p_k \le N+1$ be an ordered list of all the primes less than or equal to $N+1$.  We will show that there
is a maximal exponent $a_i$ such that $p^{a_i} \, \vert \, n$.  Supposing this, then $n \le 2^{a_1} 3^{a_2} \cdots p_k^{a_k}$, and there
can be only finitely many such $n$.  For each of the primes $p_i$, $i = 1, \dots, k$, let $e_i$ be the highest power of $p_i$
with $p_i^{e_i} \, \vert \, N$.  Then taking $a_i = e_i +1$, the formula for $\varphi(p^{a_i})$ shows that 
$p^{a_i}$ is the highest power of $p_i$ dividing $n$.

\smallskip

To establish the second claim of the problem, note that since $\varphi(n)$ is not bounded above by any $N \in \Z^+$, $\varphi(n)$ must
go to infinity as $n$ does.
\end{proof}

Indeed, the bound in the proof is \emph{overkill} for most $p_i$.  For a fun test, try finding this
bound for $N= 24$, and then find the $n$ with $\varphi(n) = 24$.

\emph{Solution:} If $N = 24 = 2^3 \cdot 3$, then a list of all the primes less than or equal to $25$ and the prime powers
$p_i^{a_i}$ from the proof are provided in the Table below,

\begin{center}
\begin{tabular}{c || ccccccccc}
$p_i$ & 2 & 3 & 5 & 7 & 11 & 13 & 17 & 19 & 23\\
\hline
$p_i^{a_i}$ & $2^4$ & $3^2$ & 5  & 7 & 11 & 13 & 17 & 19 & 23\\
\end{tabular}
\end{center}

and the bound for $n$ in the proof is the product of all the numbers in the second row!

Now, for finding the $n$ with $\varphi(n) = 24$, we methodically step through the primes, checking the values for which
we might get $\varphi(p^k) \, \vert \, 2^3 \cdot 3$.

For instance, for $p=2$, it is possible that $2, 2^2, 2^3, 2^4$ divide $n$, since the values of $\varphi(2^k)$ are
$1, 2, 2^2, 2^3$ respectively.    The possible values of $p^k$ dividing $n$ and their corresponding contributions
to $\varphi(n)$ are reported below on the left.  On the right, by piecing together divisors of $24$, I have found   
the values of $n$ with $\varphi(n) = 24$.

\end{enumerate}

% end enumerate early just so table fits better
\bigskip

\hskip -.5cm \begin{minipage}{4cm}
\begin{tabular}{c || cccc}
$p$ & $2$ & $2^2$ & $2^3$ & $2^4$\\
\hline
$\varphi(p^k)$ & $1$ & $2$ & $2^2$ & $2^3$\\ 
\end{tabular}

\smallskip

\begin{tabular}{c || cccc}
$p$ & $3$ & $3^2$ \\
\hline
$\varphi(p^k)$ & $2$ & $2 \cdot 3$\\ 
\end{tabular}

\smallskip

\begin{tabular}{c || cccc}
$p$ & 5 \\
\hline
$\varphi(p^k)$ & $2^2$\\
\end{tabular}

\smallskip

\begin{tabular}{c || cccc}
$p$ & 7 \\
\hline
$\varphi(p^k)$ & $2\cdot 3$\\
\end{tabular}

\smallskip

\begin{tabular}{c || cccc}
$p$ & 13 \\
\hline
$\varphi(p^k)$ & $2^2 \cdot 3$\\
\end{tabular}
\end{minipage}
%
\hskip 1cm \begin{minipage}{6cm}
\begin{tabular}{c || l|c}
factorization of $24$ & relevant factors of $n$ which & $n$ and $2n$ if $n$ odd\\
& must be relative prime \\
\hline
$1 \cdot 24$ & impossible \\
$2 \cdot 12$ & $\varphi(x) =2$: \quad 4, 3 & 52, 39, 78\\
&$\varphi(x)=12: \quad 13$ &  \\
$3 \cdot 8$ & impossible\\
$4 \cdot 6$ &$\varphi(x)=4: \quad 2^3, 5$ & 72, 56, 45, 35, 90, 70\\
&$\varphi(x)= 6: \quad 3^2, 7$  \\
$2 \cdot 2 \cdot 6$ & $\varphi(x) =2 \cdot 2: \quad 12 $ & \sout{108}, 84\\
& $\varphi(x)= 6: \quad 3^2, 7$  \\
$2 \cdot 3 \cdot 4$ & impossible\\

\end{tabular}
\end{minipage}


\end{document}