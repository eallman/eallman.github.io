\documentclass{report}

%\setlength{\oddsidemargin}{0.25in}
%\setlength{\textwidth}{6.5in}
%\setlength{\topmargin}{-0.5in}
%\setlength{\textheight}{9in}

\usepackage[margin=.8in]{geometry}

\usepackage{xifthen}
\newboolean{long}
\setboolean{long}{true}

\usepackage{verbatim}

\begin{document}

{\bf \centerline{{\large Lab 1: Descriptive Statistics}}}

\centerline{due date: Wednesday, September 4 in class}

\bigskip

\subsection*{Introduction}

Below are some data on violent crime in US States during the year
1973. This data set comes from {\tt R}, a free statistics software package
widely used by statisticians.

You are asked to compute some summary statistics, create boxplots
and histograms, and answer a few questions.  All of this work should 
be done with {\tt R} or {\tt RStudio}.  (Part of your grade is simply to complete
this lab using the software package {\tt R}.)
Information about the website where you can download {\tt R} and some
basic {\tt R} commands have been included at the end of this handout to help.  
After starting {\tt RStudio} and loading the data set {\tt USArrests}, type
{\tt help(USArrests)} to get some information about this dataset.

For submission to your instructor, write your answers in the space provided 
and hand in graphs.

\subsection*{Data}

\begin{verbatim}
Violent Crime Rates by US State

Description:

     This data set contains statistics, in arrests per 100,000
     residents for assault, murder, and rape in each of the 50 US
     states in 1973. Also given is the percent of the population living
     in urban areas.
\end{verbatim}

\subsection*{Questions}

\begin{enumerate}

\item What is the population under study?

\ifthenelse{\boolean{long}}{\vskip 2cm }{}

\item Compute the Min, Q1, Median, Q3, Max for the numerical variables
Murder and Rape.  Then compute the mean for these two variables.  Is
the mean bigger than, smaller than, or roughly equal to the median?
What does your answer here tell you about the variables Murder and
Rape.

\ifthenelse{\boolean{long}}{\vskip 4cm }{}

\item Compute boxplots for the variables Murder and Rape.  Summarize what
you learn from viewing the boxplots for the two variables.  Plots of these boxplots
should be handed in with your lab.

\ifthenelse{\boolean{long}}{\vskip 4cm }{}

\item Looking at the data for Alaska, how does it compare to other states?
Do you think Alaska is typical?  Can you think of any factors that
might explain the data for Alaska in comparison with the other
states?

\ifthenelse{\boolean{long}}{\vskip 3cm }{}

\item Make a frequency histogram for the numerical variable Rape.  Let each
bin be of size $5$, and have the range on the $x$-axis be from 0 to
50.  (This means your bins will be $[0,5], [5,10]$, etc.)  How many
states have between 15 and 20 rapes per 100,000 residents per year?
Write the {\tt R} command you used to plot this histogram in the space
below.

\ifthenelse{\boolean{long}}{\vskip 1cm }{}
      
\item Looking at the histogram for the variable Rape, can you tell if
the mean is bigger than the median?  Explain.

\ifthenelse{\boolean{long}}{\vskip 3cm }{}

\item Now load the data table in the file {\tt annual\_income} into
{\tt R} and compute the mean and median, and plot a histogram. 
Choose settings for the histogram to display the data in a `good'
light, and include a graph of this histogram with your completed
lab. 

Remove the largest test score and save it to a new variable called d.  Type

\hskip 1cm {\tt d = incomes[1:99]}

at the {\tt R} prompt to do this.  Explain the effect of removing the
largest test score on the mean and the median.

\end{enumerate}
 
 \newpage

{\tt R} and {\tt RStudio} are available over the internet.  The URL for the Comprehensive {\tt R}
Archive Network is

\centerline{\tt http://www.r-project.org/}

\noindent but a simple Google search will find you plenty of hits.
You will need to choose a mirror and download a
package appropriate for your operating system.  If you are not
familiar with downloading such a software package, see me during
office hours for help.

\

\noindent All commands should be typed in {\tt R}'s console window.

\smallskip

Helpful commands from {\tt R}:

\begin{verbatim}

      > help(?????)             get help on command ?????

      > data(USArrests)         loads data set USArrests
      > USArrests               displays dataset
      > attach(USArrests)       makes variables Murder, Assault,
                                UrbanPop, Rape available by name
      > help(USArrests)         get info on dataset
      > Murder                  display variable Murder
      > summary(??)             display summary statistics for variable ??
      > fivenum(??)             display Tukey's five number summary for variable ??
      > boxplot(Murder,Rape)    creates boxplots for variables Murder, Rape
      > boxplot(USArrest)       creates boxplots for all variables in dataset
      > hist(Rape,seq(0,50,5))  creates a frequency histogram for Rape, with bins of
                                size 5 and a range from 0 to 50
                                
                                
       >read(file="annual_income")      loads data from file
       >ls()                            displays variable names
       >incomes                         displays the variable `incomes'
       >getwd()                         get the working directory
       >setwd('??')                     set the working directory to ??
\end{verbatim}

\end{document}
