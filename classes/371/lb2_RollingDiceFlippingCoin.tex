\documentclass{report}

\usepackage[margin=1in]{geometry}

\usepackage{xifthen}
\newboolean{long}
\setboolean{long}{true}

\usepackage{verbatim,color}

\begin{document}

{\bf \centerline{{\large Lab 2: Throwing dice}}}

\centerline{due date: Friday, Oct 4 in class}

\bigskip

\subsection*{Introduction}

This week you will complete a short R lab that involves rolling dice and computing sums,
and tossing coins.  Again and again and again.  This is why we use R.  Each question
is worth 5 points.

You will be graded both on correctness of your answers and on the presentation of
your answer.  This means you need to \emph{explain} your methodology well.

A list of useful R commands is included as the last page of this handout.  You will
need to modify the commands to your purposes.  A copy of this file is available
on the class webpage, if you want to use something like WORD or \LaTeX 
~for your write up, 
or to cut and paste commands.

\subsection*{Problems}

\begin{enumerate}

\item In this problem, you will explore the sum of the roll of three dice.
For sample sizes, try $n = 10, 25, 50, 100$ or other values $n$ of your choice.
Suppose you roll the three dice and sum the outcomes repeatedly.  If you
were to earn a \$1 for the sum, for instance, then a roll of 1,1,1 earns you
\$3. 

How much do you expect to earn in an average game?
Carefully, explain your answer and methodology below.


%\vskip 8cm
%
%\item (Quick introduction to maximum likelihood.)  Suppose you flip a possibly
%unfair coin 10 times with the outcomes:  3 H and 7 T.   Let $p = p(H)$ denote
%the unknown parameter value for a binomial experiment with $n = 10$ and $p$
%unknown.  The \emph{data} for this experiment is denoted by $(H = 3, T = 7)$.
%
%\begin{enumerate}
%
%\item Write down explicitly the Likelihood function $L(p \mid \text{ data })$.
%$$
%
%$$
%
%
%\item 
%
%\end{enumerate}

\newpage

\item  You toss a \emph{biased} coin repeatedly, with the probability of
heads, $P (H) = p$, for some unknown value of $p$.  Your goal is to
give the best estimate of $p$ that you can.  To that end, suppose you
flip this coin \textcolor{red}{three} times and that you earn \$3 for each H that comes up, 
and lose \$1 for each T that comes up.

There are 3 data sets available for you.  They are called
\emph{earningsN} where $N = 10, 100, 1000$.
Again, your task is to give the best estimate for $p$ you can give from
these datasets and explain how you arrived at your answer.

\smallskip

I estimate the value of $p$ to be \underline{\hskip 2cm} ~ because .....


\end{enumerate}


\newpage


\section*{Helpful commands from {\tt R}:}

\bigskip

\begin{verbatim}

# generate 10 random rolls of a di with values between 5 and 9
        d10=floor(runif(10,min=5,max=10))               
                                   
# more or less the contents of my R file named dice.R                                                                               
        sampleSize<-10
        numberDice=7

        sumOfDice=rep(0,sampleSize)

        str=sprintf("Generating %d samples of the experiment   `Find the sum of %d dice'  ",
             sampleSize,numberDice)
        print(str)

        for (ii in 1:sampleSize){
           sumOfDice[ii]=sum(floor(runif(numberDice,5,11)))
           #print(sumOfDice[ii])
        }
        if (sampleSize<1000){
          print(sumOfDice)
        }
# end of batch file dice.R

# how to execute the commands in dice.R
        source("dice.R")
        
# reminder for making histograms
        hist(d10,seq(.5,6.5,1))        
        hist(d10,seq(.5,6.5,1),freq=T)
        hist(d10,seq(.5,6.5,1),freq=F)


# how to load the datasets with earnings, or load and save to a variable
        load(file="earnings10")
        ....

# count number of entries in a vector matching some value c
# In this example, count the number of times vec has entries = 0
       vec = c(0, 0, 1, 1, 1, 2, 0, 2, 0, 0, 5)
       sum(vec==0)

\end{verbatim}

\end{document}
