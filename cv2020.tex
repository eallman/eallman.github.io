% This is a latex file
% last modified Sept 2020

\documentclass[10pt]{report}
\usepackage{amsmath, amssymb, verbatim}
\usepackage{pdfsync}
\usepackage{xifthen}
\newboolean{uafCV}
\setboolean{uafCV}{true}
%\setboolean{uafCV}{false}

\newboolean{tcu}
\setboolean{tcu}{false}
\usepackage{revnum}

\usepackage[usenames]{color}
\newcommand{\tcm}{\textcolor{magenta}}

\newenvironment{pub_list}{
\begin{enumerate}
  \setlength{\itemsep}{1pt}
  \setlength{\parskip}{0pt}
  \setlength{\parsep}{0pt}
}{\end{pub_list}}

\setlength{\oddsidemargin}{0pt}
\setlength{\evensidemargin}{0pt}
\setlength{\textwidth}{6.5in}
\setlength{\topmargin}{0in}
\setlength{\textheight}{9in}

\font\namefont=cmr10 scaled\magstep2
\voffset=-.6in
\hoffset=-.15in
\parskip=.075in
\parindent=0in

\newenvironment{packed_enum}{
\begin{enumerate}
  \setlength{\itemsep}{1pt}
  \setlength{\parskip}{0pt}
  \setlength{\parsep}{0pt}
}{\end{enumerate}}

\thispagestyle{empty}

\begin{document}

\enlargethispage{1cm}

\bigskip
\centerline{\namefont ELIZABETH S. ALLMAN}

\vskip 8mm
%\bigskip

\hbox to \hsize {
\vbox{\hbox{ \bf Current Address:       }
      \hbox{ Department of Mathematics and Statistics}
      \hbox{ University of Alaska Fairbanks}
      \hbox{ PO Box 756660}
      \hbox{ Fairbanks, AK \ \ 99775}
      }
\hfill

\begin{comment}
\vbox{\hbox{ \bf Contact:               }
      \hbox{ Natural Sciences and Mathematics}
      \hbox{ Institute for Arctic Biology}
      \hbox{ https://eallman.github.io}
      \hbox{ (907) 474-2479}
      }
}
\end{comment}

%\begin{comment}
\vbox{\hbox{ \bf Contact:               }
      \hbox{ 1640 Golden View Drive}
      \hbox{ Fairbanks, AK \ 99709 \ USA}
      \hbox{ https://eallman.github.io}
      \hbox{ (907) 474-2479}
      }
}
%\end{comment}

\vspace{.1 in} \hrule \makebox[3.5in][l]{{\bf Internet:} e.allman@alaska.edu}
\makebox[2.5in][r]{\bf Citizenship: } USA
%\vspace{.1 in} \hrule \makebox[2.25in][l]{{\bf Internet:} e.allman@alaska.edu}
%\makebox[2.25in][c]{{\bf Citizenship: } USA}
%\makebox[2in][r]{{\bf Date of birth: } February, 28, 1965}

\vspace{2mm}

%\vspace{2mm}

{\bf EDUCATION}

{ \leftskip=.6in \parindent=-.3in  \parskip=.05in

{\bf University of California at Los Angeles}, (1990 -- 1995) \\
M.A.\ Mathematics (1992).  \\
Ph.D.\ Mathematics (1995).
``Polynomials Without Roots in Division Algebras'' \\
Graduate Opportunity Fellow, (1990 -- 1991).

{\bf Yale University}, (1983 --  1987) \\
B.A. Mathematics, {\it magna cum laude}, Distinction in Mathematics,
DeForest Senior Prize in Pure and Applied Mathematics, (1987).  IBM
Thomas J. Watson Scholar (1983-1987)

\mbox{}
}

{\bf EMPLOYMENT}

{\leftskip=.6in  \parindent=-.3in  \parskip=.05in

{\bf University of Alaska}, Professor (2005 --
\ifthenelse{\boolean{uafCV}}{tenure effective 2007}{}
)\\
Department of Mathematics and Statistics.  \\
Senior Research Associate, Institute for Arctic Biology (2006 --)

{\bf University of Southern Maine}, Associate Professor (2000 -- 2005) \\
\hspace{.9in} Department of Mathematics and Statistics.  Department of
Teacher Education.

{\bf Mathematical Sciences Research Institute}, Member (fall 1999) \\
\hspace{.9in} Special programs in Noncommutative Algebra and Galois
Theory and Fundamental Groups.

{\bf University of North Carolina, Asheville}, Assistant Professor of
Mathematics (1997 -- 1999)

{\bf Bates College}, Assistant Professor of Mathematics (1995 --
1997)

{\bf University of California Los Angeles}, Instructor, Teaching
Assistant Consultant, Teaching Assistant/Associate, Mathematics
Department (1991-1994)

{\bf Infoservice}, Consultant, Mestre, Italy (1989 -- 1990) %\\
%\hspace{.9in} Computer start-up company.

{\bf The American School in Switzerland}, Mathematics Teacher, Lugano, Switzerland
(1988 -- 1989)

{\bf St. Albans School}, Mathematics Teacher, Washington, D.C.
(1987 -- 1988)

\mbox{}
}

{\bf PROFESSIONAL INTERESTS}

%\medskip

{ \leftskip=.6in \parindent=-.3in  \parskip=.05in

  Phylogenetics including tree construction methods and statistical
  models; Modeling of evolutionary processes. Mathematical Biology,
  Statistical Models with Latent Variables, 
  Algebraic Statistics, Computational Algebra, Galois theory and
  Brauer groups, Mathematics Education.

\mbox{}
}

{\bf HONORS AND MEMBERSHIPS}

{ \leftskip=.6in \parindent=-.3in  \parskip=.05in

Fellow, American Mathematical Society.  (2013 --) %

Member: American Mathematical Society, Mathematical Association of
America, Society for Industrial and Applied Mathematics, 
Society for Systematic Biologists, Society for Molecular
Biology and Evolution, Association for Women in Mathematics, Project
NExT.

Teaching Awards: CNSM Outstanding Teacher Award (2009--2010).
Distinguished Teaching Assistant Award, UCLA, (1995).
%Nominee, Campus Teaching Award, UNCA and Bates College.

Erskine Fellow, University of Canterbury, Christchurch, New Zealand, 2013.

%\mbox{}

}

{\bf PUBLICATIONS}

\medskip
 
{ \leftskip=.6in \parindent=-.3in %\parskip=.05in
\parskip=3pt \parsep=0pt \itemsep=0pt

\ifthenelse{\boolean{uafCV}}
{
\begin{revnumerate}[44]
%\begin{revnumerate}[43]
 \setlength{\itemsep}{2pt}
 \setlength{\parskip}{0pt}
 \setlength{\parsep}{0pt}
 
 \item
 ``Parameter identifiability for a profile mixture model of protein evolution."
 with S.~Yourdkhani and J.~Rhodes, J. Computational Biology, \emph{to appear}.
 
 \item 
``Gene tree discord, simplex plots, and statistical tests under the coalescent."
with J.D.~Mitchell and J.~Rhodes,  \emph{Syst. Biol.}, 2021, 
{\tt https://doi.org/10.1093/sysbio/syab008}

 \item ``Testing Multispecies Coalescent Simulators using Summary Statistics,"
 with H.~Ba\~nos-Cervantes and J.~Rhodes,
 \emph{submitted}.
 
  \item 
 ``MSCquartets 1.0: Quartet methods for species trees and networks under the multispecies coalescent model in R." 
 with H.D. Banos-Cervantes, J.D. Mitchell and J.Rhodes, Bioinformatics, 10 (1367-4803), 2020. 
 
 \item ``NANUQ: A method for inferring species networks
from gene trees under the coalescent model,"
 with H.~Ba\~nos-Cervantes and J.~Rhodes,
 Algorithms for Molecular Biology.
 {\bf 14} no.~24 (2019).
% {\tt https://doi.org/10.1186/s13015-019-0159-2}
 
 \item 
 ``Hypothesis testing near singularities and boundaries,"
 with J.~Mitchell and J.~Rhodes,
Electronic Journal of Statistics,
{\bf13} no.~1 (2019) 1250-1293.
 
 \item 
 ``Species tree inference from genomic sequences using the log-det distance,"
 with C.~Long and J.~A.~Rhodes,
 SIAM J.~of Applied Algebra and Geometry,
 {\bf 3} no.~1 (2019) 107-127.
 
 \item 
 ``Maximum likelihood estimation of the Latent Class Model through model boundary decomposition,"
 with H.~Ba\~{n}os-Cervantes, R.~Evans, S.~Ho\c{s}ten, K.~Kubjas, D.~Lemke and J.~Rhodes, and P. Zwiernik,
 Journal of Algebraic Statistics,
{\bf 10} no.~1 (2019) 3-18. 
 
 \item 
``Species tree inference from gene splits by Unrooted STAR methods,''
with J.~Degnan and J.~Rhodes,
IEEE/ACM Transactions in Computational Biology and Bioinformatics, 
{\bf 15} no.~1 (2018) 337-342. 
%{\tt doi:10.1109/TCBB.2016.2604812.}

 \item ``Split probabilities and species tree inference under the
multispecies coalescent model"
 with J.H.~Degnan and J.A.~Rhodes,
 Bulletin of Mathematical Biology,
 {\bf 80}, no.~1 (2018) 64-103.
 % {\tt \ doi:10.1007/s11538-017-0363-5}.		

\item ``Split scores: a tool to quantify phylogenetic signal in genome-scale data,"
with L.S.~Kubatko and J.A.~Rhodes,
Systematic Biology,
{\bf 66} no.~4 (2017) 620--636.
%\newblock {\tt \ doi:10.1093/sysbio/syw103}.

\item 
``Statistically-Consistent $k$-mer Methods for Phylogenetic Tree Reconstruction,''
with J.~Rhodes and S.~Sullivant,
J. Computational Biology, 
{\bf 24} no.~2 (2017) 153-171.

\item 
``Parameter identifiability of discrete Bayesian networks with hidden variables,''
with J.~Rhodes, E.~Stanghellini, and M.~Valtorta,
J.~Causal Inference, {\bf 3} no.~2 (2015) 189-205.

\item 
``Tensors of nonnegative rank two,''
with J.~Rhodes, B.~Sturmfels, and P.~Zwiernik,
Linear Algebra Appl. {\bf 473} (2015) 37-53.

\item ``A Letter from the Guest Editors," special issue in 
Mathematical Biology, 
with F.~Adler and L.~de Pillis,
American Mathematical Monthly {\bf 121} no.~9 (2014) 751-753.

\item
``A semialgebraic description of the general Markov model on phylogenetic trees,"
with J.~Rhodes and A.~Taylor, 
SIAM J. Discrete Math. {\bf 28} no.~2 (2014) 736-755.

\item 
``Tensor Rank, Invariants, Inequalities, and Applications,"
with P.~Jarvis, J.~Rhodes, and J.~Sumner, 
SIAM. J. Matrix Anal.~\& Appl. {\bf 34}  no.~3 (2013) pp. 1014-1045.

\item 
``Identifiability of binary directed graphical models with hidden variables,''
with J.~Rhodes, E.~Stanghellini, and M.~Valtorta,
\emph{Approaches to Causal Structure Learning
Workshop, UAI 2013}, technical report.
 
\item
``Species tree inference by the \emph{STAR} method, and generalizations,"
with J.~Degnan and J.~Rhodes, J. Computational Biology, {\bf 20} no.~1 (2013) 50-61.

\item
``When do phylogenetic mixture models mimic other phylogenetic models?," 
with J.~Rhodes and S.~Sullivant,
Systematic Biology, {\bf 61} no.~6 (2012) 1049-1059.

\item 
``Determining species tree topologies from clade probabilities under the coalescent,"
with J.H.~Degnan, and J.A.~Rhodes,
J.~of Theoretical Biology, {\bf 289} (2011) 96-106.

\item 
``Parameter Identifiability in a class of random graph mixture models"
with C.~Matias and J.A.~Rhodes,
Journal of Statistical Planning and Inference, {\bf 141} no.~5 (2011) 1719-1736.

\item
``Identifying the rooted species tree from the distribution of unrooted gene trees under the coalescent,"
with J.H.~Degnan, and J.A.~Rhodes,
J.~of Mathematical Biology, {\bf 62} no.~6 (2011) 833--862.

\item
``Identifiability of 2-tree mixtures for group-based models,"
with  S.~Petrovic, J.~A.~Rhodes, and S.~Sullivant,
IEEE/ACM Transactions in Computational Biology and Bioinformatics, 
{\bf 8} no.~3 (2011) 710-722.

\item 
``The Phylogenetic Connection," section
in \emph{Advanced topics in linear algebra: {W}eaving matrix problems through the Weyr form} 
by John Clark, Kevin O'meara, and Charles Vinsonhaler,
Oxford University Press, (2011).

\item 
``Trees, Fast and Accurate,"
with John A.~Rhodes,
Science, {\bf 327} (2010) 1334-1335.

\item 
``Estimating trees from filtered data: Identifiability of models
for morphological phylogenetics,"
with M.~Holder and J.~A.~Rhodes,
J.~of Theoretical Biology, {\bf 263} (2010) 108-119.

\item
``Identifiability of parameters in latent structure models with many observed variables,"
with C.~Matias and J.~A.~Rhodes,  
Annals of Statistics, {\bf 37} no.~6A (2009) 3099-3132.

\item
``The identifiability of covarion models in phylogenetics,''
with J.~A.~Rhodes, IEEE/ACM Transactions in Computational Biology and
Bioinformatics, {\bf 6} no.~1 (2009) 76-88.

\item
``Identifiability of a Markovian model of molecular evolution
with Gamma-distributed rates,''
with C.~An\'e and J.~A.~Rhodes, Advances in Applied Probability,
{\bf 40} no.~1 (2008) 229-249.

\item
``Identifying evolutionary trees and substitution parameters
for the general Markov model with invariable sites,'' with
J.~A.~Rhodes, Mathematical Biosciences, {\bf 211} no.~1 (2008) 18-33.

\item
``Phylogenetic ideals and varieties for the general Markov model,''
with J.~A.~Rhodes, Advances in Applied Mathematics, {\bf 40} no.~2
(2008) 127-148.
(online Jan.~2007)

\item
``Molecular phylogenetics from an algebraic viewpoint,'' with J.~A.~Rhodes,
Statistica Sinica, {\bf 17} no.~4 (2007) 1299-1316.

\item
``Phylogenetic invariants,'' with J.~A.~Rhodes, in
\emph{Reconstructing Evolution:~New Mathematical and Computational Advances}, ed.~by O.~Gascuel and
M.~Steel. Oxford University Press, 2007.

\item
``Phylogenetics,'' with J.~A.~Rhodes, in \emph{Modeling and Simulation of
Biological Networks}, ed.~by R.~Laubenbacher. Proceedings of
Symposia in Applied Mathematics, American Mathematical Society,
2007.

\item
``The identifiability of tree topology for phylogenetic models,
including covarion and mixture models,'' with J.~A.~Rhodes, Journal
of Computational Biology, {\bf 13}  (2006) 1101--1113.

\item
``Phylogenetic invariants for stationary base composition,''
with J.~A.~Rhodes, J. Symbolic Computation, {\bf 41} no.~2
(2006) 138--150.

\item
``Quartets and parameter recovery for the general Markov model of
sequence mutation,'' \ with J.~A.~Rhodes, Applied Mathematics
Research eXpress, {\bf 2004} no.~4 (2004) 107--131.

\item
``Phylogenetic invariants for the general Markov model of sequence
mutation,'' \ with J.~A.~Rhodes, Math. Biosciences, {\bf 186}
no.~2 (2003) 113--144.

\item
``Division algebras with $PSL(2,q)$-Galois maximal subfields,'' with
M.~Schacher, J. Algebra, {\bf 240} no.~2 (2001) 808--821.

\item
``Abelian subgroups of finitely generated Kleinian groups are
separable,'' with E.~Hamilton, Bulletin of the London
Mathematical Society, {\bf 31} no.~2 (1999) 163--172.

\item
``Polynomials Without Roots in Division Algebras,''
Communications in Algebra, {\bf 24} no.~13 (1996) 3891--3919.

\end{revnumerate}
}
{ 
``Parameter identifiability for a profile mixture model of protein evolution."
with S.~Yourdkhani and J.~Rhodes, J. Computational Biology, \emph{to appear}.

``Gene tree discord, simplex plots, and statistical tests under the coalescent."
with J.D.~Mitchell and J.~Rhodes,  \emph{Syst. Biol.}, 2021, 
{\tt https://doi.org/10.1093/sysbio/syab008}

``Testing Multispecies Coalescent Simulators using Summary Statistics,"
with H.~Ba\~nos-Cervantes and J.~Rhodes,
\emph{submitted}.

``MSCquartets 1.0: Quartet methods for species trees and networks under the multispecies coalescent model in R." 
with H.D. Banos-Cervantes, J.D. Mitchell and J.Rhodes,  Bioinformatics, 10 (1367-4803), 2020.  


``NANUQ: A method for inferring species networks
from gene trees under the coalescent model,"
 with H.~Ba\~nos-Cervantes and J.~Rhodes,
 Algorithms for Molecular Biology.
 {\bf 14} no.~24 (2019).
% {\tt https://doi.org/10.1186/s13015-019-0159-2}
 
 ``Hypothesis testing near singularities and boundaries,"
 with J.~Mitchell and J.~Rhodes,
Electronic Journal of Statistics,
{\bf13} no.~1 (2019) 1250-1293.
 
 ``Species tree inference from genomic sequences using the log-det distance,"
 with C.~Long and J.~A.~Rhodes,
 SIAM J.~of Applied Algebra and Geometry,
 {\bf 3} no.~1 (2019) 107-127.

 ``Maximum likelihood estimation of the Latent Class Model through model boundary decomposition,"
 with H.~Ba\~{n}os-Cervantes, R.~Evans, S.~Ho\c{s}ten, K.~Kubjas, D.~Lemke and J.~Rhodes, and P. Zwiernik,
 Journal of Algebraic Statistics
{\bf 10} no.~1 (2019) 3-18. 
 
``Species tree inference from gene splits by Unrooted STAR methods,''
with J.~Degnan and J.~Rhodes,
IEEE/ACM Transactions in Computational Biology and Bioinformatics, 
{\bf 15} no.~1 (2018) 337-342. 
%{\tt doi:10.1109/TCBB.2016.2604812.}
 
`Split probabilities and species tree inference under the
multispecies coalescent model"
 with J.H.~Degnan and J.A.~Rhodes,
 Bulletin of Mathematical Biology,
 {\bf 80}, no.~1 (2018) 64-103.
 % {\tt \ doi:10.1007/s11538-017-0363-5}.		

``Split scores: a tool to quantify phylogenetic signal in genome-scale data,"
with L.S.~Kubatko and J.A.~Rhodes,
Systematic Biology,
{\bf 66} no.~4 (2017) 620--636.
%\newblock {\tt \ doi:10.1093/sysbio/syw103}.

``Statistically-Consistent $k$-mer Methods for Phylogenetic Tree Reconstruction,''
with J.~Rhodes and S.~Sullivant,
J. Computational Biology,
{\bf 24} no.~2 (2017) 153-171.

``Parameter identifiability of discrete Bayesian networks with hidden variables,''
with J.~Rhodes, E.~Stanghellini, and M.~Valtorta,
J.~Causal Inference, {\bf 3} no.~2 (2015) 189-205.

``Tensors of nonnegative rank two,''
with J.~Rhodes, B.~Sturmfels, and P.~Zwiernik,
Linear Algebra Appl. {\bf 473} (2015) 37-53.

 ``A Letter from the Guest Editors," special issue in 
Mathematical Biology, 
with F.~Adler and L.~de Pillis,
American Mathematical Monthly {\bf 121} no.~9 (2014) 751-753.

``A semialgebraic description of the general Markov model on phylogenetic trees,"
with J.~Rhodes and A.~Taylor, 
SIAM J. Discrete Math. {\bf 28} no.~2 (2014) 736-755.

`Tensor Rank, Invariants, Inequalities, and Applications,"
with P.~Jarvis, J.~Rhodes, and J.~Sumner, 
SIAM. J. Matrix Anal.~\& Appl. {\bf 34}  no.~3 (2013) 1014-1045.

``Identifiability of binary directed graphical models with hidden variables,''
with J.~Rhodes, E.~Stanghellini, and M.~Valtorta,
\emph{Approaches to Causal Structure Learning
Workshop, UAI 2013}, technical report.

``Species tree inference by the \emph{STAR} method, and generalizations,"
with J.~Degnan and J.~Rhodes, J. Computational Biology, {\bf 20} no.~1 (2013) 50-61.

``When do phylogenetic mixture models mimic other phylogenetic models?," 
with J.~Rhodes and S.~Sullivant,
Systematic Biology, {\bf 61} no.~6 (2012) 1049-1059.

``Determining species tree topologies from clade probabilities under the coalescent,"
with J.H.~Degnan, and J.A.~Rhodes,
J.~of Theoretical Biology, {\bf 289} (2011) 96-106.

``Parameter Identifiability in a class of random graph mixture models"
with C.~Matias and J.A.~Rhodes,
Journal of Statistical Planning and Inference, {\bf 141} no.~5 (2011) 1719-1736.

``Identifying the rooted species tree from the distribution of unrooted gene trees under the coalescent,"
with J.H.~Degnan, and J.A.~Rhodes,
J.~of Mathematical Biology, {\bf 62} no.~6 (2011) 833--862.

``Identifiability of 2-tree mixtures for group-based models,"
with  S.~Petrovic, J.~A.~Rhodes, and S.~Sullivant,
IEEE/ACM Transactions in Computational Biology and Bioinformatics, 
{\bf 8} no.~3 (2011) 710-722.

``The Phylogenetic Connection," section
in \emph{Advanced topics in linear algebra: {W}eaving matrix problems through the Weyr form} 
by John Clark, Kevin O'meara, and Charles Vinsonhaler,
Oxford University Press, (2011).

``Trees, Fast and Accurate,"
with John A.~Rhodes,
Science, {\bf 327} (2010) 1334-1335.

``Estimating trees from filtered data: Identifiability of models
for morphological phylogenetics,"
with M.~Holder and J.~A.~Rhodes,
J.~of Theoretical Biology, {\bf 263} (2010) 108-119.

``Identifiability of parameters in latent structure models with many observed variables,"
with C.~Matias and J.~A.~Rhodes,  
Annals of Statistics, {\bf 37} no.~6A (2009) 3099-3132.

``The identifiability of covarion models in phylogenetics,''
with J.~A.~Rhodes, IEEE/ACM Transactions in Computational Biology and
Bioinformatics, {\bf 6} no.~1 (2009) 76-88.

``Identifiability of a Markovian model of molecular evolution
with Gamma-distributed rates,''
with C.~An\'e and J.~A.~Rhodes, Advances in Applied Probability,
{\bf 40} no.~1 (2008) 229-249.

``Identifying evolutionary trees and substitution parameters
for the general Markov model with invariable sites,'' with
J.~A.~Rhodes, Mathematical Biosciences, {\bf 211} no.~1 (2008) 18-33.

``Phylogenetic ideals and varieties for the general Markov model,''
with J.~A.~Rhodes, Advances in Applied Mathematics, {\bf 40} no.~2
(2008) 127-148.
(online Jan.~2007)

``Molecular phylogenetics from an algebraic viewpoint,'' with J.~A.~Rhodes,
Statistica Sinica, {\bf 17} no.~4 (2007) 1299-1316.

``Phylogenetic invariants,'' with J.~A.~Rhodes, in
\emph{Reconstructing Evolution:~New Mathematical and Computational Advances}, ed.~by O.~Gascuel and
M.~Steel. Oxford University Press, 2007.

``Phylogenetics,'' with J.~A.~Rhodes, in \emph{Modeling and Simulation of
Biological Networks}, ed.~by R.~Laubenbacher. Proceedings of
Symposia in Applied Mathematics, American Mathematical Society,
2007.

``The identifiability of tree topology for phylogenetic models,
including covarion and mixture models,'' with J.~A.~Rhodes, Journal
of Computational Biology, {\bf 13}  (2006) 1101--1113.

``Phylogenetic invariants for stationary base composition,''
with J.~A.~Rhodes, J. Symbolic Computation, {\bf 41} no.~2
(2006) 138--150.

``Quartets and parameter recovery for the general Markov model of
sequence mutation,'' \ with J.~A.~Rhodes, Applied Mathematics
Research eXpress, {\bf 2004} no.~4 (2004) 107--131.

``Phylogenetic invariants for the general Markov model of sequence
mutation,'' \ with J.~A.~Rhodes, Math. Biosciences, {\bf 186}
no.~2 (2003) 113--144.

``Division algebras with $PSL(2,q)$-Galois maximal subfields,'' with
M.~Schacher, J. Algebra, {\bf 240} no.~2 (2001) 808--821.

``Abelian subgroups of finitely generated Kleinian groups are
separable,'' with E.~Hamilton, Bulletin of the London
Mathematical Society, {\bf 31} no.~2 (1999) 163--172.

``Polynomials Without Roots in Division Algebras,''
Communications in Algebra, {\bf 24} no.~13 (1996) 3891--3919.
}

%\medskip

{\sc In preparation:}

%\smallskip

\begin{itemize}

 \setlength{\itemsep}{2pt}
 \setlength{\parskip}{0pt}
 \setlength{\parsep}{0pt}

\item ``New methods of inferring network distances,"
 with H.~Ba\~nos and J.~Rhodes, 
 \emph{in preparation}.
 
 \item ``Density contours in BHV space under the coalescent model,"
 with J.H.~Degnan, M.~Owen, and Claudia Solis-Lemus, 
 \emph{in preparation}.
 
\end{itemize}

%\medskip

%\newpage

{\sc Textbooks:}

%\smallskip

\ifthenelse{\boolean{uafCV}}
{
\begin{revnumerate}[2]
%\begin{revnumerate}[10]
 \setlength{\itemsep}{2pt}
 \setlength{\parskip}{0pt}
 \setlength{\parsep}{0pt}

\item {\it Mathematical Models in Biology: An Introduction},
with J.~Rhodes, Cambridge University Press (2004) 384 pages.

\item 
{\it Solution Manual to Mathematical Models in Biology: An
Introduction}, \ Allman, E.  and Rhodes, J., Cambridge University
Press (2004) 81 pages.  (electronic)

\end{revnumerate}
}
{
{\it Mathematical Models in Biology: An Introduction},
with J.~Rhodes, Cambridge University Press (2004) 384 pages.

{\it Solution Manual to Mathematical Models in Biology: An
Introduction}, \ Allman, E.  and Rhodes, J., Cambridge University
Press (2004) 81 pages.  (electronic)
}

\medskip

{\sc Unpublished Lecture Notes}

\smallskip

 \setlength{\itemsep}{2pt}
 \setlength{\parskip}{0pt}
 \setlength{\parsep}{0pt}


\ifthenelse{\boolean{uafCV}}
{\begin{itemize}
\item
``The Mathematics of Phylogenetics,'' with J.~A.~Rhodes, Institute for
Advanced Study -- Park City Mathematics Institute: summer program in
Mathematical Biology, (June 2005, revised spring 2009) 127 pages.
\end{itemize}
}
{
``The Mathematics of Phylogenetics,'' with J.~A.~Rhodes, Institute for
Advanced Study -- Park City Mathematics Institute: summer program in
Mathematical Biology, (June 2005) 127 pages.
}

\medskip

{\sc Software packages}

\smallskip

 \begin{itemize}
 
 \setlength{\itemsep}{2pt}
 \setlength{\parskip}{0pt}
 \setlength{\parsep}{0pt}
 
 \item {\tt MSCquartets v.~1.0.5}, joint with H.~Ba\~nos-Cervantes, J.~Mitchell, J.A.~Rhodes, ($R$ package) available
 on CRAN Jan 2020.
 
 \item {\tt MSCsimtester v.~1.0}, joint with H.~Ba\~nos-Cervantes and J.A.~Rhodes.  ($R$ package) July 2019.

 \item {\tt SplitSup v.~1.02}, joint with L.S.~Kubatko and J.A.~Rhodes.  July 2016.

 
 \end{itemize}
 
 \medskip


{\sc Other Publications:}

\smallskip

\ifthenelse{\boolean{uafCV}}
{\begin{revnumerate}[8]

 \setlength{\itemsep}{2pt}
 \setlength{\parskip}{0pt}
 \setlength{\parsep}{0pt}

\item
``On Teaching Geometry to Pre-service Elementary Teachers, II,'' \
Allman, E., in Journeys: Maine Mathematics and Science Teacher
Education Excellence Collaborative monograph (2006) 36--41.

\item
``International Views on Education,'' \ Allman, E., in {\it
Complexities: Women in Mathematics}, edited by B.~Case and
A.~Leggett, Princeton University Press (2005) 128--131.

\item
``On Teaching Geometry to Pre-service Elementary Teachers'' \ Allman,
E., in Preparing Future Science and Mathematics Teachers, edited by
D.~Smith and E.~Swanson, (2005) 26--28.

\item
``Lessons Learned from Four Years in a Joint Position in Mathematics
and Mathematics Education,'' \ Allman, E., National Science
Foundation 2004 Teacher Preparation PI Conference Proceedings (April
2004) 1--9. (electronic)

\item
Review of ``Generalized Non-abelian Reciprocity Laws: A Context for
Wiles' Proof,'' by A. Ash and R. Gross, College Mathematics Journal,
{\bf 33} no.~1 (2002) 67.

\item
Review of ``Mathematics of Infectious Disease,'' by H.  Hethcote,
College Mathematics Journal, {\bf 32} no.~5 (2001) 397--399.

\item
``AWM panel in San Antonio,'' \ Allman, E., Association of Women in
Mathematics Newsletter, 29 no.~3 (1999) 8--9.

\item
``Numbers,'' Association of Women in Mathematics Newsletter, {\bf 28}
no.~3 (1998) 14.
\end{revnumerate}
}
{
``On Teaching Geometry to Pre-service Elementary Teachers, II,'' \
Allman, E., in Journeys: Maine Mathematics and Science Teacher
Education Excellence Collaborative monograph (2006) 36--41.

``International Views on Education,'' \ Allman, E., in {\it
Complexities: Women in Mathematics}, edited by B.~Case and
A.~Leggett, Princeton University Press (2005) 128--131.

``On Teaching Geometry to Pre-service Elementary Teachers'' \ Allman,
E., in Preparing Future Science and Mathematics Teachers, edited by
D.~Smith and E.~Swanson, (2005) 26--28.

``Lessons Learned from Four Years in a Joint Position in Mathematics
and Mathematics Education,'' \ Allman, E., National Science
Foundation 2004 Teacher Preparation PI Conference Proceedings (April
2004) 1--9. (electronic)

Review of ``Generalized Non-abelian Reciprocity Laws: A Context for
Wiles' Proof,'' by A. Ash and R. Gross, College Mathematics Journal,
{\bf 33} no.~1 (2002) 67.

Review of ``Mathematics of Infectious Disease,'' by H.  Hethcote,
College Mathematics Journal, {\bf 32} no.~5 (2001) 397--399.

``AWM panel in San Antonio,'' \ Allman, E., Association of Women in
Mathematics Newsletter, 29 no.~3 (1999) 8--9.

``Numbers,'' Association of Women in Mathematics Newsletter, {\bf 28}
no.~3 (1998) 14.
}

\mbox{}

\ifthenelse{\boolean{tcu}}{\newpage}{}

}

{\bf  ADDRESSES}

\medskip

{ \leftskip=.6in \parindent=-.3in  \parskip=3pt

{\sc Plenary Addresses and Short Courses:}

\smallskip

Nonlinear algebra in applications, ICERM workshop, Providence, RI,
Nov. 2018. ``Reconstructing Hybridization Networks,"

Algebraic and Combinatorial Phylogenetics Workshop, Barcelona, Spain,
June 2017.  ``An Introduction to Algebraic Methods in Phylogenetics."

Mathematical and Computational Evolutionary Biology conference,
Porquerolles Island, 
France, June 2017.  ``Split Scores for Phylogenetic Trees and Applications."

New Mexico Statewide High School Mathematics Contest plenary speaker, University
of New Mexico, Albuquerque, Feb.~2014. 

Algebraic Statistics in the Alleghenies, Pennsylvania State University, State
College, PA, June 2012.
\emph{A semialgebraic description of the general Markov model on trees}.

Workshop on Evolutionary Genomics, Institute for Pure and Applied Mathematics
(IPAM), Nov~2011. 
\emph{Identifying Rooted Species Trees from Unrooted Gene Tree and Clade Probabilities}.

2011 Workshop for Young Researchers in Mathematical Biology (WYRMB),
Mathematical Biosciences Research Institute, Columubus, Ohio, Aug 2011.
\emph{Phylogenetic tree models: An algebraic perspective}.

AMS Western Sectional Meeting, Las Vegas, NV, May 2011.
\emph{Evolutionary trees and phylogenetics: An algebraic perspective}.

A Day of Biology and Mathematics, KTH, Stockholm, Sweden, Feb.~2011,
\emph{Gene trees vs.~species trees}.

Centre of Mathematics for Applications, 
``Algebraic Geometry in the sciences,"
Oslo, Norway, Jan 2011.

Tutorial on phylogenetics at SAMSI's opening workshop for the
year-long research program in
algebraic methods in systems biology and statistics, Sept.~2008.

Workshop ``Applications in Biology, Dynamics, and Statistics,'' part
of thematic year in algebraic geometry at the Institute for
Mathematics and its Applications. Minneapolis, MN, March 2007,
\emph{Phylogenetic Models: Algebra and Evolution}.

American Mathematical Society Short Course ``Modeling and Simulation
of Biological Networks,'' Joint Mathematics Meetings. Jan,~2006,
\emph{Phylogenetics}.

Workshop ``Phylogeography and Phylogenetics,'' Mathematical
Biosciences Institute at the Ohio State University. Columbus, Ohio,
Sept.~2005, \emph{Progress and Potential for Phylogenetic
Invariants}.

XVI Coloquio Latinoamericano de \'Algebra.  Colonia, Uruguay,
Aug.~2005, \emph{Progress in Phylogenetics using Algebraic
Varieties}.

Institute for Advanced Study - Park City Mathematics Institute ``The
Mathematics of Phylogenetic Trees,'' co-instructor undergraduate
course. June - July 2005, Park City, Utah.

Mathematics of Evolution and Phylogeny - Institut Henri
Poincar\'e. Paris, France, June 2005, \emph{Phylogenetic
Invariants: Recent Progress and New Directions}.

MEGA 2005 - The Eighth International Symposium on Effective
Methods in Algebraic Geometry
: ``Computing in and with algebraic
geometry: Theory, Algorithms, Implementations, Applications.''
Porto Conte, Alghero, Sardinia, May 2005, \emph{Phylogenetics and
Algebraic Geometry: Problems from Biology}.

American Institute of Mathematics, Research conference on
Computational Algebraic Statistics, Palo Alto, CA, Dec. 2003, {\it
Progress towards understanding the phylogenetic ideal}.

\mbox{}

{\sc Selected Research Addresses:}

\smallskip

SIAM  Conference on Applied Algebraic Geometry, Bern, Switzerland,
July 2019.   \emph{Inferring Species Networks from Gene Trees}.

Evolution Meetings, Providence, Rhode Island, June 2019.
  \emph{Inferring Species Networks from Gene Trees}.

Center for Integrative Bioinformatics Vienna, June 2017.
\emph{Split scores for phylogenetic trees and applications}.

Oberwolfach Mathematics Research Institute workshop on
``Algebraic Statistics," April 2017.
\emph{Identifying species trees from split probabilities}.

Banff International Research Station Workshop on 
``Mathematical Approaches to Evolutionary Trees and Networks,"
Feb 2017. 
\emph{Identifying species trees from summary statistics}.

SIAM Conference on the Life Sciences, July 2016.
\emph{Splits scores on phylogenetic trees and applications}.

Evolution Meetings, Austin, TX, June 2016.
\emph{Splits scores on phylogenetic trees and applications}.

Joint Statistics Meeting, Seattle, WA, Aug 2015.
\emph{Splits scores on phylogenetic trees and applications}.

Texas Christian University, Mathematics Department Colloquium,
Ft.~Worth, TX, Feb 2015.
\emph{Algebraic methods in phylogenetics}.

University of New Mexico, Mathematics and Statistics Department Colloquium,
Albuquerque, New Mexico, Jan 2014.  
\emph{An algebraic perspective on evolutionary trees and phylogenetics}.

University of Canterbury, Christchurch, New Zealand, Aug 2013.
\emph{Ranks of matrices and tensors}.

UC Berkeley Computational Biology seminar, March 2013.
\emph{Inferring species trees from gene trees: Combinatorial approaches}.

AMS-MAA Joint Meetings, Boston, MA, Jan 2012. \emph{Species trees from gene trees}.

\ifthenelse{\boolean{uafCV}}{University of Tasmania colloquium, Dec~2010.
\emph{Primary decomposition and phylogenetic varieties.}}{}

Phylomania, Hobart, Tasmania, Nov 2010. \emph{On $2$-tree mixtures}.

Evolution Meetings, Portland, OR, June 2010.
 \emph{Identifying the rooted species tree from unrooted gene tree probabilities
under the coalescent model}.

University of Canterbury Mathematics and Statistics Department Colloquium, Christchurch, 
New Zealand, July 2009.

Phylogenetics 2009, Renyi Institute, Budapest, Hungary, June 2009.
 \emph{Two-tree mixtures of group-based models}.

SAMSI Algebraic Statistical Models, Raleigh, NC, Jan 2009. 
\emph{Applications of Kruskal's theorem to the identifiability of algebraic statistical models}.

Annual Meeting of the Society for Molecular Biology and Evolution,
Barcelona, Spain, June 2008, poster \emph{Identifiability of covarion models
in phylogenetics}.

University of Barcelona, Barcelona, Spain, June 2008, Mathematics
Colloquium speaker.

Joint Mathematics Meetings, AMS Special Session on Secant Varieties
and related topics, San Diego, CA,  Jan.~2008. \emph{Secant Varieties and
Statistical Models.}

University of Sheffield, Sheffield, England,  Oct.~2007. \emph{Models
of DNA site substitution}.

\ifthenelse{\boolean{uafCV}}{
Isaac Newton Institute for the Mathematical Sciences,
Cambridge, England,  Oct.~2007. \emph{On the identifiability
of substitution models}.
}
{}

Isaac Newton Institute for the Mathematical Sciences workshop,
``Current Challenges and Problems in Phylogenetics,'' Cambridge,
England, Sept.~2007. \emph{The identifiability of the GTR+$\Gamma$
model of molecular evolution}.

Joint Statistics Meetings: Annual meetings of the American
Statistical Association, the International Biometric Society (ENAR
and WNAR), the Institute of Mathematical Statistics, and the
Statistical Society of Canada, Aug.~2006.  Invited speaker in special session
on phylogenetics, organized by B. Larget.

Satellite conference to XVI Coloquio at University of Buenos Aires,
Buenos Aires, Argentina, Aug.~2005. {\it Constructing phylogenetic
invariants}.

Virginia Polytechnic Institute and State University, Blacksburg,
VA, April 2005.  \emph{Markov models of DNA mutation on
phylogenetic trees}.

Dartmouth College, Hanover, NH, Feb. 2005.  \emph{Modeling
Molecular Evolution: Probabilistic Models of Base Substitution}.

Maine Medical Center Research Institute, the Jackson Laboratories,
University of Maine Orono,
%(video lecture for Computational Methods in
%Genomics, course for Ph.D. program in Computational Genomics at Orono),
Nov. 2004. {\it Probabilistic Models of Base Substitution: the
Jukes-Cantor distance}.

Evolution 2004,
Annual Meeting of Society of Systematic Biologists,
%Annual Meeting of
Society for the Study of Evolution, Society of Systematic Biologists
and the American Society of Naturalists.

Colorado State University, Fort Collins, CO, June 2004.
 {\it Phylogenetic Invariants
  and Parameter Recovery for the General Markov-Invariable Sites
  Model}.

Valley Geometry Seminar, Amherst, MA, April 2004. {\it The Algebra of
Molecular Evolution and Phylogenetic Tree Construction}.

St.~Olaf College, Northfield, MN, Feb. 2004. {\it The Algebra of
Molecular Evolution}.

James Madison University, Harrisonburg, VA, Feb. 2004. {\it Phylogenetic
Tree Construction: An Overview}.

The Annual New Zealand Phylogenetics Meeting, Whakapapa Village, New Zealand,
Feb.~2004. {\it Constructing Phylogenetic Invariants for the General
Markov Model of Sequence Mutation}.

International Conference on Statistics, Combinatorics and Related
Areas, Session on bioinformatics, University of Southern Maine,
Oct.~2003.

University of California, Berkeley, Mathematics of Phylogenetic Trees,
seminar, Sept.~2003. {\it Algebraic techniques for phylogeny
reconstruction from DNA sequence data}.

\newpage
University of Maine, Orono, Dec.~2002. {\it Constructing Phylogenetic
  Trees using DNA sequences and the Method of Phylogenetic
  Invariants}.

International Congress of Mathematicians: Beijing, China, poster,
Aug. 2002. {\it Phylogenetic Invariants for General Models of Sequence
Mutation}.

Brauer Groups Conference, Fort Collins, CO, July 2002.

Joint Mathematics Meeting, AMS - MAA: San Diego, CA, Jan. 2002.
Special Session ``Algebras, Forms, and Algebraic Groups,''

MSRI Postdoctoral Seminar: MSRI, Berkeley, CA, Dec. 1999,

Olga Taussky Todd Conference: MSRI, Berkeley, CA, July 1999,
invited postdoctoral participant.

AWM Workshop at the Joint Mathematics Meeting, AMS - MAA: San Antonio,
TX, Jan.~1999.
\emph{Subgroup Separability: A Blending of Number Theory, Geometry and
  Topology}.

Center for Communications Research, La Jolla, CA, \ June 1996.
{\it Deficient Polynomials over Number Fields}.

Conference on Brauer Groups and Galois Groups: Technion University,
Haifa, Israel, \ June/July 1996.
\emph{Polynomials Without Roots in Division Algebras}.

Joint Meeting AMS - Israel Mathematical Union: Hebrew University,
Jerusalem, Israel, May 1995. \emph{Polynomials Without Roots in Division
Rings}.

Conference on Division Algebras: Bar-Ilan University, Bar-Ilan, Israel,
May 1995.
\emph{Polynomials Without Roots in Division Rings}.

\ifthenelse{\boolean{uafCV}}{
USC Algebra Seminar: Los Angeles, Oct. 1994.
\emph{Deficient Polynomials over Algebraic Number Fields}.}

\mbox{}

%\newpage

\smallskip

{\sc Selected Professional Service Addresses:}

\smallskip

{\it Starting a Career in Mathematics}, AWM panel, Joint Mathematics Meetings,
San Diego, Jan.~2008.

{\it Joining the Mathematical Community}.  Project NExT, San Jose State
University, Mathfest, July 2007.

\ifthenelse{\boolean{uafCV}}{
{\it Phylogenetics: Techniques of Tree Construction}.  Colby College,
Waterville, ME, Sept. 2004.}
{}

{\it A Joint Position in Mathematics and Mathematics Education at USM}.
National Visiting Committee of the National Science Foundation's review
of MMSTEC grant, Freeport, ME, April 2004.

{\it Lessons Learned from Four Years in a Joint Position in Mathematics
and Mathematics Education}. National Science Foundation 2004 Teacher
Preparation PI Conference, Crystal City, VA, March 2004.

\ifthenelse{\boolean{uafCV}}{
{\it Reconstructing Evolution: The Mathematics of DNA sequences}.
St. Olaf College, Northfield, MN, Feb. 2004.}
{}

%\enlargethispage{.5in}

%\newpage

{\it Models of Molecular Evolution: An Introduction}. Computational
Methods in Genomics, Orono and Bar Harbor, ME, Nov. 2003.

{\it Molecular Evolution Models and Phylogeny Reconstruction in an
Introductory Biomathematics Course}. Conference on Bioinformatics in the
Undergraduate Curriculum, Dickinson College, Carlisle, PA, March 2003.

{\it Mathematical Models in Biology: an undergraduate course}. MAA
MathFest, Los Angeles, CA, Aug.~2000.

\begin{comment}
{\it To reform or not to reform: Where do we stand now?}. Invited
panelist, MAA MathFest, Los Angeles, CA, Aug.~2000.

{\it Epidemic Modeling: MATLAB Exercises for an Undergraduate Course in
Mathematical Models in Biology}. Invited speaker, MAA special session on
Discrete Mathematics Revisited, Joint Mathematics Meetings, San Antonio,
TX, Jan. 1999.

{\it An Introduction to Hyperbolic Geometry}.  Colloquium, Western
Carolina University, Sylva, NC, Nov. 1998.

{\it An Introduction to the Mathematics of Special Relativity}.
Colloquium, Trinity College, Hartford, CT, April 1998.

{\it An Introduction to Hyperbolic Geometry}. Colloquium, Occidental
College, Los Angeles, CA, March 1998.

%{\it Maple and Implicit Differentiation: a Laboratory Exercise}. Project
%NExT-SE special session at the spring southeastern sectional meeting of
%the MAA, Charleston, SC, March 1998.

{\it An Introduction to Hyperbolic Geometry}.  Colloquium, Davidson
College, Davidson, NC, Feb. 1998.

{\it Growing an Active Classroom; the Pros and Cons of Different
Teaching Styles}.  Panelist and contributing author, Project NExT
special session,  Joint Mathematics Meetings, AMS-MAA, Jan. 1998.

%{\it Starting a Course from Scratch}. Co-organizer panel discussion for
%Project NExT, Joint Mathematics Meetings, AMS-MAA, Jan. 1998.
\end{comment}

\mbox{}
}

%\newpage

%\enlargethispage{.3in}

{\bf GRANTS}

\medskip

{ \leftskip=.6in \parindent=-.3in  \parskip=3pt

INBRE Sustaining Research Excellence grant, \ifthenelse{\boolean{uafCV}}{(\$75,000)}{}
June 2020-21.  

Banff International Research Station, Banff, CA, June 2020 (postponed due to Covid-19).

American Institute of Mathematics SQuaRE (Structured Quartet Research Ensembles) grant, 
``Mean trees from samples of gene trees under the coalescent model" with J.~Degnan, M.~Owen,
and C.~Sol\'is-Lemus.
2019-2021.

Oberwolfach Mathematics Research Institute, Oberwolfach, Germany, April 2017. 

Banff International Research Station, Banff, CA, Feb 2017.

NIH Award R01 GM117590,
Mathematics and computational analysis for specie tree inference, Joint DMS/NIGMS Initiative to Support 
Research at the Interface of the Biological and Mathematical Sciences, with co-PIs J.~Rhodes, J.~Degnan, and N.~Rosenberg,
\ifthenelse{\boolean{uafCV}}{(\$1,543,628),}{} 2015-2020.

Erskine Fellowship, University of Canterbury, Christchurch, New Zealand.  Erskine Fellow 2013.

\ifthenelse{\boolean{uafCV}}{Genome Consortium for Active Teaching (GCAT) Workshop on Synthetic Biology,
funded by HHMI and NSF, attended curricular development workshop with UAF B\&W/IAB faculty member K.~O'Brien,  June 2012.}
{}

National Institute for Mathematical and Biological Synthesis (NIMBioS) grant.  Funded short-term
visit to NIMBioS to collaborate with visiting scholar there,  June 2012.

%\newpage

American Institute of Mathematics SQuaRE grant, (Structured Quartet Research Ensembles)
`Integrating and extending techniques for identification of latent variables in graphical models'
with J.~Rhodes, E.~Stanghellini (Perugia) and M.~Valtorta (South Carolina).  
\ifthenelse{\boolean{uafCV}}{Grant funds trips to AIM for a week of focused research
by this quartet.}
{}
2011-2014.

NIMBioS working group on Species Delimitation, (group member).
\ifthenelse{\boolean{uafCV}}{Grant funds small research group for collaborative
investigation into delimitation.  Co-PIs are R.~Yoshida and D.~Weisrock of 
the University of Kentucky.}{}
2010-2012.

National Science Foundation Grant \#1047839:  Participant Suppport:
2011 Mittag-Leffler Institute, Division of Mathematical Sciences,
Program in Algebra, Number Theory and Combinatorics, and Program
in Other Global Learning and Training; with co-PI S.~Sullivant,
\ifthenelse{\boolean{uafCV}}{(\$48,515),}{}
2010-2011.

Mittag-Leffler Institute, Jan--March 2011, Research Fellow.

University of Tasmania Visiting Scholar, Hobart, Tasmania, fall 2010.

Statistical and Applied Mathematical Sciences Institute (SAMSI),  Raleigh, NC, 2009.\\
Research Fellow, Special program on ``Algebraic Methods in Systems
Biology and Statistics."

 National Science Foundation Grant \#0714830: Enhancing Phylogenetic
 Methods and Theory via Algebraic Perspectives, Division of
 Mathematical Sciences, Program in Mathematical Biology; with co-PI
 John Rhodes,
 \ifthenelse{\boolean{uafCV}}{(\$486,450),}{}
2007-2012.

Isaac Newton Institute for Mathematical Sciences, Cambridge
University, England; fall 2007.\\
Visiting Fellow.  Special research program in ``Phylogenetics.''

Mathematical Biosciences Institute, Columbus, OH;
two-week residency, April 2007.

Institute for Mathematics and Applications, Minneapolis, MN; Jan-Mar
2007.\\ Long-term visitor in special research program in
``Applications of Algebraic Geometry.''

Clay Mathematics Institute Workshop, ``Algebraic Statistics and
Computational Biology,'' Nov.~2005.

Mathematical Biosciences Institute at the Ohio State University,
``Phylogeography and Phylogenetics,'' workshop, Columbus, Ohio; Sept.
2005,

NSA travel subgrant, for participation in BASCOLA, Buenos Aires,
Argentina; Aug.~2005.

Institute for Advanced Study - Park City Mathematics Institute, Park
City, UT; June-July 2005.\\  Summer program in Mathematical Biology,

NSF-AWM travel grants, 2004, 1999, 1996.

Association of Women in Mathematics -- NSF, AWM Leadership Conference
grant, March 2004.

AIM and NSF workshop, American Institute of Mathematics, Palo Alto, CA; Dec.~2003.\\
Computational Algebraic Statistics.

USM College of Arts and Sciences Research and Creative Activity Fund,
spring 2003.

NSF and AMS Grant, August 2002.\\
Grant to participate in the {International Congress of Mathematicians}
in Beijing, China.

NSF and ExxonMobil Foundation Grant, fall 2001. \\
Grant to participate in the {\it National Summit on the Mathematical
Education of Teachers}.  This summit, sponsored by the MAA, was by
invitation only.

USM Faculty Technology Grant, spring 2002.

USM Faculty Development Research Grants, 2000, 2001, 2003, 2004,
2005.

NSF Grant for Junior US mathematicians, to attend the {\it Mathematical
Challenges of the 21st Century} AMS Meeting at UCLA, August 2000.

Mathematical Sciences Research Institute (MSRI) member, fall 1999.  %\\

NSA-AWM Grant, July 1999.  \\
\hspace{.9in} Olga Taussky Todd Celebration of Careers in Mathematics
for Women, MSRI.

UNCA University Research Grant recipient, spring 1999.

AWM Workshop for Women Graduate Students and Post-doctoral
Mathematicians, San Antonio; Jan.~1999.

American Association of University Women Summer Faculty Fellowship,
summer 1998. \\
\hspace{.9in} {\it Kleinian Groups, Subgroup Separability, and
Mathematical Biology}.

NSF-CBMS Conference in Division Algebras, Colorado State University,
Fort Collins, June 1998.

UNCA University Teaching Grant, fall 1997.

Project NExT Fellow, 1996--1998.  \\
\hspace{.9in} MAA/Exxon Grant for \lq\lq New Experiences in Teaching".

\mbox{}
}

{\bf OTHER PROFESSIONAL ACTIVITIES}

\medskip

{ \leftskip=.6in \parindent=-.3in  \parskip=3pt

{\sc Reviewer:}

\medskip

Associate Editor:  \emph{SIAM Review}, 2021-24.

Editorial board: \emph{American Mathematical Monthly}, 2011-2016, 2017-2022.

Guest Editor, \emph{American Mathematical Monthly} special issue
in Mathematical Biology, in Nov.~2014.

National Science Foundation reviewer:

\hskip .3in Joint NSF/NIGMS Initiative to Support Research in the Area of
Mathematical Biology

\hskip .3in DMS Mathematical Biology program 
\ifthenelse{\boolean{uafCV}}{(multiple years)}{}

\hskip .3in program ``Collaboration in
Mathematical Geosciences: Opportunities for research collaborations
between the Mathematical Sciences and the Geosciences.''

\hskip .3in OISE organization

Referee: {\sl Advances in Applied Mathematics}, {\sl Algorithms for
  Molecular Biology}, {\sl Annals of Statistics}, 
   {\sl Biometrika}, 
  {\sl Bulletin of Mathematical Biology}, 
  {\sl Cambridge University Press}, {\sl College
  Mathematics Journal}, {\sl Institute for Mathematics and its
  Applications Volume Series}, {\sl International Symposium in
  Symbolic and Algebraic Computation} (ISSAC), {\sl Journal of
  Interdisciplinary Mathematics}, {\sl Journal of Online Mathematics
  and its Applications}, {\sl Journal of Symbolic Computation}, {\sl
  MAA Online}, {\sl Journal of Mathematical Biology}, 
  {\sl Journal of Theoretical Biology},  {\sl Mathematical Biosciences}, {\sl Mathematics
  Magazine}, {\sl Molecular Biology and Evolution}, {\sl Molecular Phylogenetics
  and Evolution}, {\sl Oxford   University Press}, {\sl PeerJ}, {\sl Systematic Biology}, {\sl IEEE/ACM Transactions on
  Computational Biology and Bioinformatics}

Ph.D.~examiner:

\hskip .3in University of Canterbury Department of Mathematics and
Statistics, New Zealand.

\hskip .3in McGill University Department of Mathematics and
Statistics, Canada.

Promotion and/or Tenure outside reviewer:  University of Tasmania, North Carolina State University,
Colorado College, Clemson University, San Francisco State University,
American University of Sharjah, United Arab Emirates, University of Hawai'i

New Zealand's Marsden Fund Reviewer

AWM graduate student and recent Ph.D.~workshop co-organizer.  Review
graduate student applications for the joint meetings AWM workshop
(2008, 2009, 2011).

AWM travel grant program reviewer.  Helped review the
effectiveness of the AWM grants and activities.

\bigskip

{\sc Program Committees and Professional Leadership:}

\smallskip

Program committee, Algebraic Statistics at the University of Hawaii at Manoa, June 2020, now postponed
until May 2021.

SIAM selection committee for the SIAG/Algebraic Geometry Early Career Prize, 2018.

Mathematics and the Quest for Fundamental Principles in Biology.  Group of 17
researchers funded by Army Research Office to investigate and explore 
future directions of mathematical biology.  Met at University of Utah, Dec.~2015.

SIAM Nominating committee for the SIAM Activity Group in Algebraic Geometry, 2015.

Chair Scientific Advisory Board,
NSF-CBMS Regional Research Conference in Mathematical Phylogenetics.

Program committee, 21st International Conference on
Intelligent Systems for Molecular Biology; 12th European
Conference on Computational Biology (ISMB/ECCB 2013), Evolution and
Comparative Genomics track.

Program committee, SIAM Meeting on Applications of Algebraic Geometry, Raleigh, NC, Oct 2011.

Vice Chair, SIAM's Activity Group for Algebraic Geometry -- $\text{SI(AG)}^2$, 2009 -- 2011.

Program committee, Workshop on Algorithms in Bioinformatics (WABI 2009).

Program committee, Sixteenth International Conference on
Intelligent Systems for Molecular Biology (ISMB 2008), Evolution and
Phylogeny track.

Scientific committee, Mathematics and Informatics in Evolution
and Phylogeny, Hameau de l'Etoile, France, 2008.

\bigskip

{\sc Organizer:}

\smallskip

Co-organizer `Applications of Mathematical Biology' mini-symposium at
the SIAM meeting on Applications of Algebraic Geometry, Raleigh, NC, Oct 2011.

Co-chair SAMSI's working group on Evolutionary Biology, 2008 -- 2009.

Co-organizer, with R.~Thomas of the University of Washington, AMS
special session {\sl Applications of Algebraic Geometry}, AMS
Sectional meetings, Vancouver, BC, Oct.~2008.

Co-organizer AWM Workshop for Women Graduate Students and Recent PhDs,
Joint Mathematics Meetings, San Diego, Jan.~2008; Jan.~2010.

Co-organizer, with V.~Moulton of University of East Anglia, London
Mathematical Society's Spitalfields Day at the Isaac Newton
Institute for the Mathematical Sciences.  ``Yggdrasil:
Reconstructing the Tree of Life.'' Dec.~2007.

Co-organizer, with S. Sullivant of Harvard University, Special Session
{\sl Algebraic Statistics:  Theory and Practice}, Joint Mathematics
Meetings, San Antonio, TX, Jan. 2006.

\ifthenelse{\boolean{uafCV}}{
Chair, AMS session on Field Theory and Commutative Rings, Joint
Mathematical Meetings, New Orleans, LA; Jan. 2001.}
{}

Co-organizer Project NExT session {\it What Makes a Mathematics Program
Thrive?}, Joint Mathematics Meetings, New Orleans, LA; Jan. 2001.

\medskip

{\sc Mathematics Education:}

\smallskip

{ \leftskip=.6in \parindent=-.3in  \parskip=3pt

MMSTEC, participated in Maine Mathematics and Science Teaching
Excellence Collaborative, a NSF funded grant to the State of Maine --
DUE Collaboratives for Excellence in Teacher Preparation program. Took
a leadership role in designing a secondary mathematics certification
program at USM and improving the preparation of HS mathematics
teachers.

College Board Faculty Consultant, June 1997, 1999, 2001, 2002, 2003.

\begin{comment}
Undergraduate Research/Independent Studies.  Geometry of Surfaces,
Applications of Integration in the Sciences, Introduction to Proofs,
Networking, Using {\it Mathematica} to Model Solutions to the Lorenz
Differential Equations, Senior Seminar Talks: Number Theory
and Atonality; Generating Functions; Divergence Theorem and Gauss Flux
Theorem; Magic Squares; Latin squares and the design of tournaments.
\end{comment}

\mbox{}
}

\newpage

{\sc Curricular Interests:}

{ \leftskip=.6in \parindent=-.3in  \parskip=3pt

Courses taught include: Calculus I, II, and III, Linear Algebra,
Differential Equations, Foundations, Real Analysis, Complex
Analysis, Topology, Abstract Algebra, Number Theory, Galois Theory, Algebraic
Geometry, Numerical Analysis,
Probability, Statistical Inference, Mathematical Models in Biology,
Mathematical Modeling, Synthetic Biology, Introductory Programming, An Introduction to
the Internet, Geometry for Elementary Teachers, Mathematical Methods
for High School Teachers, Capstone course, Seminar, Supervision of
Mathematics teachers.

Software Expertise: Macaulay2, Singular, MATLAB, Mathematica, Maple,
Pari, SplitsTree, Phylip, PAUP*, PhyML, IQtreee, R, C.

%\mbox{}
}

{\sc Curricular Development:}

{ \leftskip=.6in \parindent=-.3in  \parskip=3pt

Undergraduate/graduate course:  Mathematical Models in Biology

Graduate course:  Theory of Phylogenetics, Algebraic Geometry, Galois Theory

Undergraduate math and biology course:  An Introduction to Synthetic Biology

\mbox{}
}

}

%\newpage

{\bf UAF PhD  and POSTDOC ADVISING}

\medskip

{\leftskip=.6in \parindent=-.2in  \parskip=3pt

2015--19 \, Hector Ba\~nos, PhD: ``Species Network inference under the multispecies coalescent
model."

2015--20  \, Samaneh Yourdkhani (PhD):  ``Investigations in Phylogenetics:  Tree inference and model identifiability"

2017--18, 2020-21 \, Jonathan Mitchell (Post-doc)

\ifthenelse{\boolean{uafCV}}
{
\smallskip
{\sl PhD thesis committee member:}

2018--\phantom{20} Jeffrey Parks (PhD)
}{}

}




\medskip

{\bf UAF MASTER'S PROJECTS AND THESES}

\medskip

{ \leftskip=.6in \parindent=-.2in  \parskip=3pt

2019--~~  Dakota Dragomir

2017--18 ``Testing Multispecies Coalescent Simulators with Summary Statistics," 
Hector Ban\~os, (MS Statistics) [co-advisor]

2016--18 ``Reliability Analysis of Reconstructing Phylogenies under Long Branch Attraction Conditions,"
Ranjan Dissanayake  (MS Statistics)

2015--16  ``Expectation Maximization and Latent Class Models,"  (MS Mathematics) Hector Ba\~nos

\ifthenelse{\boolean{uafCV}}{2014--14 \, Emily Hill (left program)}{}

2009--11 ``An exposition on the Kronecker-Weber Theorem,"  Jason Bagett

2006--09 ``Algebraic geometry applied to phylogenetics," Beth Zirbes

2007--08 ``A software system for simulating the evolution of biological sequences under complex models," M.M.~Haque (computer science  -- informal advisor) 

\ifthenelse{\boolean{uafCV}}
{
\smallskip
{\sl Master's thesis committee member:}

\smallskip

2017- Mason Brewer (left program)

2013--16 Caleb Jurkowski

2013--16 Samantha Warren

2012--16 Andres Dajles

2012--13, 2015-- Erin Rausch VanHouten

2012--14 Vikenty Miheev

2013--14 Mark Layer

2010--14 Lyman Gillispie 

2008--13 Yuanyuan Zhao

2008-- Mike Hazlett (left program fall 2010)

2005--06 Robert Luz

\smallskip



\smallskip

{\sl Independent studies and projects:}

2012 -- Lyman Gillispie, 697: Representation theory 

2010 -- Jason Baggett, Number theory

2008 -- Rachel Krieg, undergraduate statistics project in phylogenetics
}{}

\smallskip

%\mbox{}
}

\ifthenelse{\boolean{uafCV}}
{
{\bf UAF University Service}

\medskip

{ \leftskip=.6in \parindent=-.2in  \parskip=3pt

2019-2020 Outside Examiner, Oral Comprehensive PhD Exam, Kendall Mills, Biology and Wildlife

\hskip 1.65cm Outside Examiner, Oral Comprehensive PhD Exam, Katie Rubin, Biology and Wildlife

2015--2016 Planning and Budget Committee

2014--2016 Faculty Senate CNSM Senator

2014--2016 Faculty Affairs Committee

2014 Biology and Wildlife Instructor of the Year Selection Committee

2009-2010 Chancellor's Committee on the Integration of Research and Teaching in
the Sciences (CIRTS)

2008--2010 Faculty Senate Committee on the Status of Women

2008--2009 Core Curriculum Review Committee

Department Colloquia: 2007, 2012, 2016


%\mbox{}
}
}
{}

\ifthenelse{\boolean{uafCV}}
{
{\bf Department service}

\medskip

{ \leftskip=.6in \parindent=-.2in  \parskip=3pt

B.A./B.S.~Mathematics program review 2017-2018, edited 2019-2020

Math B.A.~plan, 2015-2016

Calculus review committee, 2015-2016

Core curriculum committee, 2015-2016

B.A./B.S.~Mathematics program and special program review, 2014-2015

Alignment committee, 2014-2015

Statistics Faculty Search committee 2014-2015

Statistics Faculty Search committee 2013-2014

Curriculum Whip 2012-2015

Graduate Ph.D. Admissions committee

Graduate Ph.D.~program review

Peer Review committee

Graduate committee

Undergraduate committee

Committee assignments as needed

%\mbox{}
}
}
{}

\ifthenelse{\boolean{uafCV}}
{
{\bf Community service}

\medskip

{ \leftskip=.6in \parindent=-.2in  \parskip=3pt

\emph{Building with Biology} at the Museum of the North, synthetic biology public outreach, Aug 2016.

New Mexico State High School Mathematics Contest plenary speaker 2014.

Fairbanks home school science fair judge, Feb 2012.

\mbox{}
}
}
{}


{\bf OTHER INTERESTS}

\medskip

{ \leftskip=.6in \parindent=-.2in  \parskip=3pt

Travel, Marathon Running, Hiking, Speaking Italian, Swimming, Skiing, Music, Kids,
Good Humor

%\mbox{}
}

\end{document}
